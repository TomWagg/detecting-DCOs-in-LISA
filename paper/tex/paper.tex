\documentclass[twocolumn, twocolappendix]{aastex63}
\received{\today}
\shorttitle{Detecting DCOs with LISA}
\graphicspath{{../figures/}}

\usepackage{lipsum}
\usepackage{physics}
\usepackage{multirow}
\usepackage{xspace}
\usepackage{natbib}
\usepackage{fontawesome5}
\usepackage{xcolor} % to add more color options
\usepackage{wrapfig}

% remove indents in footnotes
\usepackage[hang,flushmargin]{footmisc} 

\newcommand{\todo}[1]{{\color{red}{[TODO: #1}]}}
\newcommand{\tom}[1]{\textcolor{ForestGreen}{[Tom: #1]}}
\newcommand{\floor}[1]{\textcolor{Bittersweet}{[Floor: #1]}}
% custom function for adding units
\makeatletter
\newcommand{\unit}[1]{%
    \,\mathrm{#1}\checknextarg}
\newcommand{\checknextarg}{\@ifnextchar\bgroup{\gobblenextarg}{}}
\newcommand{\gobblenextarg}[1]{\,\mathrm{#1}\@ifnextchar\bgroup{\gobblenextarg}{}}
\makeatother

\newcommand{\avg}[1]{\left\langle#1\right\rangle}

%% Physics variations shortcuts
\newcommand{\modFid}{A}
\newcommand{\modBetaLow}{B}
\newcommand{\modBetaMed}{C}
\newcommand{\modBetaHigh}{D}
\newcommand{\modCaseBB}{E}
\newcommand{\modCaseBBOpt}{F}
\newcommand{\modAlphaLowest}{G}
\newcommand{\modAlphaLow}{H}
\newcommand{\modAlphaHigh}{I}
\newcommand{\modAlphaHighest}{J}
\newcommand{\modOpt}{K}
\newcommand{\modRapid}{L}
\newcommand{\modNSLow}{M}
\newcommand{\modNSHigh}{N}
\newcommand{\modNoPISN}{O}
\newcommand{\modSigLow}{P}
\newcommand{\modSigLower}{Q}
\newcommand{\modNoBH}{R}
\newcommand{\modWRLow}{S}
\newcommand{\modWRHigh}{T}

\newcommand{\modRangeMT}{B-F}
\newcommand{\modRangeCE}{G-K}
\newcommand{\modRangeSN}{L-R}
\newcommand{\modRangeML}{S-T}

\newcommand{\nModels}{20}
\newcommand{\nMinusOneModels}{19}

\newcommand{\BHBHFourYear}{74}
\newcommand{\BHNSFourYear}{42}
\newcommand{\NSNSFourYear}{8}
\newcommand{\BHBHTenYear}{118}
\newcommand{\BHNSTenYear}{71}
\newcommand{\NSNSTenYear}{13}

% percentage of systems in lower mass gap
\newcommand{\BHBHLowerMassGap}{35\%}
\newcommand{\BHNSLowerMassGap}{39\%}

% percentage of systems with q > 0.8
\newcommand{\BHBHqAbovePointEight}{8\%}
\newcommand{\BHNSqAbovePointEight}{0\%}
\newcommand{\NSNSqAbovePointEight}{90\%}

% percentage of systems with e > 0.01
\newcommand{\BHBHNotCirc}{87\%}
\newcommand{\BHNSNotCirc}{46\%}
\newcommand{\NSNSNotCirc}{86\%}

% percentage of systems with e > 0.3
\newcommand{\BHBHHighlyEccentric}{21\%}
\newcommand{\BHNSHighlyEccentric}{8\%}
\newcommand{\NSNSHighlyEccentric}{12\%}

% number of systems with fractional chirp mass uncertainty < 1
\newcommand{\BHBHChirpMassMeasureableFour}{20}
\newcommand{\BHNSChirpMassMeasureableFour}{17}
\newcommand{\NSNSChirpMassMeasureableFour}{4}
\newcommand{\BHBHChirpMassMeasureableTen}{52}
\newcommand{\BHNSChirpMassMeasureableTen}{44}
\newcommand{\NSNSChirpMassMeasureableTen}{9}

% percentage of systems above Chandrasekhar mass
\newcommand{\BHBHAboveMaxWDWDFourPerc}{24}
\newcommand{\BHNSAboveMaxWDWDFourPerc}{28}
\newcommand{\NSNSAboveMaxWDWDFourPerc}{4}
\newcommand{\BHBHAboveMaxWDWDTenPerc}{38}
\newcommand{\BHNSAboveMaxWDWDTenPerc}{41}
\newcommand{\NSNSAboveMaxWDWDTenPerc}{5}

% percentage of systems with something other than 1 detectable harmonic
\newcommand{\BHBHMultipleHarmonicsFourPerc}{55}
\newcommand{\BHNSMultipleHarmonicsFourPerc}{27}
\newcommand{\NSNSMultipleHarmonicsFourPerc}{66}
\newcommand{\BHBHMultipleHarmonicsTenPerc}{61}
\newcommand{\BHNSMultipleHarmonicsTenPerc}{29}
\newcommand{\NSNSMultipleHarmonicsTenPerc}{68}

% percentage of systems with >1 harmonic and in disc
\newcommand{\BHBHEccInDiscFourPerc}{40}
\newcommand{\BHNSEccInDiscFourPerc}{23}
\newcommand{\NSNSEccInDiscFourPerc}{59}
\newcommand{\BHBHEccInDiscTenPerc}{40}
\newcommand{\BHNSEccInDiscTenPerc}{23}
\newcommand{\NSNSEccInDiscTenPerc}{59}

% number of systems that are definitely not WDWDs
\newcommand{\BHBHNotWDWDFour}{37}
\newcommand{\BHNSNotWDWDFour}{18}
\newcommand{\NSNSNotWDWDFour}{5}
\newcommand{\BHBHNotWDWDTen}{70}
\newcommand{\BHNSNotWDWDTen}{38}
\newcommand{\NSNSNotWDWDTen}{8}

% number of systems that can be determined as either a BH or NS
\newcommand{\BHBHEitherBHOrNSFour}{16}
\newcommand{\BHNSEitherBHOrNSFour}{7}
\newcommand{\NSNSEitherBHOrNSFour}{4}
\newcommand{\BHBHEitherBHOrNSTen}{39}
\newcommand{\BHNSEitherBHOrNSTen}{17}
\newcommand{\NSNSEitherBHOrNSTen}{8}
\newcommand{\BHBHEitherBHOrNSFourPerc}{21}
\newcommand{\BHNSEitherBHOrNSFourPerc}{18}
\newcommand{\NSNSEitherBHOrNSFourPerc}{47}
\newcommand{\BHBHEitherBHOrNSTenPerc}{33}
\newcommand{\BHNSEitherBHOrNSTenPerc}{24}
\newcommand{\NSNSEitherBHOrNSTenPerc}{62}

\newcommand{\BHBHatLeastOneLowerMassGapPerc}{69}
\newcommand{\BHNSatLeastOneLowerMassGapPerc}{39}
\newcommand{\NSNSatLeastOneLowerMassGapPerc}{0}
\newcommand{\ALLatLeastOneLowerMassGapPerc}{55}

\newcommand{\confinv}[3]{$#1${\raisebox{0.5ex}{\tiny$_{-#2}^{+#3}$}}}

\definecolor{Blush}{rgb}{0.87, 0.36, 0.51}
\newcommand{\SdM}[1]{{\color{Blush}{#1}}}

% Milky Way notation
\newcommand{\achem}{\ensuremath{[\alpha/\mathrm{Fe}]}}

\begin{document}

\title{{\large Gravitational wave sources in our Galactic backyard:}\\{\normalsize Predictions for BHBH, BHNS and NSNS binaries in LISA}}

% affiliations
\newcommand{\cfa}{Center for Astrophysics | Harvard \& Smithsonian, 60 Garden Street, Cambridge, MA 02138, USA}
\newcommand{\mpa}{Max-Planck-Institut für Astrophysik, Karl-Schwarzschild-Straße 1, 85741 Garching, Germany}
\newcommand{\cca}{Center for Computational Astrophysics, Flatiron Institute, 162 Fifth Ave, New York, NY, 10010, USA}
\newcommand{\UvA}{Anton Pannekoek Institute for Astronomy \& Grappa, University of Amsterdam, Postbus 94249, 1090 GE Amsterdam, The Netherlands}

\author[0000-0001-6147-5761]{T. Wagg}
\affiliation{\cfa}
\affiliation{\mpa}

\author[0000-0002-4421-4962]{F.S. Broekgaarden}
\affiliation{\cfa}

\author[0000-0001-9336-2825]{S.E. de Mink}
\affiliation{\mpa}
\affiliation{\UvA}
\affiliation{\cfa}

\author[0000-0002-6411-8695]{N. Frankel}
\affiliation{Max Planck Institute for Astronomy, Königstuhl 17, D-69117 Heidelberg, Germany}

\author[0000-0001-5484-4987]{L.A.C. van Son}
\affiliation{\cfa}
\affiliation{\UvA}
\affiliation{\mpa}

\author[0000-0001-7969-1569]{S. Justham}
\affiliation{School of Astronomy \& Space Science, University of the Chinese Academy of Sciences, Beijing 100012, China}
\affiliation{\UvA}
\affiliation{\mpa}

\correspondingauthor{Tom Wagg}
\email{tomjwagg@gmail.com}

\begin{abstract}
    We present predictions for the main properties of the LISA detectable population of Galactic double black holes (BHBH), black hole neutron stars (BHNS) and double neutron stars (NSNS). For calculating these predictions, we use an extensive sample of double compact objects (DCOs) produced using rapid population synthesis in tandem with an empirically-informed analytical model of the Milky Way that accounts for the chemical enrichment history of the galaxy. We investigate the dependence of our results upon underlying physics assumptions by comparing the results of \nModels{} physics variations that vary assumptions relating to mass transfer, common envelope, supernova and wind mass loss physics. We find that for a 4(10)-year mission, LISA will detect on average \BHBHFourYear{}(\BHBHTenYear{}) BHBHs, \BHNSFourYear{}(\BHNSTenYear{}) BHNSs, \NSNSFourYear{}(\NSNSTenYear{}) NSNSs. The BHBH rate remains notably consistent under different physics assumptions, whilst in contrast the BHNS and NSNS rates each vary over 3 orders of magnitude. We discuss potential strategies for distinguishing this population from the more numerous double white dwarf population as well as for separating the BHBH, BHNS and NSNS populations from each other. We additionally assess the possibility of joint SKA-LISA detections for systems that contain pulsars.
\end{abstract}

\keywords{gravitational waves, LISA, black hole, neutron star, binary}

\section{Introduction} \label{sec:intro}
Since the first direct detection of gravitational waves by the LIGO scientific collaboration \citep{Abbott+2016_first_detection}, the number of black hole (BH) binaries and neutron star (NS) binaries detected by ground-based detectors has rapidly grown \citep{Abbott+2019_GWTC1,Abbott+2020_GWTC2}. These detections offer exciting insights into the evolution and endpoints of massive stars. The investigation of double compact object (DCO) population statistics will be an essential tool for predicting distributions of parameters that are observable with gravitational waves as well as constraining uncertainties in binary evolution. 

The Laser Interferometer Space Antenna (LISA, \citealp{Amaro-Seoane+2017}) will provide observations in an entirely new regime of gravitational waves. LISA will observe binaries at lower orbital frequencies than ground-based detectors ($10^{-5} \lesssim f / \unit{Hz} \lesssim 10^{-1}$) and so will enable the study of the gravitational wave mergers of supermassive black holes that are undetectable with ground-based detectors. However, this frequency regime is also of great interest for the detection of local stellar mass binaries during their inspiral phase. This means that LISA will be able to detect, and possibly localise a binary on the sky, far in advance of the merger, which presents an opportunity for both multimessenger detections to search for electromagnetic counterparts and multiband detections that would better constrain binary characteristics \citep[e.g.][]{Sesana+2016, Gerosa+2019}. In addition, DCOs may still have significant eccentricity in the LISA band and measurements of eccentricity may yield further constraints on binary evolution \citep[e.g.][]{Nelemans+2001, Breivik+2016, Antonini+2017, Rodriguez+2018}, differentiate between formation channels and distinguish between DCO types. The maximum distance at which stellar mass sources in LISA are detectable is significantly lower than in ground-based detectors since the gravitational wave signal is weaker during the inspiral phase than at the merger. This means that LISA stellar mass sources can only be detected in local galaxies, with the majority residing in the Milky Way. Therefore, these sources could be used as a probe for our galaxy's history and evolution \citep[e.g.][]{Korol+2019}.

Traditionally, investigations into detecting stellar mass sources with LISA focus on double white dwarf (WDWD) binaries \citep{Nelemans+2001,Ruiter+2010,Yu+2010,Nissanke+2012,Korol+2017,Lamberts+2018}. More recently, interest has grown in the detection of NS and BH binaries. Although these sources are rare, they could be useful for learning more about the evolution and endpoints of massive stars and thus we focus our paper on these sources.

For the purposes of this investigation, we consider only the classical isolated binary evolution channel \citep[e.g.][]{Tutukov+1973,Tutukov+1993,Kalogera+2007,Belczynski+2016} in which compact objects are formed through through highly non-conservative mass transfer or common envelope ejection. There are however several alternative proposed formation channels including: dynamical formation in dense star clusters \citep[e.g.][]{Sigurdsson+1993,PortegiesZwart+2000,Miller+2009,Rodriguez+2015} and (active) galactic nuclei discs \citep[e.g.][]{Morris+1993, Antonini+2016, McKernan+2020}, isolated hierarchical triple evolution involving Kozai-Lidov oscillations \citep[e.g.][]{Stephan+2016, Silsbee+2017,Antonini+2017} and chemically homogenous evolution through efficient rotational mixing \citep[e.g.][]{deMink+2009, deMink+2016,Marchant+2016,duBuisson+2020}.

Galactic double neutron star (NSNS) binaries have been observed with electromagnetic signals for several decades \citep[e.g.][]{Hulse+1975} and more recently the mergers of NSNS binaries with ground-based gravitational wave detectors have been observed \citep[e.g.][]{Abbott+2017_NSNS}. The detection of a NSNS in LISA in which at least one NS is a pulsar could connect these two populations as the binary could be observed from inspiral to merger. NSNS binaries are useful sources for understanding the origin of r-process elements \citep[e.g.][]{Eichler+1989} as well as the electromagnetic counterparts to gravitational wave signal such as kilonovae \citep[e.g.][]{Metzger+2017}, short gamma-ray bursts \citep[e.g.][]{Gompertz+2020}, radio emission \citep[e.g.][]{Hotokezaka+2016} and neutrinos \citep[e.g.][]{Kyutoku+2018}.

Double black hole (BHBH) binaries in the Milky Way present a greater observational challenge. To date, no BH has been observed to be in a binary with another compact object in the Milky Way and so LISA could provide the first detection of a Galactic BHBH binary. Moreover, the mass distribution of stellar mass black holes is still uncertain. The only confirmed BHs in our galaxy have been discovered as components of X-ray binaries with companion stars \citep[e.g.][]{Bolton+1972,Webster+1972}. This sample of BHs has masses mainly constrained between $5$ and $10 \unit{M_\odot}$ \citep{Corral-Santana+2016}, a stark contrast to the more massive BHs observed with LIGO/Virgo that tend to have masses concentrated around $30 \unit{M_{\odot}}$ \citep{Abbott+2020_GWTC2}. These observations indicate the presence of a lower mass gap (from $2$-$5 \unit{M_{\odot}}$) in which no black holes or neutron stars are observed \citep{Ozel+2010,Farr+2011} but its existence remains an open question \citep[e.g.][]{Woosley+2020}. Recently there has also been increased discussion over the maximum BH mass in our galaxy, with the claims of a $70 \unit{M_{\odot}}$ BH \citep{Liu+2019,Abdul-Masih+2020} and revised measurements of the mass of Cygnus X-1 \citep{Miller-Jones+2021}. A sample of BHBHs detected with LISA could possibly help to constrain these uncertainties in the stellar mass BH mass distribution.

One particularly interesting and elusive gravitational wave source is a black hole neutron star binary (BHNS). Of all the events detected by ground-based detectors, none can be confidently attributed to the merger of a black hole and a neutron star, though several events such as GW190425 and GW190814 have not been ruled out as a BHNS merger \citep{Abbott+2020_GW190425,Abbott+2020_GW190814}. Predictions for the merger rate of BHNSs range across three orders of magnitude \citep[e.g.][]{Abadie+2010, Broekgaarden+2021} so the number of detections in LISA will be important in reducing this uncertainty, thereby refining our understanding of the remnants and evolution of massive stars. These binaries are expected to have electromagnetic counterparts that can studied in the same way as NSNSs. A distinctly exciting possibility is the detection of a pulsar--BH system or millisecond pulsar--BH system \citep{Narayan+1991}. These systems could be observed not only by gravitational wave detectors, but also radio telescopes such as MeerKAT and SKA, which will help to constrain uncertain binary evolution processes \citep[e.g.][]{Pfahl+2005,Chattopadhyay+2020}.

The detection of DCOs with LISA has been investigated in many previous studies through a combination of population synthesis and Milky Way modelling. Previous studies that investigate BHBH, BHNS and NSNS binaries, as opposed to the numerous WDWD population, are still uncommon. Earlier work has used a variety of population synthesis codes, Milky Way models and LISA specifications, resulting in a wide range of predictions \citep{Nelemans+2001,Liu+2009,Belczynski+2010,Liu+2014,Lamberts+2019,Lau+2020,Breivik+2020,Sesana+2020}.

We build upon previous efforts but with several important improvements. We explore the effect of varying binary physics assumptions by repeating our analysis for 15 different models and comparing the effect on the detection rate and distributions of source parameters. We use a model for the Milky Way that is dependent on the chemical enrichment history and calibrated on the latest GAIA and APOGEE surveys \citep{GaiaCollaboration+2016,Majewski+2017,Frankel+2018}. In contrast to many previous works, we provide a full treatment of the eccentricity of detectable sources both for the inspiral evolution as well as gravitational wave signal during the LISA mission. Moreover, our binary population synthesis simulation is the most extensive of its kind to date, with 750 million simulated binaries (one million binaries for each of 50 metallicity bins and 15 physics variations) evolved to produce the DCO populations used in this work \citep{Broekgaarden+2021}. In addition, we use the adaptive sampling algorithm STROOPWAFEL \citep{Broekgaarden+2019} to further reduce our sampling noise.

In this paper, we present the most extensive simulations to date for predictions of the detection rate and distribution of binary properties (masses, frequency, eccentricity, distance, merger time) of BHBH, BHNS and NSNS binaries formed through isolated binary evolution in the Milky Way. We explore 15 different models of physical assumptions in our population synthesis model and how the changes in these assumptions alter our results. We also discuss the effect of extending the LISA mission length and the possibility of distinguishing detections.

Our paper is structured as follows. In Section~\ref{sec:method}, we describe our methods for synthesising a population of binaries, the variations of physical assumptions that we consider, how we simulate the Milky Way distribution of DCOs and our methods for calculating a detection rate for LISA. We present our main results in Section~\ref{sec:results}, analysing our findings for each DCO type and variation of physical assumptions. In Section~\ref{sec:discussion} we discuss these results. In Section~\ref{sec:compare_studies}, we compare and contrast our methods and findings to previous work and finish with our conclusions in Section~\ref{sec:conclusion}.

All data produced in this study is publicly available on Zenodo \href{https://doi.org/10.5281/zenodo.4574727}{\faFileCode}\footnote{\dots} \todo{add link} as is the population used in our simulations \href{https://doi.org/10.5281/zenodo.4574727}{\faFileCode}\footnote{\url{https://doi.org/10.5281/zenodo.4574727}}. We make all code used to produce our results available in a Github repository \href{https://github.com/TomWagg/detecting-DCOs-in-LISA}{\faGithub}\footnote{\url{https://github.com/TomWagg/detecting-DCOs-in-LISA}}. In addition, the repository contains step-by-step Jupyter notebooks that explain how to reproduce and change each figure in the paper. In a companion paper we present \href{https://legwork.readthedocs.io}{\texttt{LEGWORK}}\footnote{\url{https://legwork.readthedocs.io}}, a python package designed for making predictions for the detection of sources with LISA, which we use in this work.

\section{Method} \label{sec:method}
To produce predictions for the DCOs that are detectable with LISA, we use a synthesised population of DCOs, simulated using the methods described in Section~\ref{sec:COMPAS_explained}. In Section~\ref{sec:galaxy_synthesis} we describe our model for the Milky Way and how we place DCOs in randomly sampled Milky Way instances. We evolve the orbit of each DCO in a Milky Way instance up to the LISA mission and calculate the detection rate for that instance using the methods presented in Section~\ref{sec:gw_detection}.

\subsection{Binary population synthesis}\label{sec:COMPAS_explained}

We use the grid of \nModels{} binary population synthesis simulations recently presented in \citet{Broekgaarden+2021,Broekgaarden+2021b}. This grid of simulations is synthesised using the rapid population synthesis code \href{https://compas.science}{COMPAS} \citep{Stevenson+2017,Vigna-Gomez+2018,Stevenson+2019,Broekgaarden+2019}. COMPAS follows the approach of the population synthesis code BSE \citep{Hurley+2000,Hurley+2002} and uses fitting formula and rapid algorithms to efficiently predict the final fate of binary systems. The code is open source and documented in the papers listed above, the online documentation\footnote{\url{https://compas.science}} and in the methods paper \citep{COMPAS:2021methodsPaper}. \edit1{The model we refer to as our fiducial model allows for BHs with masses in the heavily debated mass gap, but we also consider \nModels{} physics variations. \citet{Broekgaarden+2021b} showed that all of these physics variations are still consistent with current constraints on the overall inferred GW rates, at least when considering large variations of the assumed metallicity specific cosmic star formation rate \citep{Abbott+2021_GWTC3}}. We summarise the main assumptions and settings relevant for this work in Appendix~\ref{app:pop_synth}.

The result of the simulations is a sample of binaries, which, for each metallicity $Z$, have $N_{\rm binaries}$ binaries with parameters
\begin{equation}
    \mathbf{b}_{{Z, i}} = \{m_1, m_2, a_{\rm DCO}, e_{\rm DCO}, t_{\rm evolve}, t_{\rm inspiral}, w\},
\end{equation}
for $i = 1, 2, \dots, N_{\rm binaries}$, where $m_1$ and $m_2$ are the primary and secondary masses, $a_{\rm DCO}$ and $e_{\rm DCO}$ are the semi-major axis and eccentricity at the moment of double compact object (DCO) formation, $t_{\rm evolve}$ is the time between the binary's zero-age main sequence and DCO formation, $t_{\rm inspiral}$ is the time between DCO formation (that is, immediately after the second supernova in the system) and gravitational-wave merger and $w$ is the adaptive importance sampling weight assigned by STROOPWAFEL \cite[][Eq.~7]{Broekgaarden+2019}. We sample from these sets of parameters when creating synthetic galaxies.

\subsection{Galaxy synthesis}\label{sec:galaxy_synthesis}

In order to estimate a detection rate of DCOs with statistical uncertainties, we create a series of random instances of the Milky Way, each populated with a subsample drawn (with replacement) from the synthesised binaries described in Section~\ref{sec:COMPAS_explained}.

Most previous studies that predict a detection rate for LISA place binaries in the Milky Way independently of their age or evolution. We improve upon this as the first study to use an empirically-informed analytical model of the Milky Way that takes into account the galaxy's enrichment history by applying the metallicity-radius-time relation from \citet{Frankel+2018}. Those authors developed this relation in order to measure the global efficiency of radial migration in the Milky Way and calibrated it using a sample of red clump stars measured with APOGEE \citep{Majewski+2017}. \edit1{We assess the impact of using this improved Milky Way model in Appendix~\ref{app:mw_changes} and the effect of Galactic models on LISA predictions has been investigated more generally in \citet{Storck+2022}}.

In Section~\ref{sec:mw_model}, we outline our model for the Milky Way and in Section~\ref{sec:combining_pop_gal} we explain how we combine our population of synthesised DCOs with this Milky Way model.

\subsubsection{Milky Way model}\label{sec:mw_model}

\begin{figure*}[t]
    \centering
    \includegraphics[width=\textwidth]{fig1_galaxy_diagram.png}
    \caption{A schematic illustrating how we model the Milky Way. The left panel illustrates the different model aspects: star formation history of three galactic components (individually shown in the dotted lines), radial distribution, metallicity-radius-time relation, and height distribution. The right panel shows an example instance of the Milky Way with $250000$ binaries shown as points coloured by metallicity. The top panel shows a side-on view and the bottom panel a face-on view. \href{https://github.com/TomWagg/detecting-DCOs-in-LISA/blob/main/paper/figures/fig1_galaxy_diagram.png}{\faFileImage} \href{https://github.com/TomWagg/detecting-DCOs-in-LISA/blob/main/paper/figure_notebooks/galaxy_creation_station.ipynb}{\faBook}.}
    \label{fig:galaxy_schematic}
\end{figure*}

Fig.~\ref{fig:galaxy_schematic} shows the distributions and relations outlined in this section and also displays an example random galaxy drawn using this model.

Our model for the Milky Way accounts for the low-\achem\footnote{Nomenclature used to describe the enhancement of $\alpha$ elements compared to iron in stellar atmospheres}~disc, high-\achem~disc and a central component approximating a bulge/bar. The low- and high-\achem~discs are often also referred to as the thin and thick discs because the stellar vertical distribution is better fit by a double exponential rather than a single one. However, this doesn't allow one to assign a star to either the thin or thick disk purely based on its height above the Galactic plane. Therefore, we instead use the chemical definition of the two disks (applying the \achem~nomenclature) as there is a clear bimodal distribution in the chemical plane, allowing stars to be assigned to each of the disc components based on their chemical abundances. For each of the three components, we use a separate star formation history and spatial distribution, which we combine into a single model, weighting each component by its present-day stellar mass. \citet{Licquia+2015} gives that the stellar mass of the bulge is $0.9 \times 10^{10} \unit{M_{\odot}}$ and the stellar mass of the disc is $5.2 \times 10^{10} \unit{M_\odot}$, which we split equally between the low- and high-\achem~discs \citep[e.g.,][]{Snaith+2014}.

\textit{Star formation history:} 
We use an exponentially declining star formation history \citep{Frankel+2018} (inspired by the average cosmic star formation history) for the combined low- and high-\achem~discs,
\begin{equation}\label{eq:thin_disc_tau}
    p(\tau) \propto \exp \qty(-\frac{(\tau_m - \tau)}{\tau_{\rm SFR}}),
\end{equation}
where $\tau$ is the lookback time (the amount of time elapsed between the binary's zero-age main sequence and today), $\tau_m = 12 \unit{Gyr}$ is the assumed age of the Milky Way and $\tau_{\rm SFR} = 6.8 \unit{Gyr}$ is the star formation timescale \citep{Frankel+2018}. The two discs form stars in mutually exclusive time periods, such that the high-\achem~disc forms stars in the early history of the galaxy ($8$--$12 \unit{Gyr}$ ago) and the low-\achem~disc forms stars more recently ($0$--$8 \unit{Gyr}$ ago). Both distributions are normalised so that an equal amount of mass is formed in each of the two components over their respective star forming periods.

The star formation history of the bulge/bar of the Milky Way has many uncertainties due to the (1) sizeable age measurement uncertainties at large ages in observational studies, (2) complex selection processes affecting the observed age distributions, and (3) formation mechanisms that are still under debate. However, the central bar (which we assume to dominate here) was shown to contain stars with an extended age range, with most observed stars between $6$ and $12 \unit{Gyr}$ with a younger tail of ages that could come from the subsequent secular growth of the Galactic bar \citep[e.g.,][]{Bovy+2019}. To model the bulge/bar's age distribution more realistically than in previous studies (which assume an old bulge coming from a single starburst), we choose to adopt a more extended star formation history using a $\beta(2,3)$ distribution, shifted and scaled such that stars are only formed in the range $[6, 12] \unit{Gyr}$. We show these distributions in the top left panel of Fig.~\ref{fig:galaxy_schematic}.

\textit{Radial distribution:} For each of the three components we employ the same single exponential distribution (but with different scale lengths)
\begin{equation}\label{eq:galaxy_R}
    p(R) = \exp(-\frac{R}{R_d}) \frac{R}{R_d^2},
\end{equation}
where $R$ is the Galactocentric radius and $R_d$ is the scale length of the component. For the low-\achem~disc, we set $R_d = R_{\rm exp}(\tau)$, where $R_{\rm exp}(\tau)$ is the scale length presented in \citet[][Eq.~6]{Frankel+2018}
\begin{equation}
    R_{\rm exp}(\tau) = 4 \unit{kpc} \qty(1 - \alpha_{R_{\rm exp}} \qty(\frac{\tau}{8 \unit{Gyr}})),
\end{equation}
where $\alpha_{R_{\rm exp}} = 0.3$ is the inside-out growth parameter\footnote{We find that $R_{\rm exp}(\tau) = 4$ kpc fits the data well and adopt this value rather than the 3 kpc quoted in \cite{Frankel+2018}, which was a fixed parameter (not a fit).}

This scale length accounts for the inside-out growth of the low-\achem~disc and hence is age dependent. We assume $R_d = (1 / 0.43) \unit{kpc}$ for the high-\achem~disc \citep[][Table~1]{Bovy+2016} and $R_d = 1.5 \unit{kpc}$ for the bulge/bar component \citep{Bovy+2019}. \edit1{Note that in this way we have approximated the bulge/bar component as being axi-symmetric, which is sufficient for our purposes.} We show the combination of these distributions in the second panel on the left in Fig.~\ref{fig:galaxy_schematic}.

\textit{Vertical distribution}: Similar to the radial distribution, we use the same single exponential distribution (but with different scale heights) for each component, given by
\begin{equation}\label{eq:galaxy_z}
    p(\abs{z}) = \frac{1}{z_d} \exp\qty(-\frac{z}{z_d}),
\end{equation}
where $z$ is the vertical displacement above the Galactic plane and $z_d$ is the scale height. We set $z_d = 0.3 \unit{kpc}$ for the low-\achem~disc \citep{McMillan+2011} and $z_d = 0.95 \unit{kpc}$ for the high-\achem~disc \citep{Bovy+2016}. For the bulge/bar, we set $z_d = 0.2 \unit{kpc}$ \citep{Wegg+15}. We show the combination of these distributions in the bottom left panel of Fig.~\ref{fig:galaxy_schematic}.

\textit{Metallicity-radius-time relation:} To account for the chemical enrichment of star forming gas as the Milky Way evolves, we adopt the relation given by \citep[][Eq. 7]{Frankel+2018}
\begin{equation}\label{eq:galaxy_FeH}
    \begin{split}
        [{\rm Fe} / {\rm H}] (R, \tau) &= F_m + \nabla [{\rm Fe} / {\rm H}] R \\
        &- \qty(F_m + \nabla [{\rm Fe} / {\rm H}] R^{\rm now}_{[{\rm Fe} / {\rm H}] = 0} ) f(\tau),
    \end{split}
\end{equation}
where
\begin{equation}
    f(\tau) = \qty(1 - \frac{\tau}{\tau_m})^{\gamma_{[{\rm Fe} / {\rm H}]}},
\end{equation}
$F_m = -1 \unit{dex}$ is the metallicity of the gas at the center of the disc at $\tau = \tau_m$, $\nabla [{\rm Fe} / {\rm H}] = -0.075 \unit{kpc^{-1}}$ is the metallicity gradient, $R^{\rm now}_{[{\rm Fe} / {\rm H}] = 0} = 8.7 \unit{kpc}$ is the radius at which the present day metallicity is solar and $\gamma_{[{\rm Fe} / {\rm H}]} = 0.3$ sets the time dependence of the chemical enrichment. We convert this to the representation of metallicity that we use in this paper by applying \citep[e.g.][]{Bertelli+1994}
\begin{equation}\label{eq:galaxy_FeH_to_Z}
    \log_{10} (Z) = 0.977 [{\rm Fe} / {\rm H}] + \log_{10}(Z_\odot).
\end{equation}

Although \citet{Frankel+2018} only fit this model for the low-\achem~disc, we also use this metallicity-radius-time relation for the high-$\alpha$ disc and the bar, but focusing on the chemical tracks more representative to the inner disc and large ages. \citet{Sharma+2020} showed that using a simple continuous model for both the low- and high-\achem~discs, the Milky Way abundance distributions could be well reproduced. Empirically, the abundance tracks in the [$\alpha$/Fe]-[Fe/H] plane (and other elements) of the stars in the bulge/bar follow the same track as those of the old stars in the Solar neighbourhood \citep[][Fig.~7,]{Griffith+2021,Bovy+2019}, which motivates our modelling choice to use the same metallicity-radius-time relation.

\subsubsection{Combining population and galaxy synthesis}\label{sec:combining_pop_gal}

For each Milky Way instance, we randomly sample the following set of parameters
\begin{equation}
    \mathbf{g}_{{j}} = \{\tau, R, Z, z, \theta\}
\end{equation}
for $j = 1, 2, \dots, N_{\rm MW}$, where we set $N_{\rm MW} = 2 \times 10^{5}$, $\tau, R, Z$ and $z$ are defined and sampled using the distribution functions specified in Section~\ref{sec:mw_model}, $\theta$ is the azimuthal angle sampled uniformly on $[0, 2\pi)$ and $Z$ is the metallicity. Fig.~\ref{fig:galaxy_schematic} shows an example of a random Milky Way instance created with these distributions. This shows how these distributions translate to positions and illustrates the gradient in metallicity over radius.

We match each set of galaxy parameters $\mathbf{g}_{{j}}$, to a random set of binary parameters $\mathbf{b}_{{Z, i}}$, by drawing a binary from the closest metallicity bin to the metallicity in $\mathbf{g}_{{j}}$. If the metallicity in $\mathbf{g}_{{j}}$ is below the minimum COMPAS metallicity bin ($Z = 10^{-4}$), we use this minimum bin. If the metallicity in $\mathbf{g}_{{j}}$ is above the maximum COMPAS metallicity bin ($Z = 0.03$), we use a randomly selected bin from the five highest metallicity bins\footnote{\edit1{We spread the binaries over the five highest bins, rather than just the highest bin, as we found that using a single bin led to unphysical artifacts in our results. These artifacts arose because the small population of binaries in the highest bin were oversampled.}}.

Each binary is likely to move from its birth orbit. Although all stars in the Galactic disc experience radial migration \citep{Sellwood+2002, Frankel+2018}, DCOs generally experience stronger dynamical evolution as a result of the effects of both Blaauw kicks \citep{Blaauw+1961} and natal kicks \citep[e.g.][]{Hobbs+2005}.

The magnitude of the systemic kicks are typically small compared to the initial circular velocity of a binary at each Galactocentric radius. Therefore, we expect that kicks will not significantly alter the overall distribution of their positions (see however, e.g., \citealt{Brandt+1995, Abbott+2017_GW170817_progenitor}). Given this, and for the sake of computational efficiency, we do not account for the displacement due to systemic kicks in our analysis.

\subsection{Gravitational wave detection}\label{sec:gw_detection}
We use the Python package \href{https://legwork.readthedocs.io/en/latest/}{LEGWORK} \citep{Wagg+2021} to evolve binaries and calculate their LISA detectability. For a full derivation of the equations given below see \citep[][Section~3]{Wagg+2021}, or the LEGWORK documentation \href{https://legwork.readthedocs.io/en/latest/notebooks/Derivations.html}{\faBook}.

\subsubsection{Inspiral evolution}

Each binary loses orbital energy to gravitational waves throughout its lifetime. This causes the binary to shrink and circularise over time. In order to assess the detectability of a binary, we need to know its eccentricity and frequency at the time of the LISA mission. For each binary in our simulated Milky Way, we know that the time from DCO formation to today is $\tau - t_{\rm evolve}$ and that the initial eccentricity and semi-major axis are $e_{\rm DCO}$ and $a_{\rm DCO}$. We find the eccentricity of the binary at the start of the LISA mission, $e_{\rm LISA}$, by numerically integrating its time derivative \citep[][Eq. 5.13]{Peters+1964} given the initial conditions. This can be converted to the semi-major axis at the start of LISA, $a_{\rm LISA} $\citep[][Eq. 5.11]{Peters+1964}, which in turn gives the orbital frequency, $f_{\rm orb, LISA}$, by Kepler's third law since we know the component masses.

\subsubsection{Binary detectability}

We define a binary as detectable if its gravitational wave signal has a signal-to-noise ratio (SNR) of greater than 7 \edit1{by the end of the LISA mission} \citep[e.g.][]{Breivik+2020, Korol+2020}. The sky-, polarisation- and orientation-averaged signal-to-noise ratio, $\rho$, of an inspiraling binary can be calculated with the following \citep[e.g.][]{Finn+2000}
\begin{equation}\label{eq:snr}
    \rho^2 = \sum_{n=1}^{\infty} \int_{f_{n, i}}^{f_{n, f}} \frac{h_{c, n}^{2}}{f_{n}^{2} S_{\rm n}\left(f_{n}\right)} \dd{f_n},
\end{equation}
where $n$ is a harmonic of the gravitational wave signal, $f_n = n \cdot f_{\rm orb}$ is the frequency of the $n^{\rm th}$ harmonic of the gravitational wave signal, $f_{\rm orb}$ is the orbital frequency, $S_{\rm n}(f_n)$ is the LISA sensitivity curve at frequency $f_n$ \citep[e.g.][]{Robson+2019} and $h_{c,n}$ is the characteristic strain of the $n^{\rm th}$ harmonic, given by \citep[e.g.][]{Barack+2004}
\begin{equation}\label{eq:charstrain}
    h^2_{c,n} = \frac{2^{5/3}}{3 \pi^{4/3}} \frac{(G \mathcal{M}_c)^{5/3}}{c^3 D_L^2} \frac{1}{f_{\rm orb}^{1/3}} \frac{g(n,e)}{n F(e)},
\end{equation}
where $D_L$ is the luminosity distance to the source, $f_{\rm orb}$ is the orbital frequency, $g(n, e)$ and $F(e)$ are given in \citet{Peters+1963} and $\mathcal{M}_c$ is the chirp mass, defined as
\begin{equation}\label{eq:chirp_mass}
    \mathcal{M}_c = \frac{(m_1 m_2)^{3/5}}{(m_1 + m_2)^{1/5}}.
\end{equation}

Note that increasing the length of the LISA mission allows more time for a DCO to evolve over the mission. Therefore the frequency limits in Eq.~\ref{eq:snr} are dictated by the LISA mission length. The SNR generally scales as $\sqrt{T_{\rm obs}}$ (with exceptions for sources very close to merging) and thus the SNR of a typical source in a 10-year LISA mission is approximately $1.58$ (=$\sqrt{10/4}$) times stronger than in a 4-year mission.

We use \href{https://legwork.readthedocs.io/en/latest/}{LEGWORK} \citep{Wagg+2021} to calculate the signal-to-noise ratio for each binary and the package ensures that enough harmonics are computed for each binary such that the error on the gravitational-wave luminosity remains below 1\%.

\subsubsection{Detection rate calculation}
For each physics variation model and DCO type, we first convert the COMPAS simulation results into a total number of DCOs in the Milky Way, $N_{\rm DCO}$. We do this by integrating the full mass and period distributions and stars and normalising to the total Milky Way mass. For more details see Appendix~\ref{app:rate_normalisation}.

We then determine the fraction of binaries that are detectable in each Milky Way instance by summing the adaptive importance sampling weights of the binaries that have an SNR greater than 7, and dividing by the total weights in the simulation. We multiply this fraction by $N_{\rm DCO}$ to find a detection rate (which we write as a total number of detections per LISA mission)
\begin{equation}
    N_{\rm detect} = \frac{\sum_{i = 0}^{N_{\rm MW}} w_i \phi(i)}{\sum_{i = 0}^{N_{\rm MW}} w_i} N_{\rm DCO},
\end{equation}
where $\phi(i) = 1$ if a binary is detectable and $0$ otherwise. We calculate the detection rate by Monte Carlo sampling 2500 Milky Way instances (each containing 200,000 DCOs) for each DCO type and every physics variation in order to obtain values for the uncertainty on the expected detection rate.

\section{Results I - Predictions for LISA sources} \label{sec:results}
In this section we present our main results for the detectable LISA DCO population. We find that, for our fiducial model, a four year LISA mission will detect \confinv{30.6}{20.5}{51.1} BHBHs, \confinv{26.4}{17.3}{48.5} BHNSs and \confinv{7.1}{4.8}{10.8} NSNSs where the error bars represent the 90\% confidence interval. We first show the distribution of the sources on the sensitivity curve in Section~\ref{sec:dcos_on_sc}, before exploring the variations in the detection rate over different physics variations in Section~\ref{sec:detection_rate_analysis} and analysing the parameter distributions for detectable sources in the fiducial model in Section~\ref{sec:fiducial_distributions}.

\subsection{Distribution on the sensitivity curve}\label{sec:dcos_on_sc}

We illustrate the distribution of detectable DCOs on the LISA sensitivity curve in Figure~\ref{fig:dcos_on_sc}. This shows that the detectable population of these massive DCOs is concentrated at comparatively lower frequencies than the LISA verification binaries (shown as stars in the top panel) and the more numerous WDWD population \citep[e.g.][]{Korol+2017}. This is expected since producing the same SNR as a BHBH, BHNS or NSNS with a relatively lower mass (circular) WDWD requires a higher frequency. This finding is in agreement with \citet{Sesana+2020} and as noted in that work, this could possibly be used to distinguish these more massive DCOs from WDWDs. There are several other notable features in these distributions and we plot grid lines of constant distance and inspiral time to help to explain the shape of the distribution.

The straight diagonal lines show where a binary at a fixed distance and with the average chirp mass (annotated in each panel) would lie on the sensitivity curve for different frequencies. We would therefore expect that if a population was entirely circular, it should be bounded approximately between the $0.1$--$30 \unit{kpc}$ lines (roughly the minimum and maximum distance to a source in the Milky Way). From inspection of the bottom panels with each individual DCO type we see that, though this is the true for a large fraction of the population, there is a distinct subpopulation of binaries that extend downwards, especially around $2 \unit{mHz}$. This offshoot is composed of eccentric binaries for which the circular distance contours do not apply. For instance, we plot the 90\% contour of only the circular sources in our sample over the density distribution in each of the bottom panels and it is clear that the circular subsample is bounded by the distance lines. We also plot a line of constant distance at $30 \unit{kpc}$ for an eccentric binary with $e = 0.98$ to show the differences for eccentric sources. Additionally, we note that the peak of the density distribution coincides with the centre of the Milky Way as expected, since binaries are most likely to be formed towards the centre of the Galaxy.

We plot vertical lines that give the inspiral time for a circular binary with the average chirp mass (annotated in each panel). From these lines we can understand the trend of the density distribution decreasing with increasing frequency. Sources with higher frequencies have shorter inspiral times and thus DCOs will spend less time in these regimes, meaning that more sources are detected at lower frequencies. Note that these inspiral time lines should only be used as guidelines for the population as a whole, as the inspiral time of each individual source will be a function of its mass and eccentricity. It is also evident for each DCO source that the tail of the high frequency sources is more numerous near to the Galactic centre than at short distances. This is simply because there are more sources in the galactic centre and so the chances of `catching' a binary at high frequency are better.

\begin{figure*}[tp]
    \centering
    \includegraphics[width=\textwidth]{2_dcos_on_sc.png}
    \caption{Density distribution of detectable DCOs plotted over the LISA sensitivity curve, where each panel corresponds to \textbf{top:} combined density plot for all DCO types (BHBH, BHNS and NSNS) \textbf{bottom:} three panels with individual density distribution of different DCO types. In each panel, we plot the total signal of binaries at their dominant frequency $n f_{\rm orb}$, such that $n$ is the harmonic that produces the most relative gravitational wave luminosity ($n = 2$ for circular binaries). If the density of points is below our lowest contour (2\%) then we plot the points as scatter points, where their sizes corresponds to their STROOPWAFEL weights. The inset colourbars indicate the percentage of the population represented by each contour and the annotated mass is the average chirp mass for all binaries in the panel, which is used in plotting the grid lines. We plot diagonal lines of constant distance, where the straight lines show the signal for a circular binary of average chirp mass, whilst the curved line shows the signal for an eccentric binary with $e = 0.98$. The vertical lines indicate the inspiral time for a circular binary with the average chirp mass. In the top panel we overlay the LISA verification binaries from \citet{Kupfer+2018}. In the bottom panels we add the 90\% contour line for only the circular sources to show how the distribution changes without eccentric sources present.}
    \label{fig:dcos_on_sc}
\end{figure*}

\subsection{Detection rates}\label{sec:detection_rate_analysis}
\todo{I will be updating this section once the other physics variations are done running. There will be a couple of new models and the current ones will have better high Z resolution.}
We find that for our fiducial model, a four year LISA mission will detect \confinv{30.6}{20.5}{51.1} BHBHs, \confinv{26.4}{17.3}{48.5} BHNSs and \confinv{7.1}{4.8}{10.8} NSNSs, where the error bars represent the 90\% confidence interval. Increasing the LISA mission length to ten years changes the number of detections to \confinv{50.6}{29.2}{63.3}, \confinv{46.2}{25.7}{99.3} and \confinv{12.1}{6.9}{13.5} respectively. In Figure~\ref{fig:detection_rates}, we show the expected number of LISA detections for each model variation and discuss the prominent trends in the following sections. We show the rates and uncertainties plotted in this figure in Table~\ref{tab:detection_rates}.

\begin{figure*}[p]
    \centering
    \includegraphics[width=\textwidth]{3_dco_detections.pdf}
    \caption{The number of expected detections in the LISA mission for different DCO types and model variations. Error bars show the 50\% (solid) and 90\% (dotted) confidence intervals. The left axis and grid lines show the number of detections in a four year LISA mission and the right axis shows an approximation of the number of detections in a 10 year mission (we scale the axis by $\sqrt{T_{\rm obs}}$, see Table~\ref{tab:detection_rates} for exact rates). Each model is described in further detail in Table~\ref{tab:physics_variations} and details of the fiducial assumptions are in Section~\ref{sec:fiducial_physics}. \todo{subject to change with the updated/new models}}
    \label{fig:detection_rates}
\end{figure*}

\subsubsection{BHBH detection rate trends}
The BHBH detection rate is markedly robust across physics variations, with the expected detections in each model staying within 25\% of the fiducial rate (with the exception of model \modOpt{}). Thus even if there are changes in our understanding of the underlying physics before the LISA mission commences, the expected BHBH detection rate is unlikely to change significantly.

The exception to this statement is model \modOpt{}, in which we allow Hertzsprung gap donors to survive common envelope events. A large fraction of the progenitors of BHs in this mass range expand significantly during the Hertzsprung gap phase and initiate common envelope events. Therefore, though the detectable fraction does not change significantly, the increased population of BHBHs in the Milky Way leads to this model predicting 2.5 times more detections.

\subsubsection{BHNS detection rate trends}
In contrast, the BHNS detection rate is very sensitive to changes in binary physics assumptions. Therefore, once LISA flies and we know the actual number of detections, we can compare to each model and possibly provide some constraint on binary evolution physics. There are several notable trends in the BHNS detection rate in the middle pane of Figure~\ref{fig:detection_rates}.

As $\beta$ increases in models \modBetaLow{}-\modBetaHigh{}, the BHNS detection rate steadily decreases. This may seem unintuitive since a higher mass transfer efficiency should lead to more massive compact objects and thus a more detectable population. However, one must also consider that most of these DCOs are formed through a common envelope event and so retaining more of the envelope during mass transfer means that the eventual ejection of the envelope is much more difficult, thus leading to more stellar mergers and fewer detectable BHNSs \citep[e.g.][]{Kruckow+2018}.

\todo{@ALL, the trend with common envelopes still confuses me, specifially, why does it not increase when $\alpha=2.0$? We never quite resolved this in the thread in zpro\_tom\_wagg with me and Lieke. I do see that the BHBH have a lot of only stable mass transfer and so reasonably are not too affected. NSNS basically only come through CE events and so sensibly are strongly affected but BHNS have $\sim 70\%$ classic channel and so should be affected strongly. But we don't see an increase with $\alpha = 2.0$. Any thoughts? (I'm leaving thinking about this for now in case it changes with the new data haha)}
% For a similar reason, the rate is decreased when $\alpha$ is decreased in model \modAlphaLow{}, as this reduces the amount of orbital energy that is used to eject the envelope and thus leads to more stellar mergers. \todo{Why isn't the opposite true for model \modAlphaHigh{}? It seems the number of bound DCOs \textit{does} increase but the merging total decreases...?}

Enforcing that case BB mass transfer is always unstable (model \modCaseBB{}) decreases the detection rate as fewer NSs are produced and thus fewer BHNSs form. This is explained in further detail in Section~\ref{sec:NSNS_detection_trends}. For the same reason as the BHBH rate, model \modOpt{} has a higher number of detections. This change is less prominent than in the BHBH case as the progenitors tend to be lower masses and initiate a CE event less frequently during the Hertzsprung gap phase. 

The Fryer \textit{rapid} prescription (model \modRapid{}) leads to a higher detection rate for BHNSs because progenitors that would become black holes in the \textit{delayed} prescription, instead become neutron stars and so more BHNSs are formed instead of BHBHs. For the same reason, increasing the maximum neutron star mass (model \modNSHigh{}) increases the detection rate and the inverse is true when it is decreased (model \modNSLow{}).

Finally, models \modSigLow{}-\modNoBH{} show increased detection rates since lower kicks result in fewer disrupted binaries and hence a more numerous detectable population. Following this logic it makes sense that model \modSigLower{} produces more detections than model \modSigLow{}. The model with no BH kick (\modNoBH{}) is slightly lower than model \modSigLower{} as the number of surviving binaries is limited by the neutron star kick more than the black hole kick.

\subsubsection{NSNS detection rate trends}\label{sec:NSNS_detection_trends}

As $\beta$ increases the NSNS detection rate increases, the opposite trend to that seen in the BHNS rate. This is for two main reasons: firstly the ejection of a common envelope is less problematic for the less massive NSNS binaries. Moreover, the increased mass transfer efficiency means that systems that were previously below the mass necessary to become a NS can now accrete enough mass to form a NS. Although the same is true for more massive stars becoming BHs instead of NSs, due to the IMF, there is a net flux of more stars becoming NSs.

There is a drastic decrease in detections for model \modCaseBB{} by nearly two orders of magnitude. This is because the majority of NSNS binaries are formed through case BB mass transfer and setting this mass transfer to be always unstable results in many of these binaries to merge before they could become NSNSs. As a result the total number of detections decreases, however, interestingly the remaining population represent more massive progenitors (that would not go through case BB mass transfer) and thus is skewed to higher masses and has a \textit{higher} detectable fraction.

The vast majority of NSNSs in our sample are formed through the common envelope channel and thus changing the value of $\alpha_{\rm CE}$ has an effect on the rate. We see that decreasing $\alpha_{\rm CE}$ (model \modAlphaLow) leads to a lower rate as there is less energy available to eject the envelope and so more binaries result to stellar mergers rather than NSNSs and similarly we see an inverse trend when increasing $\alpha_{\rm CE}$ (model \modAlphaHigh).

As we found in the BHNS trends, a lower value for the core-collapse supernova velocity dispersion increases the detection rate in models \modSigLow{} and \modSigLower{}, whilst changing the PISN or BH kick prescription (models \modNoPISN{} and \modNoBH{}) of course has no effect on the NSNS population.

\subsection{Distributions for the fiducial model}\label{sec:fiducial_distributions}

In Figure~\ref{fig:fiducial_pdf_distributions}, we show the distribution of the individual parameters of the population of detectable binaries and discuss the various features in the following sections.

\begin{figure*}[htbp]
    \centering
    \includegraphics[width=\textwidth]{4_distribution_grid_4yr.pdf}
    \caption{Distributions for various parameters of the DCOs that are detectable in a 4 year LISA mission in our fiducial model. Each panel shows the distribution of a single parameter, where the colour denotes the DCO type. We also plot the 1- and 2-$\sigma$ uncertainties (obtained via bootstrapping) with the dark and light shaded areas respectively. The first two rows (excluding metallicity) use kernel density estimators to show the distributions. The dotted lines in the black hole mass panel show the individual distributions of the primary and secondary masses. The metallicity panel shows the distribution over the metallicity bins used in our population synthesis, which we show in the grid lines. The final three panels for the cumulative distribution functions for observables, normalised to the expected number of fiducial detections in a four year LISA mission. The dark shaded areas indicate regimes in which the quantity cannot be measured. The dotted lines in the angular resolution plot show the maximum angular resolution that can be covered by a single pointing of the labelled instrument. In Section~\ref{sec:fiducial_distributions} we discuss the features of the distributions.}
    \label{fig:fiducial_pdf_distributions}
\end{figure*}

\subsubsection{Black Hole Mass}
For both the BHBHs and BHNSs, the black hole mass distribution extends across relatively low masses, with $88\%$ and $90\%$ respectively below $11 \unit{M_{\odot}}$. This is because, at the high metallicities in the Milky Way, stellar winds are much stronger and strip away much of the stellar mass before BH formation. The mass distribution extends down to $2.5 \unit{M_{\odot}}$, our fiducial maximum neutron star mass, since the \citet{Fryer+2012} \textit{delayed} remnant mass prescription does not produce a mass gap between neutron stars and black holes. Indeed we expect $35\%$ and $39\%$ of detected BHBH and BHNS systems to contain a black hole in the lower mass gap. Therefore, LISA could help to confirm or rule out the existence of the lower mass gap.

The bimodality of the BHBH distribution is a result of most detectable BHBHs in our sample having unequal mass ratios. The two peaks are from the primary and secondary black hole masses, which peak around $8 \unit{M_{\odot}}$ and $3.5 \unit{M_{\odot}}$ respectively. We show these individual distributions as dotted curves below the main BHBH distribution.

The reasoning for these unequal mass ratio systems is as follows: in order to produce a BHBH, most formation channels require at least the first mass transfer to be stable. This stability is strongly dependent on the mass ratio such that equal mass ratios (at the moment of mass transfer) are preferred for creating BHBHs. Yet, since stellar winds are so strong at high metallicity, and even stronger for more massive stars, the primary star will experience significant mass loss and so an initially \textit{unequal} mass ratio is preferred so that the masses are more balanced at the first instance of mass transfer. Since mass transfer occurs after the end of the main sequence for most of our BHBHs, the star will have a well defined core and these core masses, which go on to form BHs, will reflect the initially unequal mass ratios.

\subsubsection{Neutron Star Mass}
The neutron star mass distribution shows that most neutron stars have low masses, with $77\%$ and $91\%$ having masses below $1.7 \unit{M_{\odot}}$ for BHNSs and NSNSs respectively. The lack of neutron stars around $1.7 \unit{M_{\odot}}$ and the subsequent small peaks are artifacts of the discontinuous nature of the \citet{Fryer+2012} remnant mass prescription and do not have strong effects since the vast majority of NSs are formed with lower masses.

Both distributions have notable peaks around $1.26 \unit{M_{\odot}}$, though more strongly in the NSNS case, which are a result of the combination of two effects. Firstly, we set the remnant mass for all electron-capture supernovae to $1.26 \unit{M_\odot}$ (see Sec.~\ref{sec:fiducial_physics}) and thus this leads to a pileup when many systems are formed through ECSN. In addition, the Fryer remnant mass prescription gives a fixed fallback mass for any star with a CO core mass less than $2.5 \unit{M_\odot}$, such that many NSs are given the identical mass of $1.278 \unit{M_\odot}$ in the \textit{delayed} prescription \citep[see][Eq.~19]{Fryer+2012}. 

\subsubsection{Orbital Frequency}
The orbital frequency distributions for BHBHs, BHNSs and NSNSs peak at progressively increasing frequencies. This is because a higher mass DCO at the same distance and eccentricity requires a lower frequency to produce the same signal-to-noise ratio and thus be detected. The BHBH distribution has a tail that extends to $8 \times 10^{-6} \unit{Hz}$, which is comprised of highly eccentric binaries. These systems are still detectable by LISA as the high eccentricity means that the majority of the GW signal is emitted at higher harmonics at higher frequencies that are located in the LISA band. Similar tails are not as prevalent for BHNSs and NSNSs as they do not have as many eccentric binaries.

\subsubsection{Luminosity Distance}
Each DCO's luminosity distance distribution peaks around $8 \unit{kpc}$ since this is the distance to the centre of the Milky Way and thus the most dense location of DCOs. There is a clear bias in each distribution for systems at lower distances since closer binaries are easier to detect. This bias is most prominent for the NSNS distribution since, on average, their lower relative masses require a smaller distance in order to be detected.

\subsubsection{Lookback Time}
Although when creating each mock Milky Way galaxy the lookback time is drawn independently from other parameters, and in the same way for every DCO type, the distributions are clearly different for each type of DCO. This is because for a binary to be detectable, it must have an orbital frequency within the LISA band. Therefore a binary needs to have completed most of its inspiral to be visible in the LISA band and thus detectable systems will tend to have lookback time that are close to their merger times, which \textit{are} a function of other binary parameters.

Since most of the DCOs are formed through common envelope events, their initial separations are relatively tight and so if a binary it given a long lookback time, it will have merged before the LISA mission. This explains the peak in each distribution at short lookback times. The later lookback times correspond to systems that are formed through channels in which mass transfer is only stable and no common envelope event occurs.

The trends across the different DCO types can be explained by considering that the merger time is a function of the masses, frequency and eccentricity of th source. This dependence can be approximately written as $t_{\rm merge} \propto f_{\rm orb}^{-4} m_1^{-3} (1 - e^2)^{-7/2}$,  \citep[][Eq.~5.14]{Peters+1964}. Therefore, although NSNSs are the lowest mass systems, their relatively higher orbital frequencies means that they have the shortest merger times. This same logic implies that BHNSs should have the next shortest merger times and then BHBHs. This order is flipped due to the fact that high eccentricity results in a shorter merger time and we find that LISA detectable BHNSs are mostly circular, whilst BHBHs have a significant highly eccentric subpopulation.

\subsubsection{Metallicity}
\todo{Returning to this once I've made the plot that compares the Milky Way distribution and formation rates}

\subsubsection{Eccentricity}
The eccentricity distributions show that detectable BHBHs are the most eccentric of the three DCOs. This may seem counter-intuitive since neutron stars receive stronger natal kicks, which cause the orbit to become eccentric. However, these stronger kicks often instead result in disrupted or too-wide binaries in more weakly bound NSNSs. In contrast, BHBHs can receive strong kicks that impart high eccentricity without disrupting and thus tend to be more eccentric. This effect is compounded by the fact that we can see BHBHs at lower orbital frequencies, meaning that they have not had as much time to circularise and so still have significant eccentricity by the time of the LISA mission.

A significant fraction of DCOs in each population have eccentricities greater than 0.01, the lower bound on the measurable eccentricity with LISA proposed by \citet{Nishizawa+2016}, which we indicate with the shaded area. We discuss the estimation of the eccentricity uncertainty further in Section~\ref{sec:measurement_uncertainties}

\subsection{Measurement Uncertainties}\label{sec:measurement_uncertainties}
Although it is useful to investigate the underlying parameters of the detectable population, it is also important to consider what LISA will actually \textit{measure} during a detection.

\subsubsection{Eccentricity uncertainty}\label{sec:ecc_unc}
One of the main advantage of space-based gravitational wave detectors such as LISA is that systems may still have significant eccentricity during the LISA mission. Whilst also being useful a useful individual quantity for learning more about sources, the uncertainty on the eccentricity also affects the uncertainty on the chirp mass and so it is important to quantify this uncertainty.

We follow a method used in previous works that uses the individually detectable harmonics of sources to determine the uncertainty of the eccentricity of a source \citep[e.g.][]{Lau+2020, Korol+2021}. Eccentric sources emit at several evenly spaced harmonic frequencies, each with a different fraction of the gravitational wave power and the distribution of the power over the harmonics changes with eccentricity. Therefore, by measuring the SNR of each individual harmonic and finding the ratio of the SNR of the two most detectable harmonics that are above the detection threshold, one can determine the eccentricity. The uncertainty on the eccentricity can therefore be written as a combination of the uncertainty on the SNR of the two most detectable harmonics. In the limit of large SNRs, the eccentricity uncertainty, $\Delta e$, can be written as 
\begin{equation}
    \Delta e = \frac{1}{\rho_1} + \frac{1}{\rho_2},
\end{equation}
where $\rho_1$ and $\rho_2$ are the SNRs of the two most detectable harmonics.

In some cases, sources may only have one individually detectable harmonics, or even none. In these cases one can only put a lower or upper bound on the eccentricity. For sources with only one detectable harmonic, the source is either a circular source or a source with a small enough eccentricity that most of the GW power is still concentrated in the $n = 2$. Therefore, we can use this information to place an upper bound on the eccentricity. Conversely, sources with no detectable harmonics must be eccentric such that the GW power has spread over so many harmonics that no single harmonic has a significantly stronger signal. Therefore, we can use this information to place a lower bound on the eccentricity.

This method provides a pessimistic estimate of the eccentricity of sources since we do not consider the benefits of any sort of matched-filter analysis or similar methods. We discuss how the pessimism of this method affects our results and some possible alternative in Section~\ref{sec:caveats}.

\subsubsection{Chirp mass uncertainty}
The chirp mass uncertainty can be calculated using the uncertainty on the orbital frequency, the time derivative of the orbital frequency and the eccentricity. This is because the time derivative of the orbital frequency can be written as
\begin{equation}
    \dot{f}_{\rm orb} = \frac{48}{5 \pi} \frac{(G \mathcal{M}_c)^{5/3}}{c^3} (2 \pi f_{\rm orb})^{11/3} F(e),
\end{equation}
where $f_{\rm orb}$ is the orbital frequency, $\mathcal{M}_{c}$ is the chirp mass (defined in Eq.~\ref{eq:chirp_mass}) and $e$ is the eccentricity and
\begin{equation}
    F(e) = \frac{1 + \frac{73}{24} e^2 + \frac{37}{96} e^4}{(1 - e^2)^{7/2}},
\end{equation}
is the enhancement factor of gravitational wave emission for an eccentric binary over an otherwise identical circular binary \citep[][Eq.~17]{Peters+1963}.

We can invert this and instead write that the chirp mass is
\begin{equation}
    \mathcal{M}_c = \frac{c^3}{G} \left( \frac{5 \pi}{48} \frac{\dot{f}_{\rm orb}}{F(e)} \right)^{3/5} \frac{1}{(2 \pi f_{\rm orb})^{11/5}},
\end{equation}
and therefore that the chirp mass uncertainty is
\begin{equation}\label{eq:chirp_mass_uncertainty}
    \frac{\Delta \mathcal{M}_c}{\mathcal{M}_c} = \frac{11}{5} \frac{\Delta f_{\rm orb}}{f_{\rm orb}} + \frac{3}{5} \frac{\Delta \dot{f}_{\rm orb}}{\dot{f}_{\rm orb}} + \frac{3}{5} \frac{\Delta F(e)}{F(e)}.
\end{equation}

We calculate the frequency uncertainties using \citet{Takahashi+2002}, such that
\begin{align}
    \frac{\Delta f_{\rm orb}}{f_{\rm orb}} &= 4 \sqrt{3} \cdot \frac{1}{\rho} \frac{1}{T_{\rm obs}} \frac{1}{f_{\rm orb}}, \\
    \frac{\Delta \dot{f}_{\rm orb}}{\dot{f}_{\rm orb}} &= 6 \sqrt{5} \cdot \frac{1}{\rho} \left(\frac{1}{T_{\rm obs}} \right)^2 \frac{1}{\dot{f}_{\rm orb}},
\end{align}
where $\rho$ is the signal-to-noise ratio and $T_{\rm obs}$ is the LISA mission length. Then we calculate the eccentricity certainty as discussed in Sec.~\ref{sec:ecc_unc} and propagate it so that
\begin{equation}
    \frac{\Delta F(e)}{F(e)} = \Delta e \cdot \frac{(1256 + 1608 e^2 + 111 e^4) e}{96 + 196 e^2 - 255 e^4 - 37 e^6}.
\end{equation}

We use Eq.~\ref{eq:chirp_mass_uncertainty} to calculate the chirp mass uncertainty for each DCO in our sample and plot it in the lower centre panel of Fig.~\ref{fig:fiducial_pdf_distributions}. We show the distribution as a CDF that is normalised to the expected number of detections to better indicate the absolute number of detections that will have well determined chirp masses. We find that approximately 10 BHBHs and BHNSs and 5 NSNSs have measurable chirp masses (as indicated by the shaded region). We emphasise that this is a pessimistic estimate due to the pessimistic nature of how we determine the eccentricity uncertainty.

\subsubsection{Angular resolution}

\subsection{Model variations}
\todo{This will be about how the shapes change with model variations}

\section{Results II - Impact of physics assumptions} \label{sec:variations}
In this section we explore the effect of varying our underlying physics assumptions and model for the Milky Way. In Section~\ref{sec:detection_rate_analysis} we explore the changes in the detection rate over different physics variations  and in Section~\ref{sec:property_variations} we consider how these model physics variations change the properties of detectable systems. Finally we demonstrate how different models for the Milky Way can affect our results in Section~\ref{sec:mw_changes}.

\subsection{Detection rates}\label{sec:detection_rate_analysis}
\tom{I will be updating this section once the other physics variations are done running. There will be a couple of new models and the current ones will have better high Z resolution.}

We find that for our fiducial model on average, a 4-year LISA mission will detect \BHBHFourYear{} BHBHs, \BHNSFourYear{} BHNSs and \NSNSFourYear{} NSNSs. Increasing to a 10-year LISA mission length changes the number of detections to \BHBHTenYear{}, \BHNSTenYear{} and \NSNSTenYear{} respectively. In Figure~\ref{fig:detection_rates}, we show the expected number of LISA detections for each model variation and discuss the prominent trends in the following sections. We show the rates and uncertainties plotted in this figure in Table~\ref{tab:detection_rates}.

\begin{figure*}[p]
    \centering
    \includegraphics[width=\textwidth]{3_dco_detections.pdf}
    \caption{The number of expected detections in the LISA mission for different DCO types and model variations. Error bars show the 1- (solid) and 2-$\sigma$ (dotted) Poisson uncertainties. An arrow indicates that the error bar extends to zero. The left axis and grid lines show the number of detections in a 4-year LISA mission and the right axis shows an approximation of the number of detections in a 10-year mission (we scale the axis by $\sqrt{T_{\rm obs}}$, see Table~\ref{tab:detection_rates} for exact rates). Each model is described in further detail in Table~\ref{tab:physics_variations} and details of the fiducial assumptions are in Section~\ref{sec:fiducial_physics}. See Sec.~\ref{sec:detection_rate_analysis} for a discussion. \todo{subject to change with the updated/new models and new uncertainty estimates}}
    \label{fig:detection_rates}
\end{figure*}

\subsubsection{BHBH detection rate trends}
The BHBH detection rate is markedly robust across physics variations, with the expected detections in each model staying within 25\% of the fiducial rate (with the exception of model \modOpt{}). Thus even if there are changes in our understanding of the underlying physics before the LISA mission commences, the expected BHBH detection rate is unlikely to change significantly.

The exception to this statement is model \modOpt{}, in which we allow Hertzsprung gap donors to survive common envelope events. A large fraction of the progenitors of BHs in this mass range expand significantly during the Hertzsprung gap phase and initiate common envelope events. Therefore, though the detectable fraction does not change significantly, the increased population of BHBHs in the Milky Way leads to this model predicting 2.5 times more detections.

\subsubsection{BHNS detection rate trends}
In contrast, the BHNS detection rate is very sensitive to changes in binary physics assumptions. Therefore, once LISA flies and we know the actual number of detections, we can compare to each model and possibly provide some constraint on binary evolution physics. There are several notable trends in the BHNS detection rate in the middle pane of Figure~\ref{fig:detection_rates}.

As $\beta$ increases in models \modBetaLow{}-\modBetaHigh{}, the BHNS detection rate steadily decreases. This may seem unintuitive since a higher mass transfer efficiency should lead to more massive compact objects and thus a more detectable population. However, one must also consider that most of these DCOs are formed through a common envelope event and so retaining more of the envelope during mass transfer means that the eventual ejection of the envelope is much more difficult, thus leading to more stellar mergers and fewer detectable BHNSs \citep[e.g.][]{Kruckow+2018}.

\tom{@ALL, the trend with common envelopes still confuses me, specifially, why does it not increase when $\alpha=2.0$? We never quite resolved this in the thread in zpro\_tom\_wagg with me and Lieke. I do see that the BHBH have a lot of only stable mass transfer and so reasonably are not too affected. NSNS basically only come through CE events and so sensibly are strongly affected but BHNS have $\sim 70\%$ classic channel and so should be affected strongly. But we don't see an increase with $\alpha = 2.0$. Any thoughts? (I'm leaving thinking about this for now in case it changes with the new data haha)}
% For a similar reason, the rate is decreased when $\alpha$ is decreased in model \modAlphaLow{}, as this reduces the amount of orbital energy that is used to eject the envelope and thus leads to more stellar mergers. \todo{Why isn't the opposite true for model \modAlphaHigh{}? It seems the number of bound DCOs \textit{does} increase but the merging total decreases...?}

Enforcing that case BB mass transfer is always unstable (model \modCaseBB{}) decreases the detection rate as fewer NSs are produced and thus fewer BHNSs form. This is explained in further detail in Section~\ref{sec:NSNS_detection_trends}. For the same reason as the BHBH rate, model \modOpt{} has a higher number of detections. This change is less prominent than in the BHBH case as the progenitors tend to be lower masses and initiate a CE event less frequently during the Hertzsprung gap phase. 

The Fryer \textit{rapid} prescription (model \modRapid{}) leads to a higher detection rate for BHNSs because progenitors that would become black holes in the \textit{delayed} prescription, instead become neutron stars and so more BHNSs are formed instead of BHBHs. For the same reason, increasing the maximum neutron star mass (model \modNSHigh{}) increases the detection rate and the inverse is true when it is decreased (model \modNSLow{}).

Finally, models \modSigLow{}-\modNoBH{} show increased detection rates since lower kicks result in fewer disrupted binaries and hence a more numerous detectable population. Following this logic it makes sense that model \modSigLower{} produces more detections than model \modSigLow{}. The model with no BH kick (\modNoBH{}) is slightly lower than model \modSigLower{} as the number of surviving binaries is limited by the neutron star kick more than the black hole kick.

\subsubsection{NSNS detection rate trends}\label{sec:NSNS_detection_trends}

As $\beta$ increases the NSNS detection rate increases, the opposite trend to that seen in the BHNS rate. This is for two main reasons: firstly the ejection of a common envelope is less problematic for the less massive NSNS binaries. Moreover, the increased mass transfer efficiency means that systems that were previously below the mass necessary to become a NS can now accrete enough mass to form a NS. Although the same is true for more massive stars becoming BHs instead of NSs, due to the IMF, there is a net flux of more stars becoming NSs.

There is a drastic decrease in detections for model \modCaseBB{} by nearly two orders of magnitude. This is because the majority of NSNS binaries are formed through case BB mass transfer and setting this mass transfer to be always unstable results in many of these binaries to merge before they could become NSNSs. As a result the total number of detections decreases, however, interestingly the remaining population represent more massive progenitors (that would not go through case BB mass transfer) and thus is skewed to higher masses and has a \textit{higher} detectable fraction.

The vast majority of NSNSs in our sample are formed through the common envelope channel and thus changing the value of $\alpha_{\rm CE}$ has an effect on the rate. We see that decreasing $\alpha_{\rm CE}$ (model \modAlphaLow) leads to a lower rate as there is less energy available to eject the envelope and so more binaries result to stellar mergers rather than NSNSs and similarly we see an inverse trend when increasing $\alpha_{\rm CE}$ (model \modAlphaHigh).

As we found in the BHNS trends, a lower value for the core-collapse supernova velocity dispersion increases the detection rate in models \modSigLow{} and \modSigLower{}, whilst changing the PISN or BH kick prescription (models \modNoPISN{} and \modNoBH{}) of course has no effect on the NSNS population.

\subsection{Properties of detectable systems}\label{sec:property_variations}
\tom{This will be about how the shapes of the parameter distributions change for the different model variations. I won't detail every variation but I'll point out anything that stands out and leave the rest in the appendix plots.}

\subsection{Assessing the impact of Milky Way model choices}\label{sec:mw_changes}
The model that we use for the Milky Way adds several layers of complexity, accounting for the inside-out growth of the thin disc, using empirically informed star formation histories that are a function of time and assigning metallicities based on the position and age of binaries. In this section, we repeat our main analysis but instead apply a simpler model for the Milky Way in order to assess the effect of these added features. For this purpose, we use model for the Milky Way used in \citet{Breivik+2020} as this is representative of the models used in most previous works.

Their model can be summarised as follows: the Milky Way is assumed to comprise of three components, a thin disc, a thick disc and a bulge. The spatial distributions and relative masses for these components are given in \citet{McMillan+2011}. \citet{Breivik+2020} assume constant star formation over 10 Gyr for the thin disc, a 1 Gyr burst of star formation 11 Gyr ago for the thick disc and a 1 Gyr burst of star formation 10 Gyr ago for the bulge. A major difference is that only two metallicities are used and they are assigned independent of age or position. Binaries formed in the thin disc and bulge are assumed to have a metallicity of $Z = 0.02$ and those formed in the thick disc are assumed to have $Z = 0.003$.

We show this model in Fig.~\ref{fig:simple_mw} in the same form as Fig.~\ref{fig:galaxy_schematic} for ease of comparison between our models. The two main differences we can see in these plots are that the \citet{Breivik+2020} model is more centrally concentrated and only has two fixed metallicity populations.
%In addition, though it can't be seen in these plots, the simpler prescriptions for the star formation histories mean that the overall star formation history is highly discontinuous at $\tau = 9 \unit{Gyr}$ and significantly more star formation occurs from $\tau = 9 \unit{Gyr}$ to $\tau = 11 \unit{Gyr}$.

\begin{figure}[t]
    \centering
    \includegraphics[width=\columnwidth]{../figures/random_simple_galaxy.png}
    \caption{As Fig.~\ref{fig:galaxy_schematic} (right panel), but for the Milky Way model used in \citet{Breivik+2020}.}
    \label{fig:simple_mw}
\end{figure}

We find that when applying this simpler Milky Way model and our fiducial physics assumptions (model \modFid{}), the expected number of detections for BHBHs, BHNSs and NSNSs for a 4-year LISA mission is $39$, $39$ and $14$ respectively. Thus the BHBH and BHNS detection rates are only marginally increased from our main findings, but the NSNS detection is overestimated by nearly a factor of 2.

Moreover, the distribution of parameters within the population, particularly the mass distributions, are notably disparate. By using only two fixed metallicity populations, unphysical artifacts are introduced into distribution of DCO masses (Kummer at al. (in prep)). For example, in Fig.~\ref{fig:bh_mass_simple_mw}, we show the black hole mass distribution produced by the simulation using the simple Milky Way model. Despite the fact that these KDEs use the same bandwidth as Fig.~\ref{fig:fiducial_pdf_distributions}, the distributions show many more sharp transitions, which is a result of pileups occurring at specific masses for specific metallicities. Moreover, the lack of lower metallicities systems means that higher mass systems are not formed and so we see the distributions do not include a high mass tail such as in our fiducial results.

The unphysical artifacts present in the mass distributions can have far-reaching effects since the masses of DCOs affect most other parameters. The inspiral time and SNR are directly dependent on the mass, whilst the uncertainty estimates depend on the SNR. This means that the artifacts can affect the predictions for most distributions of LISA detectable populations.

Overall, we find that previous studies that use Milky Way models analogous to this simpler model may significantly overestimate the LISA NSNS detection rate as well as contain unphysical artifacts in their parameter distributions.

\begin{figure}[h]
    \centering
    \includegraphics[width=\columnwidth]{../figures/BH_mass_dist_simple_mw.pdf}
    \caption{As Fig.~\ref{fig:fiducial_pdf_distributions} (top left panel), but for the Milky Way model used in \citet{Breivik+2020}.}
    \label{fig:bh_mass_simple_mw}
\end{figure}



\section{Discussion} \label{sec:discussion}
In this section we discuss the prospects of (and methods for) identifying LISA sources (Sec.~\ref{sec:identify_sources}), the possibility of matching LISA signals to SKA detections (Sec.~\ref{sec:pulsar_matching}), the main caveats for this study (Sec.~\ref{sec:caveats}) and the possible contribution from other formation channels (Sec.~\ref{sec:other_formation_channels}). All predictions quoted in each subsection are derived for the fiducial model (model \modFid{}).

\subsection{Identification of GW sources}\label{sec:identify_sources}
It is important to note that, though we present predictions for the detection rates of specific DCO types, the nature of the source may not be immediately apparent from the gravitational wave signal. LISA can detect a variety of sources, from exoplanets \citep[e.g.][]{Tamanini+2019} to common-envelopes \citep[e.g.][]{Ginat+2020, Renzo+2021} that may cause confusion. However, by far the most prominent will be the population of Galactic WDWDs detectable with LISA, which will be several orders of magnitude larger than the population of the more massive DCOs that we focus on in this paper \citep[e.g.][]{Korol+2017}. It is therefore imperative that we consider how to distinguish NS and BH binaries from this much more numerous population of sources. In addition to distinguishing them from WDWDs, we must consider how to discriminate between BHBHs, BHNSs and NSNSs themselves.

\subsubsection{Distinguishing from WDWD population}\label{sec:WDWD_distinguish}
The simplest way to check whether a source is a WDWD is to evaluate its chirp mass. The mass of a non-rotating white dwarf cannot be larger than the Chandrasekhar limit of $1.4 \unit{M_\odot}$ \citep{Chandrasekhar+1931, Hamada+1961}, so we can take the maximum chirp mass of a WDWD to be ${\sim}1.2 \unit{M_{\odot}}$. Therefore, any DCO with a chirp mass that satisfies $\mathcal{M}_c > 1.2 \unit{M_{\odot}} + \Delta \mathcal{M}_c$ must not be a WDWD (where $\Delta \mathcal{M}_c$ is the error on the chirp mass, estimated using Eq.~\ref{eq:chirp_mass_uncertainty}). We find that for the detectable population of a 4(10)-year LISA mission, \BHBHAboveMaxWDWDFourPerc{}(\BHBHAboveMaxWDWDTenPerc{})\% of BHBHs, \BHNSAboveMaxWDWDFourPerc{}(\BHNSAboveMaxWDWDTenPerc{})\% of BHNSs and \NSNSAboveMaxWDWDFourPerc{}(\NSNSAboveMaxWDWDTenPerc{})\% of NSNSs satisfy this condition. This method is not particularly effective for NSNSs since their average chirp mass, $1.17 \unit{M_\odot}$, is below the Chandrasekhar limit.

Another discriminator between WDWDs and other DCOs is eccentricity. WDWDs formed in the disc are thought to be formed mainly through isolated binary formation and have little to no eccentricity (e.g.\ \citealt{Nelemans+2001}, see however \citealt{Dosopoulou+2016a, Dosopoulou+2016b, Gosnell+2019}). This is because WDWDs formed through isolated binary evolution all experience a phase of mass transfer or a common envelope, which typically circularises the binary \citep[e.g.][]{Marsh+2004}. However, in contrast to the more massive DCOs that we study, WDWDs do not experience strong natal kicks which we find to be the main source of eccentricity. Therefore, if any system is detected with anything other than one detectable harmonic, this suggests that the system is unlikely to be a WDWD. We find that for a 4(10)-year LISA mission, \BHBHMultipleHarmonicsFourPerc{}(\BHBHMultipleHarmonicsTenPerc{})\% of BHBHs, \BHNSMultipleHarmonicsFourPerc{}(\BHNSMultipleHarmonicsTenPerc{})\% of BHNSs and \NSNSMultipleHarmonicsFourPerc(\NSNSMultipleHarmonicsTenPerc{})\% of NSNSs are detected with multiple harmonics (see also Sec.~\ref{sec:fiducial_distributions}). Both the absolute percentage and the relative improvement with an extended LISA mission is lower for the BHNSs with respect to other DCOs as we find that these BHNSs are less eccentric on average (see Fig.~\ref{fig:fiducial_pdf_distributions}d and discussion in Sec.~\ref{sec:fiducial_distributions}).

However, we should also consider that eccentric WDWDs could be formed through dynamical formation in Milky Way globular clusters \citep[e.g.][]{Willems+2007, Kremer+2018}, or with third companions \citep[e.g.][]{Antonini+2017}. This means that we cannot assume that eccentric binaries are not WDWDs unless they are detected in the Galactic plane (though even then there is a chance they were formed dynamically). We can use the sky localisation, scale height of the disc and distance to the source to estimate what fraction of eccentric sources can be localised to the Galactic plane. This condition can be written as $\sigma_\theta < \arcsin(z_{\rm plane} / D_L)$ or $D_L < z_{\rm plane}$, where we set the height of the Galactic plane, $z_{\rm plane}=0.95 \unit{kpc}$, to the scale height of the high-$\alpha$ disc. We apply this condition to find that the fraction of sources that are eccentric and localised within the disc for a 4(10)-year LISA mission are \BHBHEccInDiscFourPerc{}(\BHBHEccInDiscTenPerc{})\% for BHBHs, \BHNSEccInDiscFourPerc{}(\BHNSEccInDiscTenPerc{})\% for BHNSs and \NSNSEccInDiscFourPerc{}(\NSNSEccInDiscTenPerc{})\% for NSNSs. Note that although the fractions are the same for the 10-year mission, the absolute number of detections is still greater.

Overall, combining these methods (chirp mass, eccentricity and sky localisation) we find that for a 4(10)-year mission, LISA will detect at least \BHBHNotWDWDFour{}(\BHBHNotWDWDTen{}) BHBHs, \BHNSNotWDWDFour{}(\BHNSNotWDWDTen{}) BHNSs and \NSNSNotWDWDFour{}(\NSNSNotWDWDTen{}) NSNSs that are distinguishable from the WDWD population. Thus we will be able to confidently distinguish approximately half of all detected sources from WDWDs. This increases to roughly 60\% for a 10-year mission. We highlight that, though the overall number of LISA detections in an extended mission only increases by a factor of $\sqrt{T_{\rm obs}}$, the number of distinguishable detections increases by a greater factor since each of the more numerous sources are better measured. This further underlines the benefits of extending the LISA mission to 10 years.

\subsubsection{Discriminating between BHBHs, BHNSs and NSNSs}

The problem of discriminating between the BHBH, BHNS and NSNS populations can be more difficult than distinguishing them from WDWDs. For NSNSs, we can follow a similar method to the WDWDs (see Sec.~\ref{sec:WDWD_distinguish}) by applying our knowledge of the maximum mass of a neutron star. Following our fiducial assumption, we can take the maximum mass of a neutron star as $2.5 \unit{M_{\odot}}$ and thus the maximum chirp mass that a system can attain without one of the components being a black hole is $\mathcal{M}_{c} = 2.2 \unit{M_\odot}$. For a 4(10)-year LISA mission, the fraction of systems that are above or below this limit (and thus \textit{must} respectively contain or not contain a BH component) by more than $\Delta \mathcal{M}_c$ is \BHBHEitherBHOrNSFourPerc{}(\BHBHEitherBHOrNSTenPerc{})\% for BHBHs, \BHNSEitherBHOrNSFourPerc{}(\BHNSEitherBHOrNSTenPerc{})\% for BHNSs and \NSNSEitherBHOrNSFourPerc{}(\NSNSEitherBHOrNSTenPerc{})\% of NSNSs, which in terms of absolute detections is \BHBHEitherBHOrNSFour{}(\BHBHEitherBHOrNSTen{}) for BHBHs, \BHNSEitherBHOrNSFour{}(\BHNSEitherBHOrNSTen{}) for BHNSs and \NSNSEitherBHOrNSFour{}(\NSNSEitherBHOrNSTen{}) for NSNSs.

For separating the BHBH and BHNS population one could do so probabilistically given the properties that are measured, particularly the orbital frequency, mass ratio and eccentricity, since these distributions are fairly different for the two DCO types (see Fig.~\ref{fig:fiducial_pdf_distributions}). This method would pose a challenge, however, as it would likely only indicate which type was more likely rather than discriminate between them with strong evidence.

Another possible solution would be the existence of electromagnetic counterparts to the gravitational wave signal. In Section~\ref{sec:pulsar_matching} we consider the possibility of detecting a pulsar within a BHNS or NSNS system. This could be used to identify the type of the source.

\subsection{Matching LISA detections to pulsars with the SKA}\label{sec:pulsar_matching}
Since the vast majority of the LISA detectable population of DCOs will not merge for many years, the main type of electromagnetic counterpart for this population is pulsars. Therefore, for this section we focus only on BHNSs and NSNSs since no BHBH system will contain a pulsar. The joint detection of a binary pulsar with LISA and the Square Kilometre Array (SKA, \citealt{Dewdney+2009}) would not only help to constrain the parameters of the binary, but also enable investigation of other compact object physics. A pulsar(PSR)+BH can provide stringent tests of theories of gravity, in particular the ``No-hair theorem'' \citep{Keane+2015}. Alternatively, an ultrarelativstic PSR+NS system could be used to measure the neutron star equation of state up to an order of magnitude more accurately than other proposed observational constraints \citep{Kyutoku+2019, Thrane+2020}.

We estimate that on average, given the number of detectable pulsars and the SKA sky area, each pulsar in the SKA occupies a region with an angular resolution of $\sigma_{\theta} < 1.3^\circ$ or $0.7^\circ$ for SKA-1 and SKA-2 respectively (see Appendix~\ref{app:ska_area}). Therefore, any DCOs containing NSs localised by LISA with an angular resolution lower than these values can be unambiguously matched to the radio signal in the SKA. By considering Fig.~\ref{fig:ang_res}, approximately $11$ and $6$ (for SKA-1 and SKA-2) DCOs will satisfy this constraint.

If there is more than one pulsar in the region given by the LISA sky localisation, one can compare the measured parameters of the system in LISA and the SKA. Both the SKA and LISA will measure the orbital frequency to high precision, as well as the time derivative of the frequency and chirp mass to a lesser precision, of each of these systems. Therefore, one could perform a targeted search with the SKA that checks the sky location given by LISA, only looking for binary pulsars with orbital frequencies within the uncertainties. If there was \textit{still} more than one possible pulsar one could also check against the chirp mass. In this way, we expect it will be possible to get a joint detection between the SKA and LISA even when the sky area implied by the LISA detection contains more than one pulsar.

In order to assess the efficacy of this method, we would need to know the probability that two random binary pulsars would have orbital frequencies and chirp masses close enough that one could not tell which pulsar matches the LISA detection. This would require simulating the SKA population of pulsars with a code such as PSRPOPPy \citep{Bates+2014} to find the frequency and chirp mass distribution, which is beyond the scope of this paper. However, the uncertainty in the orbital frequency of a binary on the detection threshold (${\rm SNR} = 7$) for a 4-year LISA mission is $2.5 \times 10^{-9} \unit{Hz}$ and $1.0 \times 10^{-9} \unit{Hz}$ for a 10-year mission (calculated using Eq.~\ref{eq:f_orb_unc}). Therefore, we expect that the SKA could likely isolate the correct binary pulsar to match to a LISA detection even when several are present in the sky localisation region.

\subsection{Caveats}\label{sec:caveats}

Our predictions are subject to various uncertainties which can be broadly divided into two different categories: those arising from the progenitor models for the population of DCOs and those arising from the choices we have made when placing these DCOs in our model for the Milky-Way. Although we are unable, at present, to evaluate the impact of all these uncertainties, the reader should nevertheless keep in mind that they are likely very substantial. Most of these concerns are not unique to these study, but apply to most of the predictions available in present literature. We highlight a few main concerns. 

\paragraph{Progenitor models} Our binary-star progenitors models have been computed with a rapid population synthesis code (see Sec.~\ref{sec:COMPAS_explained}). This code relies on approximate parametric prescriptions for the stellar evolutionary tracks of single tracks and simple algorithms to mimic the effects of evolutionary and binary interaction  processes. Even though we explicitly consider the impact of some of the main physics uncertainties (see Sect.~\ref{sec:variation_assumptions}) the list of variations that we considered is far from exhaustive. Moreover, it is by no means guaranteed that the parametric prescriptions used in this code lead to realistic results, even when varying the values of the parameters to their extremes. We stress in particular the uncertainties affecting our most massive progenitor models. Observational constraints are scarce for high mass stars and practically non-existent for the rapid evolutionary phases \citep[e.g.][]{Langer2012, Mapelli+2021}. This is even more true for the evolution of massive stars at low metallicity. In addition to our limited understanding of massive stars, we note that the rapid population synthesis code, such as the one employed to compute the models used in this study, rely on extrapolations of the original fitting formulae to approximate the evolutionary tracks for these higher mass progenitors \citep{Hurley+2000,Hurley+2002}. 
 
\paragraph{Populating the Milky Way} Our Milky Way model is semi-empirical and has been calibrated based on observations. Unfortunately, the early evolution of the (metallicity dependence of the) star formation history is poorly constrained. We do not expect this to be a major concern, as most of the double compact objects have relatively short delay times of less than $2 \unit{Gyr}$ (see Fig.~\ref{fig:fiducial_pdf_distributions}e), but this is a caveat that should be kept in mind. Furthermore, to estimate the rate of detectable systems, we rely on normalisation choices (e.g.\ how many detectable double compacts are formed per unit of star formation). This depends heavily on the initial mass function, as low mass stars account for most of the mass while high mass stars are the progenitors of double compact objects. Further choices, such as the binary fraction and the initial distributions of binary parameters also play a lesser but probably still significant role \citet[e.g.][]{deMink+2015, Chruslinska+2017, Klencki+2018}. 

We also note that, for reasons of computational efficiency, we have not accounted for the spatial velocities resulting from the Blaauw-Boersma kick \citep{Blaauw+1961,Boersma1961}. In test simulations we find that accounting for this spreads out the population (increasing the typical height above the Galactic plane and Galactocentric radius), but we find that the impact on the rate is limited. In light of the other much larger uncertainties, we felt that this was justified (see however, e.g., \citealt{Brandt+1995, Abbott+2017_GW170817_progenitor}). We have further ignored a possible contribution coming from the Galactic halo, as \citet{Lamberts+2018} estimates this not be significant. However, this may not be true for other formation channels other than those we have considered here.

\subsection{Other formation channels}\label{sec:other_formation_channels}

In this paper we considered the formation of NS and BH binaries formed via isolated binary evolution, through the classical CE channel, the stable mass transfer channel and variations on these (see Fig.~\ref{fig:formation_channels}). We did not consider further possible contributions from other formation channels, which may play a role.

We highlight the possible role of dynamical formation in globular clusters, which was investigated by \citet{Kremer+2018}. For a nominal 4-year LISA mission, they estimate 21 sources in total will have SNR $> 7$, of which 7 are BHBHs, 0 are BHNSs and 1 is a NSNS \citep[see Table~1][]{Kremer+2018}. In this case, we predict that the sources formed through isolated binary formation will be dominant in nearly every model variation.

In addition, \citet{Banerjee+2020} investigates formation of LISA detectable BHBHs in young massive and open stellar clusters. and estimates approximately 128 BHBHs with SNR $>5$ in a 5-year LISA mission \citep[see Table~1, Column 9][]{Banerjee+2020}. Note that this total uses a slightly longer mission length and a lower SNR threshold than those used in this paper. Nevertheless, this rate is on the same order as our predictions for LISA BHBHs.


\section{Comparison with previous studies}\label{sec:compare_studies}
\begin{figure*}[p]
    \centering
    \includegraphics[width=\textwidth]{5_compare_dco_1.png}

    \vspace{0.5cm}

    \includegraphics[width=\textwidth]{5_compare_dco_2.png}
    \caption{A table comparing previous studies of a similar nature to this work. The works listed in the table are \citet{Nelemans+2001}, \citet{Belczynski+2010}, \citet{Liu+2014}, \citet{Lamberts+2018}, \citet{Sesana+2020}, \citet{Lau+2020}, \citet{Breivik+2020} and \citet{Shao+2021}.}
    \label{fig:previous_studies}
\end{figure*}

In Figure~\ref{fig:previous_studies}, we compare our results to similar previous studies that investigate the population of stellar mass BHBHs, BHNSs and NSNSs that are detectable with LISA. Figure~\ref{fig:previous_studies} details the expected detection rates predicted by each paper as well as their assumptions regarding their Milky Way galaxy model, binary population synthesis simulation and LISA mission specifications. We only include papers that are similar to our work, such that they use population synthesis and simulate sources in the Galactic plane. Moreover, Figure~\ref{fig:previous_studies} does not include the numerous papers on the LISA WDWD population as we do not make predictions for these DCOs.
 
\paragraph{\citet{Nelemans+2001}} were the first to investigate the population of LISA detectable stellar mass double compact objects. We find a significantly higher detection rate for BHBHs and BHNSs, as well as a slightly lower rate for NSNSs. We can understand this difference from changes both to the specifications of LISA (such as the mission length and SNR threshold for detection) and our understanding of massive star evolution since the publication of their paper, which both strongly affect the expected detections rates.

\paragraph{\citet{Belczynski+2010}} built upon the work of \citet{Nelemans+2001}, by using a different population synthesis code with two model variations and a multi-component model for the Milky Way. They find a much lower detection rate for BHNSs and NSNSs (and agreed on zero BHBHs) when compared to \citet{Nelemans+2001}. They claim that this discrepancy comes from differences in their population synthesis and an overall lower formation rate rather than any changes to LISA detectability. The low total detection rate for all DCOs in this paper compared to our work is unsurprising given the relatively high SNR threshold of 10 and short mission length of 1 year. The reduced mission length means that the source signal has much less time to accumulate, whilst also fewer WDWDs can be resolved in this time, leading to a weaker signal and an increased Galactic confusion noise relative to our work.

\paragraph{\citet{Liu+2014}} performed a similar investigation using a different population synthesis code and find higher rates than earlier works. Their lower detection threshold and longer mission length compared to \citet{Belczynski+2010} likely explains the relatively increased rates. Yet their rates are still significantly below what we find. This could be for several reasons; they assume all binaries are circular both in their evolution and for detection. This means that systems may not have inspiralled as far before the LISA mission or may appear to have weaker gravitational waves when eccentricity is not accounted for. They also use a simplified model for the Milky Way with a single disc of one metallicity and constant star formation, whilst also using a mission length half what we assume. Each of these factors likely contributes to the lower overall detection rates.

\paragraph{\citet{Lamberts+2018}} presented a new approach to the problem by using the FIRE simulation \citep{Hopkins+2014} to distribute their sources rather than an analytical model of the Milky Way, thus being the first paper in this area to incorporate metallicity dependence into their Milky Way model. \citet{Sesana+2020} followed up on this paper using the same simulated BHBH population and presented updated results for the number of expected BHBH detections. They find significantly fewer BHBHs than our fiducial model despite using the same SNR threshold and LISA mission length.
%
The discrepancy between the results of \citet{Sesana+2020} and those presented in this work could be caused by different treatments of eccentricity. Unlike our work, \citet{Sesana+2020} assume that all binaries are circular for the purpose of detection in LISA, which could result in a lower number of detections by missing eccentric binaries that appear as weaker signals when assumed to be circular. This is especially relevant as we find that around $\BHBHNotCirc{}$ of LISA detectable BHBHs are not circular and around $\BHBHHighlyEccentric{}$ have significant eccentricity (see Section~\ref{sec:fiducial_distributions}). We also improve upon this work by using a larger number of metallicity bins compared to \citet{Sesana+2020}, since a low number of metallicity bins can produce artificial features in the mass distribution of DCOs and possibly affect the detection rate (see Section~\ref{sec:mw_changes}). Finally, it could be that different implicit assumptions in their population synthesis code lead to differences in our results \citep{Toonen+2014}.

\paragraph{\citet{Lau+2020}} focussed on the number of Galactic NSNS binaries that could be detected by LISA. Their study uses the same population synthesis code, COMPAS, as this work, though an earlier version. Despite this, their study finds a much larger number of detections. They make several different physical assumptions in their population synthesis, using the \citet{Fryer+2012} \textit{rapid} remnant mass prescription, limiting the maximum neutron star mass to $2 \unit{M_{\odot}}$ and not implementing PISN. However, we note that none of these assumptions strongly affect the NSNS LISA detection rate (see bottom panel of Fig.~\ref{fig:detection_rates}, models \modRapid{}, \modNSLow{} and \modNoPISN{}) and so this is unlikely to entirely account for our differences. It is also important to highlight that COMPAS has had many improvements and bug fixes since \citet{Vigna-Gomez+2018} (which contains the simulations used by \citet{Lau+2020}) and these could have affected the formation rate of NSNSs.
%
It is also possible that the remaining difference between our results is due to way in which we simulate the Milky Way. \citet{Lau+2020} use a model for the Milky Way similar to that of \citet{Breivik+2020}, which we used to estimate how impactful the choice of MW model was in Appendix~\ref{sec:mw_changes}. We find that this simpler model for the Milky Way could result in an overestimate of the NSNS detection rate and so this may explain the discrepancy between our results. \citet{Lau+2020} additionally only uses a single metallicity rather than the two used in \citet{Breivik+2020} and so this effect could be even stronger as their results will have no contribution from lower metallicity systems.

\paragraph{\citet{Breivik+2020}} introduced the population synthesis code COSMIC and presented detections for many different DCO types in LISA using this code. They find that LISA will detect 93 BHBH, 33 BHNS and 8 NSNS binaries in the Milky Way over a 4 year mission. They make many different physical assumptions, the most notable being that \citet{Breivik+2020} assume the optimistic CE scenario and that case BB mass transfer is always unstable. Thus it would be more prudent to compare to our results from model \modCaseBBOpt{} in which we find 154, 148 and 18 detections respectively. \todo{I am in the process of running some sims using Katie's MW model to compare more easily, will update soon}%Thus the NSNS rate is consistent but we find higher rates for the BHBHs and BHNSs. These differences are likely due to using a different population synthesis code (COSMIC) and using a different model for the Milky Way, particularly the assumptions of two fixed metallicities.

\paragraph{\citet{Shao+2021}} most recently investigated the detectability binaries containing BHs in LISA using BSE and a relatively simple model for the Milky Way (assuming a uniform flat disc, constant star formation and a single metallicity). They assume that kicks for NSs formed through ECSN are slightly higher than our work ($50 \unit{km}{s^{-1}}$ instead of $30 \unit{km}{s^{-1}}$). This may account for their particular low BHNS rate (as the binaries would be more likely to disrupt), which is a factor of 20 lower than ours, but we expect their assumption of the optimistic CE scenario, reduced Wolf-Rayet winds and lower SNR detection threshold would offset this. It is however difficult to compare further due to our vastly different assumptions about the Milky Way, particularly their assumption of only solar metallicity which eliminates any contribution from low metallicity systems.

\section{Conclusion \& Summary} \label{sec:conclusion}
We provide predictions for the detection rate and population properties of LISA detectable BHBH, BHNS and NSNS.
To this end, we use the rapid population synthesis code COMPAS to simulate over two billion massive binaries, to explore the effect of varying underlying physics assumptions that represent the most common uncertainties in binary physics. We use an new empirically-informed analytical model to distribute the resulting BHBH, BHNS and NSNS populations in a Milky-Way like galaxy based on their birth metallicity, in order to estimate and investigate the LISA detectable population of BH and NS binaries.

Our main conclusions can be summarised as:
\begin{enumerate}
    \item \textbf{Detections:} We predict 30-300 detections in a 4-year LISA mission, across all our simulations for varying physics assumptions. Although the number of detections per type can vary by about 2 orders of magnitude, we find that the total detection rate is fairly robust, among the variations we have considered.
    
    Specifically, our fiducial model predicts a total of $124 \pm 11$ detections and out of these we find about $\BHBHFourYear{}\pm 9$ BHBHs, $\BHNSFourYear{}\pm 6$ BHNSs and $\NSNSFourYear{}\pm 3$ NSNSs. The errors quoted here are the $1$-$\sigma$ Poisson uncertainties resulting from the random initialisation of the Milky Way (see Table~\ref{tab:detection_rates}).
    
    \item \textbf{Benefits of extended LISA mission} Increasing the LISA mission length to 10 years results increases the number of detections to about 50-500 detections, a 60\% increase, because the number of detections scale approximately as $T_{\rm obs}^{0.5}$.
%    
    However, the real benefit is the improvement of the characterisation of the sources, since the relative error on the frequency derivative (which dominates the relative error in the chirp mass) scales as $T_{\rm obs}^{-2.5}$  for stationary sources (Eq.~\ref{eq:f_orb_dot_unc}).
    
    We find an increase of a factor 2.4 for the number of systems with chirp masses that can be measured better than 10\%. 
%    
    The number of systems with a sky localisation better than one degree increases by about 50\%.
%    
    The number of sources that can be unambiguously distinguished from WDWDs increases by almost a factor 2 (see Section~\ref{sec:WDWD_distinguish}).
    
    \item \textbf{Probing the black hole mass distribution and the lower mass gap:} We expect LISA to predominantly detect lower mass BHs (with 90\% of BHBH and BHNSs having BH masses lower than $11 \unit{M_\odot}$ in our fiducial simulations) in stark contrast to current ground-based detectors which are heavily biased towards high mass systems. This implies that LISA can potentially make important contributions to the debate about the existence of a lower mass gap (\citealt{Shao+2021} and see our Fig.~\ref{fig:lower_mass_gap_variation}).
    
    \item \textbf{Eccentricity distribution:} We find that for all DCO types a large fraction of detectable systems still have nonzero eccentricities ($e = 0.01$) when entering the LISA band, unlike what is expected for the more numerous WDWD LISA population. In particular, we find that this is the case for the vast majority of BHBHs and NSNSs and nearly half of BHNSs. Furthermore, over a fifth of detectable BHBHs have eccentricities large enough such that their primary gravitational wave emission occurs in a higher harmonic ($e > 0.3$).
    
    % We predict that for the observed population \BHBHNotCirc{}(\BHBHHighlyEccentric{}) of BHBHs, \BHNSNotCirc{}(\BHNSHighlyEccentric{}) of BHNSs and \NSNSNotCirc{}(\NSNSHighlyEccentric{})\% of NSNSs have eccentricity of $e > 0.01(0.3)$ at the start of the LISA mission.
    \item \sout{\textbf{Physics variations:} For BHBHs, we find that the detection rate is largely unaffected by changes in underlying physics assumptions, except for the optimistic CE scenario and increased Wolf-Rayet winds. Conversely, the BHNS detection rate varies widely with physics variation, ranging across three orders of magnitude. The NSNS detection rate shows some variation but only shows large changes under the assumption that case BB mass transfer is unstable and that core-collapse supernovae produce weaker kicks.} - We have now said most of this is point 1. We could additionally talk about \textit{which} assumptions are causing the largest variations.
    \item \textbf{Source identification:} For a 4(10)-year LISA mission we estimate that, of the detectable population, at least \BHBHNotWDWDFour{}(\BHBHNotWDWDTen{}) BHBHs, \BHNSNotWDWDFour{}(\BHNSNotWDWDTen{}) BHNSs and \NSNSNotWDWDFour{}(\NSNSNotWDWDTen{}) NSNSs will produce signals that are distinguishable from a signal produced by a WDWD. Additionally, we predict that we will be able to determine whether \BHBHEitherBHOrNSFour{}(\BHBHEitherBHOrNSTen{}) BHBHs, \BHNSEitherBHOrNSFour{}(\BHNSEitherBHOrNSTen{}) BHNSs and \NSNSEitherBHOrNSFour{}(\NSNSEitherBHOrNSTen{}) NSNSs are signals from a binary that contains a black hole.
    \item \textbf{Joint SKA-LISA detections:} We expect that \textit{if} every LISA detectable DCO contained a pulsar that is beaming towards Earth, SKA-1 would be able to detect at least 11 of these objects, whilst the increased number of detectable pulsars that could crowd the sky with SKA-2 sensitivity means that this total decreases to 6. This number is a pessimistic estimate and could be much higher if we consider that SKA and LISA could match their orbital frequency estimates to select the correct pulsar from a crowded field.
    \item \textbf{Importance of choice of MW model:} Many studies use Milky Way models that use fixed metallicity populations which are assigned irrespective of birth time or position, do not account for the inside-out growth of the thin disc and use constant star formation rates. The use of these simpler MW models may lead to an overestimation of the LISA NSNS detection rate and an underestimation of the BHNS detection rate. It may also introduce unphysical artifacts into DCO parameter distributions, particularly the mass distributions, which lead to inaccurate predictions.
\end{enumerate}


\software{We used LEGWORK to evolve sources over time and calculate signal-to-noise ratios, it is freely available at \url{https://legwork.readthedocs.io/en/latest/}. Simulations in this paper made use of the COMPAS rapid binary population synthesis code, which is freely available at \url{http://github.com/TeamCOMPAS/COMPAS} \citep{Stevenson+2017, Vigna-Gomez+2018, Broekgaarden+2019}. The simulations performed in this work were simulated with a COMPAS version that predates the publicly available code. Our version of the code is most similar to version 02.13.01 of the publicly available COMPAS code. Requests for the original code can be made to Floor Broekgaarden. The authors used {\sc{STROOPWAFEL}} from \citep{Broekgaarden+2019}, publicly available at \url{https://github.com/FloorBroekgaarden/STROOPWAFEL}\footnote{For the latest pip installable version of STROOPWAFEL please contact Floor Broekgaarden.}.
The authors also made use of Python from the Python Software Foundation. Python Language Reference, version 3.6. Available at \url{http://www.python.org}. In addition the following Python packages were used: matplotlib (\url{https://matplotlib.org/}), {NumPy} (\url{https://numpy.org/}), SciPy (\url{https://www.scipy.org/}), Jupyter Lab (\url{https://jupyter.org/}), seaborn (\url{https://seaborn.pydata.org/}), h5py (https://www.h5py.org/) and Astropy (\url{http://www.astropy.org}).  This research has made use of NASA’s Astrophysics Data System Bibliographic Services. We also made use of the computational facilities from the Harvard FAS Research Computing cluster.}

\acknowledgments{We thank Mike Lau, Floris Kummer, Eva Laplace, Rob Farmer, Katie Sharpe, Mathieu Renzo, Katie Brevik, Alberto Sesana, Valeriya Korol, Will Farr, Tyson Littenberg and the members of the COMPAS collaboration for insightful discussions. TW also thanks Terence Lovatt for his advice and help on improving an earlier draft of this project. This project was funded in part by the European Union’s Horizon 2020 research and innovation program from the European Research Council (ERC, Grant agreement No. 715063), and by the Netherlands Organization for Scientific Research (NWO) as part of the Vidi research program BinWaves with project number 639.042.728. We further acknowledge the Black Hole Initiative funded by a generous contribution of the John Templeton Foundation and the Gordon and Betty Moore Foundation.}

\clearpage

\bibliographystyle{aasjournal}
\bibliography{references}

\allowdisplaybreaks
\appendix
\section{Population Synthesis}\label{app:pop_synth}

In this section we summarise the main assumptions and settings that we use when performing population synthesis for this work. For a more general overview of every setting see \citet{Broekgaarden+2021}.

\subsection{Initial conditions}

We simulate between 1 and 100 million massive binaries for each of 50 metallicities equally spaced in log space between $Z \in [0.0001, 0.022]$, where $Z$ is the mass fraction of heavy elements. We simulate more binaries for higher metallicities so that large enough sample of DCOs at each metallicity (since DCOs are formed at a lower rate at higher metallicities). These metallicities span the allowed metallicity range for the original fitting formulae on which COMPAS is based \citep{Hurley+2000}. This is repeated for \nMinusOneModels{} physics variations (see Section \ref{sec:variation_assumptions}) and so in total over two billion binaries were simulated.

Each binary is sampled from initial distributions for the primary and secondary masses as well as the separation. The primary mass, that is the mass of the initially more massive star, is restricted to $m_1 \in [5, 150] \unit{M_{\odot}}$, which spans the range of interest for NS and BH formation in binary systems, and drawn from the \citet{Kroupa+2001} initial mass function (IMF), $p(m_1) \propto m_1^{-2.3}$. The secondary mass, $m_2$, is drawn using the initial mass ratio of the binary, $q \equiv m_2 / m_1$, which we assume to be uniform on $[0, 1]$, therefore $p(q) = 1$ \citep[e.g.\ consistent with][]{Sana+2012}. We additionally restrict the secondary masses $m_2 \ge 0.1 \unit{M_{\odot}}$, which is approximately the minimum mass for a main sequence star. We assume that the initial separation follows a flat in the log distribution with $p(a_i) \propto 1 / a_i$ and $a_i \in [0.01, 1000] \unit{AU}$ \citep{Opik+1924, Abt+1983}. We assume that all binary orbits are circular at birth to reduce the dimensions of initial parameters. Since we focus on post-interaction binaries which will have circularised after mass transfer we argue this is an reasonable assumption (as many studies have in the past) and is likely not critical for predicting detection rates \citep{Hurley+2002, deMink+2015}.

We apply the adaptive importance sampling algorithm STROOPWAFEL \citep{Broekgaarden+2019} to improve the yield of our sample. This algorithm increases the prevalence of target DCOs (BHBHs, BHNSs and NSNSs in this case) in the sample and assigns each a weight, $w$, which represents the probability of drawing the DCO without STROOPWAFEL in effect.

\subsection{Physical assumptions in our fiducial model}\label{app:fiducial_physics}

\textit{Stellar Evolution:} To follow the evolution of massive stars, COMPAS relies on fitting formulae by \citet{Hurley+2000} to detailed single star models by \citet{Pols+1998}. COMPAS models the evolution of stars that lose or gain mass closely following the algorithms originally described in \citet{Tout+1996} and \citet{Hurley+2002}.

\textit{Wind mass loss:} We follow the wind prescription from \citet{Belczynski+2008}, which was based on results from Monte Carlo radiative transfer simulation of \citet{Vink+2000, Vink+2001}. We use the wind mass loss rates from \citet{Vink+2001} for stars above $12500 \unit{K}$ and the rates from \citet{Hurley+2000} for cooler stars. Additionally, we use a separate, higher wind mass loss rate for luminous blue variable (LBV) stars, following \citet{Belczynski+2008}, to mimic observed LBV eruptions for stars with luminosities and effective temperatures above the Humphreys-Davidson limit. We use the Wolf-Rayet-like mass loss rate from \citet{Hamann+1998} with an additional metallicity scaling from \citet{Vink+2005} for helium stars, and set $f_{\rm WR} = 1$. See \citet{COMPAS:2021methodsPaper}, Section 3 for the explicit equations.

\textit{Mass Transfer:} In determining the stability of mass transfer we use the $\zeta$-prescription, which compares the radial response of the star with the response of the Roche lobe radius to the mass transfer \citep[e.g.][]{Hjellming+1987}. The mass transfer efficiency, $\beta \equiv \Delta M_{\rm acc} / \Delta M_{\rm don}$, is defined as the fraction of the mass transferred by the donor that is actually accreted by the accretor. We limit the maximum accretion rate for stars to $\Delta M_{\rm acc} / \Delta t \le 10 M_{\rm acc} / \tau_{\rm KH}$, where $\tau_{\rm KH}$ is the Kelvin-Helmholtz timescale of the star \citep{Paczynski+1972, Hurley+2002}. The maximum accretion rate for compact objects is limited to the Eddington accretion rate. If more mass than these rates is accreted then we assume that the excess is lost through isotropic re-emission in the vicinity of the accreting star \citep[e.g.][]{Massevitch+1975, Soberman+1997}. We assume that all mass transfer from a stripped post-helium-burning-star (case BB) onto a neutron star or black hole is unstable \citep{Tauris+2015}.

\textit{Common-Envelope:} A common-envelope phase follows dynamically unstable mass transfer and we parameterise this using the $\alpha$-$\lambda$ prescription from \citet{Webbink+1984} and \citet{deKool+1990}. We assume $\alpha = 1$, such that all of the gravitational binding energy is available for the ejection of the envelope. For $\lambda$ we use the fitting formulae from \citet{Xu+2010, Xu+2010a}. We assume that any Hertzsprung gap donor stars that initiate a common-envelope phase will not survive this phase due to a lack of a steep density gradient between the core and envelope \citep{Taam+2000, Ivanova+2004, Klencki+2021}. This follows the `pessimistic' common-envelope scenario \citep[c.f.][]{Belczynski+2007}. We remove any binaries where the secondary immediately fills its Roche lobe upon the conclusion of the common-envelope phase as we treat these as failed common-envelope ejections, likely leading to a stellar merger.

\textit{Supernovae:} We draw the remnant masses and natal kick magnitudes from different distributions depending on the type of supernova that occurs. For stars undergoing a general core-collapse supernova, we use the \textit{delayed} supernova remnant mass prescription from \citet{Fryer+2012}. The \textit{delayed} prescription does not reproduce a neutron star black hole mass gap and we use this as our default as it has been shown to provide a better fit for observed populations of DCOs \citep[e.g.][]{Vigna-Gomez+2018}. We draw the natal kick magnitudes from a Maxwellian velocity distribution with a one-dimensional root-mean-square velocity dispersion of $\sigma_{\rm rms}^{\rm 1D} = 265 \unit{km}{s^{-1}}$ \citep{Lyne+1994, Hobbs+2005}. We assume that stars with helium core masses between $1.6$--$2.25 \unit{M_{\odot}}$ \citep{Hurley+2002} experience electron-capture supernovae (ECSN) \citep{Nomoto+1984, Nomoto+1987, Ivanova+2008}. We set all remnant masses to $1.26 \unit{M_{\odot}}$ in this case as an approximation of the solution to Equation 8 of \citet{Timmes+1996}. For these supernovae, we set $\sigma_{\rm rms}^{\rm 1D} = 30 \unit{km}{s^{-1}}$ \citep[e.g.][]{Pfahl+2002, Podsiadlowski+2004}. We assume that stars that undergo case BB mass transfer \citep{Dewi+2002} experience extreme stripping which leads to an ultra-stripped supernova \citep{Tauris+2013, Tauris+2015}. For these supernovae we calculate the remnant mass using the \citet{Fryer+2012} prescription and use $\sigma_{\rm rms}^{\rm 1D} = 30 \unit{km}{s^{-1}}$ (as with ECSN). Stars with final helium core masses between $35$-$135 \unit{M_{\odot}}$ are presumed to undergo a pair-instability, or pulsational pair-instability supernova \citep[e.g.][]{Woosley+2007, Farmer+2019}. We follow the prescription from \citet{Marchant+2019} as implemented in \citep{Stevenson+2019} for these supernovae. We assume that kicks are isotropic in the frame of the collapsing star. We adopt a maximum neutron star mass of $2.5 \unit{M_{\odot}}$ \citep[e.g.][]{Kalogera+1996, Fryer+2015, Margalit+2017} for the fiducial model and change the \citet{Fryer+2012} prescription accordingly.

\subsection{Model variations} \label{sec:variation_assumptions}
In addition to our fiducial model for the formation of DCOs, we explore \nMinusOneModels{} other models in which we change various aspects of the mass transfer, common-envelope, supernova and wind mass loss physics assumptions in order to assess the effect of their uncertainties on the overall double compact object detection rates and distributions. Each of the models varies a single physics assumption (fiducial assumptions are outlined in Section~\ref{app:fiducial_physics}) and these models are outlined in Table~\ref{tab:physics_variations}.

Our fiducial model is labelled model \modFid{}. Models \modRangeMT{} focus on changes to the mass transfer physics assumptions. We explore the effect of fixing the mass transfer efficiency $\beta$ to a constant value, rather than allowing it to vary based on the maximum accretion rate. In models \modBetaLow{}, \modBetaMed{}, \modBetaHigh{}, in which we set the value of $\beta$ to $0.25$, $0.5$ and $0.75$ respectively. In model \modCaseBB{} we investigate the consequence of assuming that case BB mass transfer onto a neutron star or black hole is always stable rather than always unstable.

Models \modRangeCE{} focus on altering the common-envelope physics. We change the common-envelope efficiency parameter to $\alpha_{\rm CE} = 0.1, 0.5, 2.0, 10.0$ in models \modAlphaLowest{}, \modAlphaLow{}, \modAlphaHigh{} and \modAlphaHighest{} respectively. In model \modOpt{}, we relax our restriction that Hertzsprung gap donor stars cannot survive common-envelope events, thereby following the `optimistic' common-envelope scenario. We combine this with model \modCaseBB{} in model \modCaseBBOpt{}.

In models \modRangeSN{} we consider changes related to our assumptions about supernova physics. Model \modRapid{} uses the alternate \textit{rapid} remnant mass prescription from \citet{Fryer+2012} instead of the \textit{delayed} prescription. We change the maximum neutron star mass in models \modNSLow{} and \modNSHigh{} to $2$ and $3 \unit{M_{\odot}}$ respectively to account for the range of predicted maximum neutron star masses. Model \modNoPISN{} removes the implementation of pair-instability and pulsational pair-instability supernovae. In models \modSigLow{} and \modSigLower{} we decrease the root-mean-square velocity dispersion for core-collapse supernovae to explore the effect of lower kicks. Model \modNoBH{} removes the natal kick for all black holes.

Finally, in models \modRangeML{} we investigate the effect of changing our assumption about wind mass loss rates, specifically for Wolf-Rayet winds. We vary $f_{\rm WR}$ to $0.1$ and $5.0$ in models \modWRLow{} and \modWRHigh{} respectively. These values approximately span the current range of possible Wolf-Rayet wind efficiencies suggested from observations \citep[e.g.][]{Vink+2017, Hamann+2019, Shenar+2019, Miller-Jones+2021, vanSon+2021}.

\begin{table}[htb]
    \centering
    \begin{tabular}{cl}
        \hline \hline
        Model & Physics Variation \\
        \hline \hline
        \modFid & Fiducial (see Section~\ref{app:fiducial_physics}) \\
        \hline
        \modBetaLow & Fixed mass transfer efficiency of $\beta=0.25$ \\ 
        \modBetaMed & Fixed mass transfer efficiency of $\beta=0.5$  \\ 
        \modBetaHigh & Fixed mass transfer efficiency of $\beta=0.75$ \\ 
        \modCaseBB & Case BB mass transfer is always unstable \\
        \modCaseBBOpt & Model \modCaseBB{} + Model \modOpt{} \\
        \hline
        \modAlphaLowest & CE efficiency parameter $\alpha = 0.1$ \\
        \modAlphaLow & CE efficiency parameter $\alpha = 0.5$ \\
        \modAlphaHigh & CE efficiency parameter $\alpha = 2$   \\
        \modAlphaHighest & CE efficiency parameter $\alpha = 10$   \\
        \modOpt & HG donor stars initiating a CE survive CE \\
        \hline
        \modRapid & Fryer rapid SN remnant mass prescription \\
        \modNSLow & Maximum NS mass is fixed to $2\unit{M_{\rm \odot}}$ \\
        \modNSHigh & Maximum NS mass is fixed to $3\unit{M_{\rm \odot}}$ \\
        \modNoPISN & PISN and pulsational-PISN not implemented \\
        \modSigLow & $\sigma_{\rm{rms}}^{\rm{1D}}=100 \unit{km}{s^{-1}}$ for core-collapse supernova \\  
        \modSigLower & $\sigma_{\rm{rms}}^{\rm{1D}}=30  \unit{km}{s^{-1}}$ for core-collapse supernova \\ 
        \modNoBH & Black holes receive no natal kick \\
        \hline
        \modWRLow & Wolf-Rayet wind factor $f_{\rm WR} = 0.1$ \\
        \modWRHigh & Wolf-Rayet wind factor $f_{\rm WR} = 5.0$ \\
        \hline \hline
    \end{tabular}%
    \caption{A description of the \nModels{} binary population synthesis models used in this study. \modFid{} is the fiducial model, \modRangeMT{} change mass transfer physics, \modRangeCE{} change common-envelope physics , \modRangeSN{} change supernova physics and \modRangeML{} change wind mass loss \citep[c.f.][Table 2]{Broekgaarden+2021}.}
    \label{tab:physics_variations}
\end{table}


\section{Detection Rate Normalisation}\label{app:rate_normalisation}
In this section we explain the normalisation process that we refer to in Section~\ref{sec:gw_detection}. From each simulated instance of the Milky Way we extract the fraction of targets that are detectable, where we define a target as one of BHBH, BHNS or NSNS that merges in a Hubble time. To convert the detectable fraction to a detection rate for the Milky Way, we write that the \textit{number} of detectable targets in the Milky Way is
\begin{equation}\label{eq:norm_frame}
    N_{\rm detect} = f_{\rm detect} \cdot N_{\rm target, MW},
\end{equation}
where $f_{\rm detect}$ is the fraction of targets in the instance that were detectable and $N_{\rm target, MW}$ is the total number of targets that have been formed in the Milky Way's history. We can further break this total down into
\begin{equation}
    N_{\rm target, MW} = \avg{ \mathcal{R}_{\rm target} } \cdot M_{\rm SF, MW},
\end{equation}
where $\avg{ \mathcal{R}_{\rm target} }$ is the average number of targets formed per star forming mass and $M_{\rm SF, MW}$ is the star forming mass of the Milky Way, meaning the total mass of every star ever formed in the Milky Way.

\subsection{Average target formation rate}
Double compact object formation is metallicity dependent, so we find the average rate as the integral over metallicity, which is given by
\begin{equation}\label{eq:norm_avg_target_formation}
    \avg{ \mathcal{R}_{\rm target} } = \int_{Z_{\rm min}}^{Z_{\rm max}} p_{Z} \mathcal{R}_{\rm target, Z} \dd{Z},
\end{equation}
where $Z_{\rm min}, Z_{\rm max}$ are the minimum and maximum sampled metallicities, $p_Z$ is the probability of forming a star at the metallicity $Z$ (which can be found using the distribution in \citealp{Frankel+2018}) and $\mathcal{R}_{\rm target, Z}$ is the number of targets formed per star forming mass,
\begin{equation}
    \mathcal{R}_{\rm target, Z} =  \frac{N_{\rm target, Z}}{ M_{\rm SF, Z} }.
\end{equation}
In practice, this integral is instead approximated as a sum over the metallicity bins that we use in our simulation. The number of targets in our sample at a metallicity $Z$, $N_{\rm target, Z}$, can be written simply as the sum of the targets' weights:
\begin{equation}
    N_{\rm target, Z} = \sum_{i=1}^{N_{\rm binaries, Z}} w_i \theta_{\rm target, i},
\end{equation}
where $w_i$ is the binary's adaptive importance sampling weight assigned, $N_{\rm binaries, Z}$ is the number of binaries at metallicity $Z$ in our sample and $\theta_{\rm target, i}$ is only $1$ when the binary is a target and otherwise $0$.

The total star forming mass at a metallicity $Z$, $M_{\rm SF, Z}$, can be written as
\begin{equation}
    M_{\rm SF, Z} = \frac{\avg{m}_{\rm COMPAS, Z}}{f_{\rm trunc}} N_{\rm binaries, Z},
\end{equation}
where $\avg{m}_{\rm COMPAS}$ is the average star forming mass of a binary in a simulation using our cutoffs (discussed in Section~\ref{sec:COMPAS_explained}) and $f_{\rm trunc}$ is the fraction of the total stellar mass from which our COMPAS simulations sample, given our truncated mass and separation ranges (see Section~\ref{sec:COMPAS_explained}). These truncations mean that only $f_{\rm trunc} \approx 0.17$ of the stellar mass in the galaxy is sampled from.

\subsection{Total star forming mass in the Milky Way}
It is important to distinguish between the \textit{total} mass of every star formed over the entire history of the Milky Way and the \textit{current} stellar mass in the Milky Way. Many stars born in the Milky Way are no longer living and have lost much of their mass to stellar winds and supernovae, thus the current stellar mass in the Milky Way is an underestimate of the total star forming mass.

\citet{Licquia+2015} find that the total stellar mass today in the Milky Way is $6.08 \pm 1.14 \times 10^{10} \unit{M_{\odot}}$. This total includes all stars and stellar remnants (white dwarfs, neutrons stars and black holes) but \textit{excludes} brown dwarfs. We can write that the total mass of every star every formed in the Milky Way is
\begin{equation}\label{eq:m_SF_MW}
    M_{\rm SF, MW} = (6.08 \pm 1.14) \times 10^{10} \unit{M_{\rm \odot}} \cdot \frac{\avg{m}_{\rm SF, total}}{\avg{m}_{\rm SF, today}},
\end{equation}
where $\avg{m}_{\rm SF, total}$ is the average mass of a star over the history of the Milky Way and is defined as
\begin{equation}
    \avg{m}_{\rm SF, total} = \int_{0}^{t_{\rm MW}} p_{\rm birth}(\tau) \int_{0.01}^{200} \zeta(m)\ m \dd{m} \dd{\tau},
\end{equation}
where $t_{\rm MW}$ is the age of the Milky Way, $\zeta(m)$ is the \citet{Kroupa+2001} IMF function and $p_{\rm birth}(\tau)$ is the probability of a star being formed at a lookback time $\tau$ (Eq.~\ref{eq:thin_disc_tau}). $\avg{m}_{\rm SF, today}$ is the average mass of all stars and stellar remnants (excluding brown dwarfs) present in the Milky Way today is defined as follows (note that we integrate from $0.08$ not $0.01$ since observations of today's Milky Way mass exclude brown dwarfs)
\begin{equation}
    \avg{m}_{\rm SF, today} = \int_{0}^{t_{\rm MW}} p_{\rm birth}(\tau) \int_{0.08}^{200} \zeta(m)\ m_{\rm today} \dd{m} \dd{\tau},
\end{equation}
where $m_{\rm today}(m, Z, \tau)$ is the current mass of a star that was formed $\tau$ years ago at a metallicity $Z$. We calculate $m_{\rm today}(m, Z, \tau)$ by interpolating the final masses given by COMPAS for a grid of single stars over different masses and metallicities using the \citet{Fryer+2012} delayed prescription and default wind mass loss settings. For $Z$, we use the average star forming metallicity in the Milky Way at a lookback time $\tau$ using our galaxy model. Evaluating Equation~\ref{eq:m_SF_MW}, we find that the total mass of every star that has ever formed in the Milky Way is
\begin{align}
    M_{\rm SF, MW} &= (6.1 \pm 1.1) \times 10^{10} \unit{M_{\odot}} \cdot \frac{0.378 \unit{M_{\odot}}}{0.221 \unit{M_{\odot}}}, \nonumber \\
    &= (10.4 \pm 1.1) \times 10^{10} \unit{M_{\odot}},
\end{align}
an increase of approximately 70\% from the value still in stars today!

\subsection{Normalisation summary}
Finally, we can substitute Equations~\ref{eq:norm_avg_target_formation} and \ref{eq:m_SF_MW} into \ref{eq:norm_frame} and write that the overall normalisation of the detection rate is calculated as
\begin{align}
    N_{\rm detect} &= f_{\rm detect} \cdot 10.4 \times 10^{10} \unit{M_{\odot}} \nonumber \\
    &\times \sum_{Z=Z_{\rm min}}^{Z_{\rm max}} p_{Z} \qty(\sum_{i=1}^{N_{\rm binaries, Z}} w_i \theta_{\rm target, i}) \nonumber \\
    &\times \qty(\frac{\avg{m}_{\rm COMPAS, Z}}{f_{\rm trunc}} \sum_{i=1}^{N_{\rm binaries, Z}} w_i)^{-1}.
\end{align}

\section{Calculation of the uncertainties in the chirp mass for detectable sources}\label{app:chirp_mass_uncertainty}

How accurately the chirp mass of a detected binary can be determined depends on the signal to noise ratio, duration of the mission, its orbital frequency and the time derivative of the orbital frequency. 

Here we describe how we estimate the uncertainty of the chirp mass . First, consider the chirp mass, which can be expressed as
\begin{equation}
    \mathcal{M}_c = \frac{c^3}{G} \left( \frac{5 \pi}{48 n} \frac{\dot{f}_{n}}{F(e)} \right)^{3/5} \frac{1}{(2 \pi f_{\rm orb})^{11/5}},
\end{equation}
where ${f}_{n}$ is the frequency of the n-th harmonic,  $f_{\rm orb}$ is the orbital frequency, $\mathcal{M}_{c}$ is the chirp mass (defined in Eq.~\ref{eq:chirp_mass}), $e$ is the eccentricity and
\begin{equation}\label{eq:peters_f}
    F(e) = \frac{1 + \frac{73}{24} e^2 + \frac{37}{96} e^4}{(1 - e^2)^{7/2}},
\end{equation}
is the enhancement factor of gravitational wave emission for an eccentric binary over an otherwise identical circular binary \citep[][Eq.~17]{Peters+1963}. 
%
In practice, we will use the dominating harmonic, with $n=n_{\rm dom}$ and $f_n =  f_{\rm dom}$. The dominating harmonic for circular binaries is $n_{\rm dom} = 2$ and the dominating frequency is twice the orbital frequency. 

Therefore the chirp mass uncertainty can be estimated as
\begin{equation}\label{eq:chirp_mass_uncertainty}
    \frac{\Delta \mathcal{M}_c}{\mathcal{M}_c} = \frac{11}{5} \frac{\Delta f_{\rm orb}}{f_{\rm orb}} + \frac{3}{5} \frac{\Delta \dot{f}_{\rm dom}}{\dot{f}_{\rm dom}} + \frac{3}{5} \frac{\Delta F(e)}{F(e)},
\end{equation}
%where $f_{\rm dom}$ is the harmonic frequency with the strongest SNR ($f_{\rm dom} = n_{\rm dom} f_{\rm orb}$ and $n_{\rm dom} = 2$ for circular binaries) as this will provide the best measurement.

We estimate the frequency uncertainties using \citet{Takahashi+2002}, such that
\begin{align}\label{eq:f_orb_unc}
    \frac{\Delta f_{\rm orb}}{f_{\rm orb}} &= 4 \sqrt{3} \cdot \frac{1}{\rho} \frac{1}{T_{\rm obs}} \frac{1}{f_{\rm orb}}, \\
    \frac{\Delta \dot{f}_{\rm dom}}{\dot{f}_{\rm dom}} &= 6 \sqrt{5} \cdot \frac{1}{\rho} \left(\frac{1}{T_{\rm obs}} \right)^2 \frac{1}{\dot{f}_{\rm dom}},\label{eq:f_orb_dot_unc}
\end{align}
where $\rho$ is the signal-to-noise ratio and $T_{\rm obs}$ is the LISA mission length. We estimate the eccentricity certainty, $\Delta e$, following the methods of \citet{Lau+2020} and \citet{Korol+2021}, which use the relative SNRs of different harmonics to work out the eccentricity. We propagate this uncertainty such that
\begin{equation}
    \frac{\Delta F(e)}{F(e)} = \Delta e \cdot \frac{(1256 + 1608 e^2 + 111 e^4) e}{96 + 196 e^2 - 255 e^4 - 37 e^6}.
\end{equation}

We use Eq.~\ref{eq:chirp_mass_uncertainty} to calculate the chirp mass uncertainty for each DCO type in our sample and plot it in Fig.~\ref{fig:m_c_unc}.


\section{Assessing the impact of Milky Way model choices}\label{app:mw_changes}
The model that we use for the Milky Way adds several layers of complexity, accounting for the inside-out growth of the thin disc, using empirically informed star formation histories that are a function of time and assigning metallicities based on the position and age of binaries. In this section, we repeat our main analysis but instead apply a simpler model for the Milky Way in order to assess the effect of these added features. For this purpose, we use model for the Milky Way used in \citet{Breivik+2020} as this is representative of the models used in most previous works.

Their model can be summarised as follows: the Milky Way is assumed to comprise of three components, a thin disc, a thick disc and a bulge. The spatial distributions and relative masses for these components are given in \citet{McMillan+2011}. \citet{Breivik+2020} assume constant star formation over 10 Gyr for the thin disc, a 1 Gyr burst of star formation 11 Gyr ago for the thick disc and a 1 Gyr burst of star formation 10 Gyr ago for the bulge. A major difference is that only two metallicities are used and they are assigned to binaries independent of age or position. Binaries formed in the thin disc and bulge are assumed to have a metallicity of $Z = 0.02$ and those formed in the thick disc are assumed to have $Z = 0.003$.

We show the spatial metallicity distribution for this model in Fig.~\ref{fig:simple_mw} in the same form as Fig.~\ref{fig:galaxy_schematic} for ease of comparison between our models. The two main differences we can see between Fig.~\ref{fig:galaxy_schematic} and \ref{fig:simple_mw} are that the \citet{Breivik+2020} model is more centrally concentrated and only has two fixed metallicity populations.

\begin{figure}[htb]
    \centering
    \includegraphics[width=\columnwidth]{../figures/fig11_random_simple_galaxy.png}
    \caption{As Fig.~\ref{fig:galaxy_schematic} (right panel), but for the Milky Way model used in \citet{Breivik+2020}.  \href{https://github.com/TomWagg/detecting-DCOs-in-LISA/blob/main/paper/figures/fig11_random_simple_galaxy.png}{\faFileImage} \href{https://github.com/TomWagg/detecting-DCOs-in-LISA/blob/main/paper/figure_notebooks/galaxy_creation_station.ipynb}{\faBook}.}
    \label{fig:simple_mw}
\end{figure}

When applying this simpler Milky Way model in combination with our fiducial binary physics assumptions (model \modFid{}), we find that the expected number of detections for BHBHs, BHNSs and NSNSs for a 4-year LISA mission is $52$, $25$ and $17$ respectively. Thus the BHBH detection has decreased slightly compared to our main findings, whilst for BHNSs and NSNSs the rate has approximately halved and doubled respectively.

Moreover, the distribution of parameters within the population, particularly the mass distributions, are notably disparate. By using only two fixed metallicity populations, unphysical artifacts are introduced into distribution of DCO masses (Kummer et al. (in prep)). For example, in Fig.~\ref{fig:bh_mass_simple_mw}, we show the black hole mass distribution produced by the simulation using the simple Milky Way model. Despite the fact that these KDEs use the same bandwidth as Fig.~\ref{fig:fiducial_pdf_distributions}, the distributions show many more sharp transitions, which is a result of pileups occurring at specific masses for specific metallicities. Moreover, the lack of lower metallicities systems means that higher mass systems are not formed and so we see the distributions do not include a high mass tail such as in our fiducial results.

The unphysical artifacts present in the mass distributions can have far-reaching effects since the masses of DCOs affect most other parameters. The inspiral time and SNR are directly dependent on the mass, whilst the uncertainty estimates depend on the SNR. This means that the artifacts can affect the predictions for most distributions of LISA detectable populations.

Overall, we find that previous studies that use Milky Way models analogous to this simpler model may significantly underestimate the LISA BHNS rate whilst overestimating the NSNS detection rate. They may also miss higher mass systems (particular for BHNSs) and contain unphysical artifacts in their parameter distributions.

\begin{figure}[htb]
    \centering
    \includegraphics[width=\columnwidth]{../figures/fig12_mBH_simple_mw_variation.pdf}
    \caption{As Fig.~\ref{fig:fiducial_pdf_distributions} (top left panel), but for the Milky Way model used in \citet{Breivik+2020}. Dotted lines show the distribution from Fig.~\ref{fig:fiducial_pdf_distributions} for comparison. \href{https://github.com/TomWagg/detecting-DCOs-in-LISA/blob/main/paper/figures/fig12_mBH_simple_mw_variation.pdf}{\faFileImage} \href{https://github.com/TomWagg/detecting-DCOs-in-LISA/blob/main/paper/figure_notebooks/fiducial.ipynb}{\faBook}.}
    \label{fig:bh_mass_simple_mw}
\end{figure}

\section{Estimating the number of pulsars for a given sky area in SKA}\label{app:ska_area}

In this section, we perform some back-of-the-envelope calculations in order to estimate the number of pulsars that SKA will observe within a given sky area.

First, we consider how many pulsars SKA is likely to detect. \citet{Keane+2015} uses PSRPOPPy \citep{Bates+2014} to simulate the Milky Way pulsar population. They find that for SKA-1, approximately $10000$ pulsars will be discovered. The second phase of SKA, which should be in operation by the time of the LISA mission, would yield a total of $35000$-$41000$ pulsars \citep{Keane+2015}. We use the average, $38000$, in further estimates below. Moreover, we are only interested in pulsars that are part of a binary system. We estimate this pulsar binary fraction as the fraction of known pulsars that are in binaries using the ATNF Pulsar Catalogue\footnote{\url{https://www.atnf.csiro.au/research/pulsar/psrcat}} \citep{Manchester+2005}. $290$ of the $2872$ currently known pulsars are in binary systems and thus we estimate the binary fraction of pulsars as $10\%$. Therefore, we expect that SKA-1 and SKA-2 will detect approximately $1000$ and $3800$ binary pulsars respectively.

Next, we can find the total number of pulsars SKA will detect in a patch on the sky. The total sky area that the SKA mission covers is approximately $5700 \unit{deg^2}$, which is calculated by integrating over the sky for all Galactic longitudes and Galactic latitudes limited to $\abs{b} < 10^\circ$ and $\delta < 45^\circ$, which are the limits on SKA-mid \citep{Keane+2015}. If we assume that the pulsars are found uniformly across the sky, this means that roughly $0.2$ and $0.7$ binary pulsars are expected per square degree for SKA-1 and SKA-2 respectively. Note that the assumption of a uniform distribution is not realistic as pulsars will tend to be far more concentrated in the Galactic centre but we use it to provide a slightly optimistic estimate.

Overall, we therefore expect a single pulsar per $5.7 \unit{deg^2}$ and $1.5 \unit{deg^2}$ for SKA-1 and SKA-2 respectively, which correspond to angular resolutions of $\sigma_\theta = 1.3^\circ$ and $\sigma_\theta = 0.7^\circ$.

\clearpage
\onecolumngrid

\section{Supplementary material}

% \subsection{Detection rate table}

\begin{table*}[htb]
    \centering
    \caption{The number of detectable binaries in a 4- and 10-year LISA mission for the \nModels{} different model variations and each DCO type. The `All' column contains the total expected detections when summed over the three types. Each value shows the mean and the 1-$\sigma$ Poisson uncertainty. \href{https://github.com/TomWagg/detecting-DCOs-in-LISA/blob/main/paper/figure_notebooks/detections.ipynb}{\faBook}.}
    \begin{tabular}{cl|cccc|cccc}
        \hline
        \multirow{2}{*}{Model} & \multirow{2}{*}{DESC} & \multicolumn{4}{c|}{LISA detections (4 year)} & \multicolumn{4}{c}{LISA detections (10 year)} \\ \cline{3-10}
        & & {All} & {BHBH} & {BHNS} & {NSNS} &
        {All} &
        {BHBH} & {BHNS} & {NSNS} \\
        \hline
        A & Fiducial & \confinv{124.3}{11.3}{10.7} & \confinv{74.0}{9.0}{9.0} & \confinv{42.4}{6.4}{6.6} & \confinv{7.9}{2.9}{3.1} & \confinv{202.2}{14.2}{13.8} & \confinv{117.9}{10.9}{11.1} & \confinv{71.3}{8.3}{8.7} & \confinv{13.0}{4.0}{4.0}\\
        B & $\beta=0.25$ & \confinv{94.1}{10.1}{9.9} & \confinv{68.8}{7.8}{8.2} & \confinv{22.4}{4.4}{4.6} & \confinv{2.9}{1.9}{2.1} & \confinv{149.3}{12.3}{12.7} & \confinv{107.8}{10.8}{10.2} & \confinv{36.9}{5.9}{6.1} & \confinv{4.6}{1.6}{2.4}\\
        C & $\beta=0.5$ & \confinv{59.2}{8.2}{7.8} & \confinv{47.0}{7.0}{7.0} & \confinv{8.3}{3.3}{2.7} & \confinv{3.9}{1.9}{2.1} & \confinv{95.9}{9.9}{10.1} & \confinv{75.8}{8.8}{9.2} & \confinv{13.6}{3.6}{3.4} & \confinv{6.4}{2.4}{2.6}\\
        D & $\beta=0.75$ & \confinv{67.0}{8.0}{8.0} & \confinv{46.9}{6.9}{7.1} & \confinv{7.4}{2.4}{2.6} & \confinv{12.7}{3.7}{3.3} & \confinv{104.5}{10.5}{10.5} & \confinv{71.2}{8.2}{8.8} & \confinv{12.1}{3.1}{3.9} & \confinv{21.1}{4.1}{4.9}\\
        E & Case BB unstable & \confinv{76.7}{8.7}{8.3} & \confinv{69.3}{8.3}{8.7} & \confinv{7.3}{2.3}{2.7} & \confinv{0.2}{0.2}{0.8} & \confinv{121.4}{11.4}{10.6} & \confinv{109.3}{10.3}{10.7} & \confinv{11.8}{3.8}{3.2} & \confinv{0.4}{0.4}{0.6}\\
        F & E+K & \confinv{320.9}{17.9}{18.1} & \confinv{154.3}{12.3}{12.7} & \confinv{148.4}{12.4}{12.6} & \confinv{18.2}{4.2}{3.8} & \confinv{483.7}{21.7}{22.3} & \confinv{239.5}{15.5}{15.5} & \confinv{216.8}{14.8}{15.2} & \confinv{27.4}{5.4}{5.6}\\
        G & $\alpha_{\rm CE}=0.1$ & \confinv{40.2}{6.2}{6.8} & \confinv{27.9}{4.9}{5.1} & \confinv{2.1}{1.1}{1.9} & \confinv{10.2}{3.2}{2.8} & \confinv{64.1}{8.1}{7.9} & \confinv{43.9}{6.9}{7.1} & \confinv{3.5}{1.5}{1.5} & \confinv{16.7}{3.7}{4.3}\\
        H & $\alpha_{\rm CE}=0.5$ & \confinv{85.9}{8.9}{9.1} & \confinv{58.3}{7.3}{7.7} & \confinv{21.8}{4.8}{4.2} & \confinv{5.9}{2.9}{2.1} & \confinv{136.2}{11.2}{11.8} & \confinv{91.6}{9.6}{9.4} & \confinv{34.9}{5.9}{6.1} & \confinv{9.7}{2.7}{3.3}\\
        I & $\alpha_{\rm CE}=2.0$ & \confinv{133.3}{11.3}{11.7} & \confinv{67.6}{8.6}{8.4} & \confinv{38.0}{6.0}{6.0} & \confinv{27.7}{5.7}{5.3} & \confinv{218.4}{14.4}{14.6} & \confinv{109.6}{10.6}{10.4} & \confinv{62.7}{7.7}{8.3} & \confinv{46.0}{7.0}{7.0}\\
        J & $\alpha_{\rm CE}=10.0$ & \confinv{77.9}{8.9}{9.1} & \confinv{26.7}{4.7}{5.3} & \confinv{16.3}{4.3}{3.7} & \confinv{34.9}{5.9}{6.1} & \confinv{126.2}{11.2}{10.8} & \confinv{42.4}{6.4}{6.6} & \confinv{26.6}{5.6}{5.4} & \confinv{57.2}{7.2}{7.8}\\
        K & Optimistic
        CE & \confinv{218.3}{15.3}{14.7} & \confinv{151.5}{12.5}{12.5} & \confinv{56.8}{7.8}{7.2} & \confinv{9.9}{2.9}{3.1} & \confinv{341.5}{18.5}{18.5} & \confinv{229.7}{14.7}{15.3} & \confinv{96.1}{10.1}{9.9} & \confinv{15.8}{3.8}{4.2}\\
        L & Fryer Rapid SN & \confinv{127.3}{11.3}{11.7} & \confinv{50.4}{7.4}{6.6} & \confinv{70.2}{8.2}{8.8} & \confinv{6.7}{2.7}{2.3} & \confinv{204.9}{13.9}{14.1} & \confinv{76.6}{8.6}{8.4} & \confinv{117.4}{10.4}{10.6} & \confinv{10.8}{2.8}{3.2}\\
        M & Max $m_{\rm NS}$
        $2.0 \, {\rm M_{\odot}}$ & \confinv{133.5}{11.5}{11.5} & \confinv{96.2}{10.2}{9.8} & \confinv{30.1}{5.1}{5.9} & \confinv{7.2}{2.2}{2.8} & \confinv{215.2}{14.2}{14.8} & \confinv{153.7}{12.7}{12.3} & \confinv{49.9}{6.9}{7.1} & \confinv{11.6}{3.6}{3.4}\\
        N & Max $m_{\rm NS}$
        $3.0 \, {\rm M_{\odot}}$ & \confinv{118.4}{10.4}{10.6} & \confinv{58.3}{7.3}{7.7} & \confinv{51.9}{6.9}{7.1} & \confinv{8.2}{3.2}{2.8} & \confinv{189.9}{13.9}{14.1} & \confinv{91.6}{9.6}{9.4} & \confinv{84.8}{8.8}{9.2} & \confinv{13.5}{3.5}{3.5}\\
        O & No PISN & \confinv{126.7}{11.7}{11.3} & \confinv{75.3}{8.3}{8.7} & \confinv{43.4}{6.4}{6.6} & \confinv{8.0}{3.0}{3.0} & \confinv{205.7}{14.7}{14.3} & \confinv{120.5}{10.5}{10.5} & \confinv{72.3}{8.3}{8.7} & \confinv{12.8}{3.8}{3.2}\\
        P & $\sigma_{\rm cc}$
        $100 \, {\rm km s^{-1}}$ & \confinv{184.6}{13.6}{13.4} & \confinv{82.7}{8.7}{9.3} & \confinv{86.6}{9.6}{9.4} & \confinv{15.4}{4.4}{3.6} & \confinv{300.8}{17.8}{17.2} & \confinv{130.1}{11.1}{11.9} & \confinv{145.1}{12.1}{11.9} & \confinv{25.6}{4.6}{5.4}\\
        Q & $\sigma_{\rm cc}$
        $30 \, {\rm km s^{-1}}$ & \confinv{268.3}{16.3}{16.7} & \confinv{91.8}{9.8}{9.2} & \confinv{142.9}{11.9}{12.1} & \confinv{33.6}{5.6}{5.4} & \confinv{426.8}{20.8}{21.2} & \confinv{142.9}{11.9}{12.1} & \confinv{229.0}{15.0}{15.0} & \confinv{54.9}{7.9}{7.1}\\
        R & No BH
        kicks & \confinv{230.2}{15.2}{14.8} & \confinv{90.8}{9.8}{9.2} & \confinv{132.1}{11.1}{11.9} & \confinv{7.2}{2.2}{2.8} & \confinv{372.7}{19.7}{19.3} & \confinv{142.3}{12.3}{11.7} & \confinv{218.6}{14.6}{14.4} & \confinv{11.8}{3.8}{3.2}\\
        S & $f_{\rm WR} = 0.1$ & \confinv{118.5}{10.5}{10.5} & \confinv{75.7}{8.7}{8.3} & \confinv{34.0}{6.0}{6.0} & \confinv{8.8}{2.8}{3.2} & \confinv{182.5}{13.5}{13.5} & \confinv{112.4}{10.4}{10.6} & \confinv{55.8}{7.8}{7.2} & \confinv{14.3}{4.3}{3.7}\\
        T & $f_{\rm WR} = 5$ & \confinv{29.7}{5.7}{5.3} & \confinv{5.7}{2.7}{2.3} & \confinv{15.5}{3.5}{3.5} & \confinv{8.5}{2.5}{2.5} & \confinv{48.8}{6.8}{7.2} & \confinv{9.2}{3.2}{2.8} & \confinv{26.2}{5.2}{4.8} & \confinv{13.4}{3.4}{3.6}\\
        \hline
    \end{tabular}
    \label{tab:detection_rates}
\end{table*}

% \subsection{Distribution of eccentric sources on sensitivity curve}
\begin{figure*}[b]
    \centering
    \includegraphics[width=\textwidth]{fig13_dcos_on_sc_eccentric_colours.png}
    \caption{As the bottom panels of Fig.~\ref{fig:dcos_on_sc}, but without the density distributions and scatter points are coloured by their eccentricity. We show eccentric sources are located in an offshoot below the $30 \unit{kpc}$ around $2 \unit{mHz}$. \href{https://github.com/TomWagg/detecting-DCOs-in-LISA/blob/main/paper/figures/fig13_dcos_on_sc_eccentric_colours.png}{\faFileImage} \href{https://github.com/TomWagg/detecting-DCOs-in-LISA/blob/main/paper/figure_notebooks/sensitivity_curve.ipynb}{\faBook}.}
    \label{fig:dcos_on_sc_ecc_col}
\end{figure*}

% \subsection{Formation channels}

\begin{figure}[p]
    \centering
    \includegraphics[height=0.85\textheight]{fig15_formation_channels.pdf}
    \caption{Fraction of each DCO type that is formed through different formation channels for all physics variations. Channels are described in detail in \citet{Broekgaarden+2021}. The classic, single core CEE and double core CEE channels all require at least one common-envelope event whilst only `only stable' consists of only stable mass transfer and `other' contains the remaining binaries which are mainly formed from `lucky' supernova kicks that shrink the binary. \href{https://github.com/TomWagg/detecting-DCOs-in-LISA/blob/main/paper/figures/fig15_formation_channels.pdf}{\faFileImage} \href{https://github.com/TomWagg/detecting-DCOs-in-LISA/blob/main/paper/figure_notebooks/formation_channels.ipynb}{\faBook}.}
    \label{fig:formation_channels}
\end{figure}

% \subsection{Properties of detectable systems (progenitors and at DCO formation)}
\begin{figure*}[ht]
    \centering
    \includegraphics[width=\textwidth]{progenitor_distributions.pdf}
    \includegraphics[width=0.66\textwidth]{dco_formation_distributions.pdf}
    \caption{As Fig.~\ref{fig:fiducial_pdf_distributions}, but for the properties of the progentiors of detectable systems (top row) and the properties of the detectable systems at DCO formation (bottom row). Note that the $m_{1, {\rm ZAMS}}$ panel uses a log scaled $y$-axis. \href{https://github.com/TomWagg/detecting-DCOs-in-LISA/blob/main/paper/figures/progenitor_distributions.pdf}{\faFileImage} \href{https://github.com/TomWagg/detecting-DCOs-in-LISA/blob/main/paper/figures/dco_formation_distributions.pdf}{\faFileImage} \href{https://github.com/TomWagg/detecting-DCOs-in-LISA/blob/main/paper/figure_notebooks/fiducial.ipynb}{\faBook}.}
    \label{fig:progenitor_and_dco_properties}
\end{figure*}

% \subsection{Detection rates relative to fiducial model}
\begin{figure*}[hb]
    \centering
    \includegraphics[width=\textwidth]{fig14_dco_relative_rates.pdf}
    \caption{Similar to Fig.~\ref{fig:detection_rates}, this shows the \textit{relative} detection rates compared to the fiducial model in order to emphasise how much each DCO is affected by model variations compared to other DCOs. The scatter points show the mean and the shaded areas show the 1-$\sigma$ uncertainty range. \href{https://github.com/TomWagg/detecting-DCOs-in-LISA/blob/main/paper/figures/fig14_dco_relative_rates.pdf}{\faFileImage} \href{https://github.com/TomWagg/detecting-DCOs-in-LISA/blob/main/paper/figure_notebooks/detections.ipynb}{\faBook}.}
    \label{fig:dco_relative_rates}
\end{figure*}

% \begin{sidewaystable}
%     \centering
%     \begin{tabular}{c|ccccccccc|ccccccccc}
%         \hline
%         \multirow{3}{*}{Model} & \multicolumn{9}{c|}{4-year Mission} & \multicolumn{9}{c}{10-year Mission} \\
%         \cline{2-9} \cline{10-19}
%         & \multicolumn{3}{c|}{$\Delta \mathcal{M}_c / \mathcal{M}_c < 0.1$} & \multicolumn{3}{c|}{$\sigma_\theta < 0.1 \unit{deg}$} & \multicolumn{3}{c|}{Distinguishable} & \multicolumn{3}{c|}{$\Delta \mathcal{M}_c / \mathcal{M}_c < 0.1$} & \multicolumn{3}{c|}{$\sigma_\theta < 0.1 \unit{deg}$} & \multicolumn{3}{c}{Distinguishable}  \\
%         \cline{2-9} \cline{10-19}
%         & \scriptsize{BHBH} & \scriptsize{BHNS} & \scriptsize{NSNS} & \scriptsize{BHBH} & \scriptsize{BHNS} & \scriptsize{NSNS} & \scriptsize{BHBH} & \scriptsize{BHNS} & \scriptsize{NSNS} & \scriptsize{BHBH} & \scriptsize{BHNS} & \scriptsize{NSNS} & \scriptsize{BHBH} & \scriptsize{BHNS} & \scriptsize{NSNS} & \scriptsize{BHBH} & \scriptsize{BHNS} & \scriptsize{NSNS} \\
%         \hline
%         A & \scriptsize{42} & \scriptsize{42} & \scriptsize{42} & \scriptsize{42} & \scriptsize{42} & \scriptsize{42} & \scriptsize{42} & \scriptsize{42} & \scriptsize{42} & \scriptsize{42} & \scriptsize{42} & \scriptsize{42} & \scriptsize{42} & \scriptsize{42} & \scriptsize{42} & \scriptsize{42} & \scriptsize{42} & \scriptsize{42} \\
%         \hline
%     \end{tabular}
%     \caption{A table}
% \end{sidewaystable}

\begin{sidewaystable}
    \centering
    \begin{tabular}{c|ccccccccc|ccccccccc}
        \hline
        \multirow{3}{*}{Model} & \multicolumn{9}{c|}{4-year Mission} & \multicolumn{9}{c}{10-year Mission} \\
        \cline{2-9} \cline{10-19}
        & \multicolumn{3}{c|}{$\Delta \mathcal{M}_c / \mathcal{M}_c < 0.1$} & \multicolumn{3}{c|}{$\sigma_\theta < 0.1 \unit{deg}$} & \multicolumn{3}{c|}{Distinguishable} & \multicolumn{3}{c|}{$\Delta \mathcal{M}_c / \mathcal{M}_c < 0.1$} & \multicolumn{3}{c|}{$\sigma_\theta < 0.1 \unit{deg}$} & \multicolumn{3}{c}{Distinguishable}  \\
        \cline{2-9} \cline{10-19}
        & \scriptsize{BHBH} & \scriptsize{BHNS} & \scriptsize{NSNS} & \scriptsize{BHBH} & \scriptsize{BHNS} & \scriptsize{NSNS} & \scriptsize{BHBH} & \scriptsize{BHNS} & \scriptsize{NSNS} & \scriptsize{BHBH} & \scriptsize{BHNS} & \scriptsize{NSNS} & \scriptsize{BHBH} & \scriptsize{BHNS} & \scriptsize{NSNS} & \scriptsize{BHBH} & \scriptsize{BHNS} & \scriptsize{NSNS} \\
        \hline
        A & 0.15 & 0.11 & 0.68 & 0.23 & 0.15 & 0.65 & 0.29 & 0.24 & 0.61 & 0.12 & 0.10 & 0.59 & 0.18 & 0.12 & 0.53 & 0.22 & 0.19 & 0.60 \\
B & 0.15 & 0.10 & 0.68 & 0.22 & 0.16 & 0.66 & 0.31 & 0.23 & 0.48 & 0.13 & 0.10 & 0.59 & 0.17 & 0.13 & 0.55 & 0.24 & 0.20 & 0.48 \\
C & 0.15 & 0.10 & 0.69 & 0.20 & 0.14 & 0.68 & 0.29 & 0.24 & 0.49 & 0.13 & 0.10 & 0.59 & 0.16 & 0.12 & 0.58 & 0.24 & 0.20 & 0.49 \\
D & 0.18 & 0.09 & 0.64 & 0.21 & 0.13 & 0.65 & 0.31 & 0.24 & 0.44 & 0.16 & 0.08 & 0.54 & 0.17 & 0.11 & 0.53 & 0.25 & 0.20 & 0.47 \\
E & 0.15 & 0.12 & 0.70 & 0.14 & 0.15 & 0.76 & 0.22 & 0.25 & 0.76 & 0.12 & 0.10 & 0.61 & 0.11 & 0.14 & 0.67 & 0.17 & 0.22 & 0.74 \\
F & 0.19 & 0.09 & 0.63 & 0.08 & 0.22 & 0.91 & 0.04 & 0.35 & 0.96 & 0.16 & 0.07 & 0.53 & 0.07 & 0.24 & 0.86 & 0.04 & 0.36 & 0.96 \\
G & 0.15 & 0.10 & 0.68 & 0.17 & 0.17 & 0.75 & 0.27 & 0.24 & 0.55 & 0.13 & 0.09 & 0.59 & 0.14 & 0.14 & 0.66 & 0.22 & 0.20 & 0.57 \\
H & 0.14 & 0.11 & 0.69 & 0.19 & 0.20 & 0.70 & 0.32 & 0.24 & 0.37 & 0.12 & 0.09 & 0.59 & 0.16 & 0.18 & 0.60 & 0.26 & 0.20 & 0.40 \\
I & 0.14 & 0.10 & 0.69 & 0.24 & 0.16 & 0.65 & 0.28 & 0.24 & 0.64 & 0.12 & 0.09 & 0.60 & 0.18 & 0.13 & 0.54 & 0.22 & 0.19 & 0.63 \\
J & 0.14 & 0.11 & 0.70 & 0.17 & 0.16 & 0.75 & 0.29 & 0.24 & 0.49 & 0.12 & 0.10 & 0.60 & 0.13 & 0.14 & 0.67 & 0.23 & 0.20 & 0.51 \\
K & 0.19 & 0.08 & 0.64 & 0.25 & 0.15 & 0.63 & 0.25 & 0.24 & 0.67 & 0.16 & 0.08 & 0.53 & 0.18 & 0.12 & 0.53 & 0.19 & 0.19 & 0.63 \\
L & 0.18 & 0.08 & 0.63 & 0.22 & 0.15 & 0.68 & 0.27 & 0.24 & 0.66 & 0.16 & 0.08 & 0.53 & 0.17 & 0.12 & 0.57 & 0.21 & 0.21 & 0.67 \\
M & 0.14 & 0.13 & 0.72 & 0.23 & 0.16 & 0.64 & 0.28 & 0.24 & 0.59 & 0.12 & 0.11 & 0.62 & 0.19 & 0.13 & 0.53 & 0.23 & 0.20 & 0.60 \\
N & 0.16 & 0.10 & 0.66 & 0.22 & 0.14 & 0.66 & 0.28 & 0.23 & 0.61 & 0.14 & 0.09 & 0.57 & 0.18 & 0.13 & 0.55 & 0.21 & 0.19 & 0.60 \\
O & 0.15 & 0.11 & 0.69 & 0.23 & 0.15 & 0.65 & 0.28 & 0.24 & 0.61 & 0.13 & 0.10 & 0.59 & 0.18 & 0.12 & 0.54 & 0.22 & 0.19 & 0.61 \\
P & 0.19 & 0.09 & 0.64 & 0.24 & 0.14 & 0.64 & 0.27 & 0.22 & 0.61 & 0.15 & 0.08 & 0.53 & 0.19 & 0.12 & 0.53 & 0.22 & 0.19 & 0.60 \\
Q & 0.18 & 0.09 & 0.64 & 0.26 & 0.15 & 0.60 & 0.29 & 0.23 & 0.57 & 0.17 & 0.08 & 0.54 & 0.20 & 0.12 & 0.49 & 0.22 & 0.19 & 0.56 \\
R & 0.15 & 0.10 & 0.68 & 0.24 & 0.15 & 0.63 & 0.27 & 0.23 & 0.60 & 0.12 & 0.10 & 0.59 & 0.19 & 0.12 & 0.50 & 0.21 & 0.18 & 0.59 \\
S & 0.18 & 0.08 & 0.65 & 0.20 & 0.13 & 0.68 & 0.23 & 0.22 & 0.71 & 0.16 & 0.08 & 0.54 & 0.17 & 0.12 & 0.57 & 0.18 & 0.19 & 0.70 \\
T & 0.16 & 0.09 & 0.70 & 0.23 & 0.15 & 0.62 & 0.33 & 0.25 & 0.43 & 0.13 & 0.08 & 0.60 & 0.18 & 0.12 & 0.50 & 0.26 & 0.20 & 0.45 \\
        \hline
    \end{tabular}
    \caption{A table}
\end{sidewaystable}


\end{document}
