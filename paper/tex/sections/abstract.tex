Future searches for gravitational waves from space will be sensitive to double compact objects (DCOs) in our Milky-Way. We present new simulations of the populations of double black holes (BHBHs), black hole neutron stars (BHNSs) and double neutron stars (NSNSs) that will be detectable by the planned space-based gravitational wave detector LISA. 
%
For our estimates, we use an empirically-informed model of the metallicity dependent star formation history of the Milky Way. We populate using an extensive suite of binary population-synthesis predictions for varying assumptions relating to mass transfer, common-envelope, supernova kicks, remnant masses and wind mass loss physics. 


For a 4(10)-year LISA mission, we predict between \rangeFourYear{}(\rangeTenYear{}) detections over these variations, out of which 6-154(9-238) are BHBHs, 2-198(3-289) are BHNSs and 3-35(4-57) are NSNSs.
%
We discuss how the variations in the physics assumptions alter the distribution of properties of the detectable systems, even when the detection rates are unchanged. In particular we discuss the observable characteristics such as the chirp mass, eccentricity and sky localisation and how the BHBH, BHNS and NSNS populations can be distinguished, both from each other and from the more numerous double white dwarf population. 
%
We further discuss the possibility of multi-messenger observations of pulsar populations with the Square Kilometre Array (SKA) and assess the benefits of extending the LISA mission. 

