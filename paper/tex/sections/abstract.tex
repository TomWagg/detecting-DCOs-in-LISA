We present predictions for the properties of the population of Galactic double black holes (BHBHs), black hole neutron stars (BHNSs) and double neutron stars (NSNSs) that will be detectable by the planned space-based gravitational wave detector LISA. We use rapid population synthesis to produce an extensive sample of double compact objects (DCOs) and combine this with an empirically-informed model to distribute them in a Milky Way-like galaxy based on their birth metallicity. We investigate the dependence of our results upon underlying physics assumptions by comparing the results of \nModels{} physics variations that vary assumptions relating to mass transfer, common-envelope, supernova and wind mass loss physics. We predict that for a 4(10)-year mission, LISA will (typically) detect about \BHBHFourYear{}(\BHBHTenYear{}) BHBHs, \BHNSFourYear{}(\BHNSTenYear{}) BHNSs, \NSNSFourYear{}(\NSNSTenYear{}) NSNSs. The predicted BHBH rate remains notably consistent under different physics assumptions, whilst in contrast the BHNS and NSNS rates each vary over 3 orders of magnitude. We discuss observable characteristics that could be used to distinguish the aforementioned DCOs from the more numerous double white dwarf population as well as for disentangling the BHBH, BHNS and NSNS populations from each other. We additionally assess the possibility of multi-messenger observations of pulsar populations by combining the capabilities of LISA and SKA.