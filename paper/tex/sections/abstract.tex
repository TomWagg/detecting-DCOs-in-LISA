We present predictions for the properties of the LISA detectable population of Galactic double black holes (BHBH), black hole neutron stars (BHNS) and double neutron stars (NSNS). We use an extensive sample of double compact objects (DCOs) produced using rapid population synthesis in tandem with an empirically-informed analytical model of the Milky Way that accounts for the chemical enrichment history of the galaxy. We investigate the dependence of our results upon underlying physics assumptions by comparing the results of \nModels{} physics variations that vary assumptions relating to mass transfer, common envelope, supernova and wind mass loss physics. We find that for a 4(10)-year mission, LISA will detect on average \BHBHFourYear{}(\BHBHTenYear{}) BHBHs, \BHNSFourYear{}(\BHNSTenYear{}) BHNSs, \NSNSFourYear{}(\NSNSTenYear{}) NSNSs. The BHBH rate remains notably consistent under different physics assumptions, whilst in contrast the BHNS and NSNS rates each vary over 3 orders of magnitude. We discuss potential strategies for distinguishing this population from the more numerous double white dwarf population as well as for separating the BHBH, BHNS and NSNS populations from each other. We additionally assess the possibility of joint SKA-LISA detections for systems that contain pulsars.