The Laser Interferometer Space Antenna (LISA) presents an exciting new lens through which to view the realm of gravitational waves. Detections of double compact objects with LISA may help to constrain uncertainties in binary evolution and understand the prevalence of electromagnetic counterparts to gravitational wave events. We estimate the number of Galactic DCOs that will be detected by LISA using a sample generated with the rapid population synthesis code COMPAS and a model of the Milky Way that includes not only the radial birth profiles but also the star formation and enrichment histories. We explore the effects of varying 15 physical assumptions that focus on key uncertainties in our understanding of mass transfer, common envelope and supernova physics. For our fiducial model, we find that over a 4 year mission, LISA will detect 26 BHBHs, 27 BHNSs and 12 NSNSs. These detection rates increase to 41, 45 and 19 respectively for a 10 year mission length. We present the distributions of the characteristics of the detectable binaries and explore how our different models affect them. We also discuss how well we can distinguish between different DCO types and methods for doing so.