Future searches for gravitational waves from space will be sensitive to double compact objects (DCOs) in our Milky Way. We present new simulations of the populations of double black holes (BHBHs), black hole neutron stars (BHNSs) and double neutron stars (NSNSs) that will be detectable by the planned space-based gravitational wave detector LISA. 
%
For our estimates, we use an empirically-informed model of the metallicity dependent star formation history of the Milky Way. We populate it using an extensive suite of binary population-synthesis predictions for varying assumptions relating to mass transfer, common-envelope, supernova kicks, remnant masses and wind mass loss physics. \edit1{Each model in the grid has been recently shown to be consistent with current constraints on the inferred overall GW rates from ground-based detectors.}

For a 4(10)-year LISA mission, we predict between \rangeFourYear{}(\rangeTenYear{}) detections over these variations, out of which 6-154(9-238) are BHBHs, 2-198(3-289) are BHNSs and 3-35(4-57) are NSNSs.
%
\edit1{We expect that about 50\%(60\%) can be distinguished from WDWD sources, based on their mass or eccentricity and localisation. Specifically, for about 10\%(15\%) we expect to be able to determine chirp masses better than 10\%. For 13\%(13\%) we expect sky-localisations better than 1 degree.} We discuss how the variations in the physics assumptions alter the distribution of properties of the detectable systems, even when the detection rates are unchanged. We further discuss the possibility of multi-messenger observations of pulsar populations with the Square Kilometre Array (SKA) and assess the benefits of extending the LISA mission. 

