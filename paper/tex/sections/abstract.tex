We present models for the properties of the population of Galactic double black holes (BHBHs), black hole neutron stars (BHNSs) and double neutron stars (NSNSs) that will be detectable by the planned space-based gravitational wave detector LISA. We use rapid population synthesis to produce an extensive sample of double compact objects (DCOs) and combine this with a new empirically-informed model to distribute them in a Milky Way-like galaxy based on their birth metallicity. For our fiducial model, we predict that on average, for a 4(10)-year mission, LISA will detect about \BHBHFourYear{}(\BHBHTenYear{}) BHBHs, \BHNSFourYear{}(\BHNSTenYear{}) BHNSs, \NSNSFourYear{}(\NSNSTenYear{}) NSNSs. We investigate the dependence of our results upon underlying physics assumptions by comparing the results of \nModels{} physics variations that vary assumptions relating to mass transfer, common-envelope, supernova and wind mass loss physics. For these variations, in a 4(10)-year mission, the total detection rate across all three DCO types ranges between \rangeFourYear{}(\rangeTenYear{}).
%We find that uncertainties related to the onset and the efficiency of ejection of the common-envelope, the magnitude of natal kicks and the strength of Wolf-Rayet winds each have a strong impact on the rate predictions.
We show that altering the underlying physics assumptions can lead to contrasting population parameter distributions, even when detection rates are unchanged. We discuss observable characteristics that could be used to distinguish the aforementioned DCOs from the more numerous double white dwarf population, as well as for disentangling the BHBH, BHNS and NSNS populations from each other. We additionally assess the possibility of multi-messenger observations of pulsar populations by combining the capabilities of LISA and the Square Kilometre Array (SKA).