To produce predictions for the DCOs that are detectable with LISA, we use a synthesised population of DCOs, simulated using the methods described in Section~\ref{sec:COMPAS_explained}. In Section~\ref{sec:galaxy_synthesis} we describe our model for the Milky Way and how we place DCOs in randomly sampled Milky Way instances. We evolve the orbit of each DCO in a Milky Way instance up to the LISA mission and calculate the detection rate for that instance using the methods presented in Section~\ref{sec:gw_detection}.

\subsection{Binary population synthesis}\label{sec:COMPAS_explained}

We use the grid of \nModels{} binary population synthesis simulations recently presented in \citet{Broekgaarden+2021,Broekgaarden+2021b}. This grid of simulations is synthesised using the rapid population synthesis code \href{https://compas.science}{COMPAS} \citep{Stevenson+2017,Vigna-Gomez+2018,Stevenson+2019,Broekgaarden+2019}. COMPAS follows the approach of the population synthesis code BSE \citep{Hurley+2000,Hurley+2002} and uses fitting formula and rapid algorithms to efficiently predict the final fate of binary systems. The code is open source and documented in the papers listed above, the online documentation\footnote{\url{https://compas.science}} and in the methods paper \citep{COMPAS:2021methodsPaper}. \edit1{The model we refer to as our fiducial model allows for BHs with masses in the heavily debated mass gap, but we also consider \nModels{} physics variations. \citet{Broekgaarden+2021b} showed that all of these physics variations are still consistent with current constraints on the overall inferred GW rates, at least when considering large variations of the assumed metallicity specific cosmic star formation rate \citep{Abbott+2021_GWTC3}}. We summarise the main assumptions and settings relevant for this work in Appendix~\ref{app:pop_synth}.

The result of the simulations is a sample of binaries, which, for each metallicity $Z$, have $N_{\rm binaries}$ binaries with parameters
\begin{equation}
    \mathbf{b}_{{Z, i}} = \{m_1, m_2, a_{\rm DCO}, e_{\rm DCO}, t_{\rm evolve}, t_{\rm inspiral}, w\},
\end{equation}
for $i = 1, 2, \dots, N_{\rm binaries}$, where $m_1$ and $m_2$ are the primary and secondary masses, $a_{\rm DCO}$ and $e_{\rm DCO}$ are the semi-major axis and eccentricity at the moment of double compact object (DCO) formation, $t_{\rm evolve}$ is the time between the binary's zero-age main sequence and DCO formation, $t_{\rm inspiral}$ is the time between DCO formation (that is, immediately after the second supernova in the system) and gravitational-wave merger and $w$ is the adaptive importance sampling weight assigned by STROOPWAFEL \cite[][Eq.~7]{Broekgaarden+2019}. We sample from these sets of parameters when creating synthetic galaxies.

\subsection{Galaxy synthesis}\label{sec:galaxy_synthesis}

In order to estimate a detection rate of DCOs with statistical uncertainties, we create a series of random instances of the Milky Way, each populated with a subsample drawn (with replacement) from the synthesised binaries described in Section~\ref{sec:COMPAS_explained}.

Most previous studies that predict a detection rate for LISA place binaries in the Milky Way independently of their age or evolution. We improve upon this as the first study to use an empirically-informed analytical model of the Milky Way that takes into account the galaxy's enrichment history by applying the metallicity-radius-time relation from \citet{Frankel+2018}. Those authors developed this relation in order to measure the global efficiency of radial migration in the Milky Way and calibrated it using a sample of red clump stars measured with APOGEE \citep{Majewski+2017}. \edit1{We assess the impact of using this improved Milky Way model in Appendix~\ref{app:mw_changes} and the effect of Galactic models on LISA predictions has been investigated more generally in \citet{Storck+2022}}.

In Section~\ref{sec:mw_model}, we outline our model for the Milky Way and in Section~\ref{sec:combining_pop_gal} we explain how we combine our population of synthesised DCOs with this Milky Way model.

\subsubsection{Milky Way model}\label{sec:mw_model}

\begin{figure*}[t]
    \centering
    \includegraphics[width=\textwidth]{fig1_galaxy_diagram.png}
    \caption{A schematic illustrating how we model the Milky Way. The left panel illustrates the different model aspects: star formation history of three galactic components (individually shown in the dotted lines), radial distribution, metallicity-radius-time relation, and height distribution. The right panel shows an example instance of the Milky Way with $250000$ binaries shown as points coloured by metallicity. The top panel shows a side-on view and the bottom panel a face-on view. \href{https://github.com/TomWagg/detecting-DCOs-in-LISA/blob/main/paper/figures/fig1_galaxy_diagram.png}{\faFileImage} \href{https://github.com/TomWagg/detecting-DCOs-in-LISA/blob/main/paper/figure_notebooks/galaxy_creation_station.ipynb}{\faBook}.}
    \label{fig:galaxy_schematic}
\end{figure*}

Fig.~\ref{fig:galaxy_schematic} shows the distributions and relations outlined in this section and also displays an example random galaxy drawn using this model.

Our model for the Milky Way accounts for the low-\achem\footnote{Nomenclature used to describe the enhancement of $\alpha$ elements compared to iron in stellar atmospheres}~disc, high-\achem~disc and a central component approximating a bulge/bar. The low- and high-\achem~discs are often also referred to as the thin and thick discs because the stellar vertical distribution is better fit by a double exponential rather than a single one. However, this doesn't allow one to assign a star to either the thin or thick disk purely based on its height above the Galactic plane. Therefore, we instead use the chemical definition of the two disks (applying the \achem~nomenclature) as there is a clear bimodal distribution in the chemical plane, allowing stars to be assigned to each of the disc components based on their chemical abundances. For each of the three components, we use a separate star formation history and spatial distribution, which we combine into a single model, weighting each component by its present-day stellar mass. \citet{Licquia+2015} gives that the stellar mass of the bulge is $0.9 \times 10^{10} \unit{M_{\odot}}$ and the stellar mass of the disc is $5.2 \times 10^{10} \unit{M_\odot}$, which we split equally between the low- and high-\achem~discs \citep[e.g.,][]{Snaith+2014}.

\textit{Star formation history:} 
We use an exponentially declining star formation history \citep{Frankel+2018} (inspired by the average cosmic star formation history) for the combined low- and high-\achem~discs,
\begin{equation}\label{eq:thin_disc_tau}
    p(\tau) \propto \exp \qty(-\frac{(\tau_m - \tau)}{\tau_{\rm SFR}}),
\end{equation}
where $\tau$ is the lookback time (the amount of time elapsed between the binary's zero-age main sequence and today), $\tau_m = 12 \unit{Gyr}$ is the assumed age of the Milky Way and $\tau_{\rm SFR} = 6.8 \unit{Gyr}$ is the star formation timescale \citep{Frankel+2018}. The two discs form stars in mutually exclusive time periods, such that the high-\achem~disc forms stars in the early history of the galaxy ($8$--$12 \unit{Gyr}$ ago) and the low-\achem~disc forms stars more recently ($0$--$8 \unit{Gyr}$ ago). Both distributions are normalised so that an equal amount of mass is formed in each of the two components over their respective star forming periods.

The star formation history of the bulge/bar of the Milky Way has many uncertainties due to the (1) sizeable age measurement uncertainties at large ages in observational studies, (2) complex selection processes affecting the observed age distributions, and (3) formation mechanisms that are still under debate. However, the central bar (which we assume to dominate here) was shown to contain stars with an extended age range, with most observed stars between $6$ and $12 \unit{Gyr}$ with a younger tail of ages that could come from the subsequent secular growth of the Galactic bar \citep[e.g.,][]{Bovy+2019}. To model the bulge/bar's age distribution more realistically than in previous studies (which assume an old bulge coming from a single starburst), we choose to adopt a more extended star formation history using a $\beta(2,3)$ distribution, shifted and scaled such that stars are only formed in the range $[6, 12] \unit{Gyr}$. We show these distributions in the top left panel of Fig.~\ref{fig:galaxy_schematic}.

\textit{Radial distribution:} For each of the three components we employ the same single exponential distribution (but with different scale lengths)
\begin{equation}\label{eq:galaxy_R}
    p(R) = \exp(-\frac{R}{R_d}) \frac{R}{R_d^2},
\end{equation}
where $R$ is the Galactocentric radius and $R_d$ is the scale length of the component. For the low-\achem~disc, we set $R_d = R_{\rm exp}(\tau)$, where $R_{\rm exp}(\tau)$ is the scale length presented in \citet[][Eq.~6]{Frankel+2018}
\begin{equation}
    R_{\rm exp}(\tau) = 4 \unit{kpc} \qty(1 - \alpha_{R_{\rm exp}} \qty(\frac{\tau}{8 \unit{Gyr}})),
\end{equation}
where $\alpha_{R_{\rm exp}} = 0.3$ is the inside-out growth parameter\footnote{We find that $R_{\rm exp}(\tau) = 4$ kpc fits the data well and adopt this value rather than the 3 kpc quoted in \cite{Frankel+2018}, which was a fixed parameter (not a fit).}

This scale length accounts for the inside-out growth of the low-\achem~disc and hence is age dependent. We assume $R_d = (1 / 0.43) \unit{kpc}$ for the high-\achem~disc \citep[][Table~1]{Bovy+2016} and $R_d = 1.5 \unit{kpc}$ for the bulge/bar component \citep{Bovy+2019}. \edit1{Note that in this way we have approximated the bulge/bar component as being axi-symmetric, which is sufficient for our purposes.} We show the combination of these distributions in the second panel on the left in Fig.~\ref{fig:galaxy_schematic}.

\textit{Vertical distribution}: Similar to the radial distribution, we use the same single exponential distribution (but with different scale heights) for each component, given by
\begin{equation}\label{eq:galaxy_z}
    p(\abs{z}) = \frac{1}{z_d} \exp\qty(-\frac{z}{z_d}),
\end{equation}
where $z$ is the vertical displacement above the Galactic plane and $z_d$ is the scale height. We set $z_d = 0.3 \unit{kpc}$ for the low-\achem~disc \citep{McMillan+2011} and $z_d = 0.95 \unit{kpc}$ for the high-\achem~disc \citep{Bovy+2016}. For the bulge/bar, we set $z_d = 0.2 \unit{kpc}$ \citep{Wegg+15}. We show the combination of these distributions in the bottom left panel of Fig.~\ref{fig:galaxy_schematic}.

\textit{Metallicity-radius-time relation:} To account for the chemical enrichment of star forming gas as the Milky Way evolves, we adopt the relation given by \citep[][Eq. 7]{Frankel+2018}
\begin{equation}\label{eq:galaxy_FeH}
    \begin{split}
        [{\rm Fe} / {\rm H}] (R, \tau) &= F_m + \nabla [{\rm Fe} / {\rm H}] R \\
        &- \qty(F_m + \nabla [{\rm Fe} / {\rm H}] R^{\rm now}_{[{\rm Fe} / {\rm H}] = 0} ) f(\tau),
    \end{split}
\end{equation}
where
\begin{equation}
    f(\tau) = \qty(1 - \frac{\tau}{\tau_m})^{\gamma_{[{\rm Fe} / {\rm H}]}},
\end{equation}
$F_m = -1 \unit{dex}$ is the metallicity of the gas at the center of the disc at $\tau = \tau_m$, $\nabla [{\rm Fe} / {\rm H}] = -0.075 \unit{kpc^{-1}}$ is the metallicity gradient, $R^{\rm now}_{[{\rm Fe} / {\rm H}] = 0} = 8.7 \unit{kpc}$ is the radius at which the present day metallicity is solar and $\gamma_{[{\rm Fe} / {\rm H}]} = 0.3$ sets the time dependence of the chemical enrichment. We convert this to the representation of metallicity that we use in this paper by applying \citep[e.g.][]{Bertelli+1994}
\begin{equation}\label{eq:galaxy_FeH_to_Z}
    \log_{10} (Z) = 0.977 [{\rm Fe} / {\rm H}] + \log_{10}(Z_\odot).
\end{equation}

Although \citet{Frankel+2018} only fit this model for the low-\achem~disc, we also use this metallicity-radius-time relation for the high-$\alpha$ disc and the bar, but focusing on the chemical tracks more representative to the inner disc and large ages. \citet{Sharma+2020} showed that using a simple continuous model for both the low- and high-\achem~discs, the Milky Way abundance distributions could be well reproduced. Empirically, the abundance tracks in the [$\alpha$/Fe]-[Fe/H] plane (and other elements) of the stars in the bulge/bar follow the same track as those of the old stars in the Solar neighbourhood \citep[][Fig.~7,]{Griffith+2021,Bovy+2019}, which motivates our modelling choice to use the same metallicity-radius-time relation.

\subsubsection{Combining population and galaxy synthesis}\label{sec:combining_pop_gal}

For each Milky Way instance, we randomly sample the following set of parameters
\begin{equation}
    \mathbf{g}_{{j}} = \{\tau, R, Z, z, \theta\}
\end{equation}
for $j = 1, 2, \dots, N_{\rm MW}$, where we set $N_{\rm MW} = 2 \times 10^{5}$, $\tau, R, Z$ and $z$ are defined and sampled using the distribution functions specified in Section~\ref{sec:mw_model}, $\theta$ is the azimuthal angle sampled uniformly on $[0, 2\pi)$ and $Z$ is the metallicity. Fig.~\ref{fig:galaxy_schematic} shows an example of a random Milky Way instance created with these distributions. This shows how these distributions translate to positions and illustrates the gradient in metallicity over radius.

We match each set of galaxy parameters $\mathbf{g}_{{j}}$, to a random set of binary parameters $\mathbf{b}_{{Z, i}}$, by drawing a binary from the closest metallicity bin to the metallicity in $\mathbf{g}_{{j}}$. If the metallicity in $\mathbf{g}_{{j}}$ is below the minimum COMPAS metallicity bin ($Z = 10^{-4}$), we use this minimum bin. If the metallicity in $\mathbf{g}_{{j}}$ is above the maximum COMPAS metallicity bin ($Z = 0.03$), we use a randomly selected bin from the five highest metallicity bins\footnote{\edit1{We spread the binaries over the five highest bins, rather than just the highest bin, as we found that using a single bin led to unphysical artifacts in our results. These artifacts arose because the small population of binaries in the highest bin were oversampled.}}.

Each binary is likely to move from its birth orbit. Although all stars in the Galactic disc experience radial migration \citep{Sellwood+2002, Frankel+2018}, DCOs generally experience stronger dynamical evolution as a result of the effects of both Blaauw kicks \citep{Blaauw+1961} and natal kicks \citep[e.g.][]{Hobbs+2005}.

The magnitude of the systemic kicks are typically small compared to the initial circular velocity of a binary at each Galactocentric radius. Therefore, we expect that kicks will not significantly alter the overall distribution of their positions (see however, e.g., \citealt{Brandt+1995, Abbott+2017_GW170817_progenitor}). Given this, and for the sake of computational efficiency, we do not account for the displacement due to systemic kicks in our analysis.

\subsection{Gravitational wave detection}\label{sec:gw_detection}
We use the Python package \href{https://legwork.readthedocs.io/en/latest/}{LEGWORK} \citep{Wagg+2021} to evolve binaries and calculate their LISA detectability. For a full derivation of the equations given below see \citep[][Section~3]{Wagg+2021}, or the LEGWORK documentation \href{https://legwork.readthedocs.io/en/latest/notebooks/Derivations.html}{\faBook}.

\subsubsection{Inspiral evolution}

Each binary loses orbital energy to gravitational waves throughout its lifetime. This causes the binary to shrink and circularise over time. In order to assess the detectability of a binary, we need to know its eccentricity and frequency at the time of the LISA mission. For each binary in our simulated Milky Way, we know that the time from DCO formation to today is $\tau - t_{\rm evolve}$ and that the initial eccentricity and semi-major axis are $e_{\rm DCO}$ and $a_{\rm DCO}$. We find the eccentricity of the binary at the start of the LISA mission, $e_{\rm LISA}$, by numerically integrating its time derivative \citep[][Eq. 5.13]{Peters+1964} given the initial conditions. This can be converted to the semi-major axis at the start of LISA, $a_{\rm LISA} $\citep[][Eq. 5.11]{Peters+1964}, which in turn gives the orbital frequency, $f_{\rm orb, LISA}$, by Kepler's third law since we know the component masses.

\subsubsection{Binary detectability}

We define a binary as detectable if its gravitational wave signal has a signal-to-noise ratio (SNR) of greater than 7 \edit1{by the end of the LISA mission} \citep[e.g.][]{Breivik+2020, Korol+2020}. The sky-, polarisation- and orientation-averaged signal-to-noise ratio, $\rho$, of an inspiraling binary can be calculated with the following \citep[e.g.][]{Finn+2000}
\begin{equation}\label{eq:snr}
    \rho^2 = \sum_{n=1}^{\infty} \int_{f_{n, i}}^{f_{n, f}} \frac{h_{c, n}^{2}}{f_{n}^{2} S_{\rm n}\left(f_{n}\right)} \dd{f_n},
\end{equation}
where $n$ is a harmonic of the gravitational wave signal, $f_n = n \cdot f_{\rm orb}$ is the frequency of the $n^{\rm th}$ harmonic of the gravitational wave signal, $f_{\rm orb}$ is the orbital frequency, $S_{\rm n}(f_n)$ is the LISA sensitivity curve at frequency $f_n$ \citep[e.g.][]{Robson+2019} and $h_{c,n}$ is the characteristic strain of the $n^{\rm th}$ harmonic, given by \citep[e.g.][]{Barack+2004}
\begin{equation}\label{eq:charstrain}
    h^2_{c,n} = \frac{2^{5/3}}{3 \pi^{4/3}} \frac{(G \mathcal{M}_c)^{5/3}}{c^3 D_L^2} \frac{1}{f_{\rm orb}^{1/3}} \frac{g(n,e)}{n F(e)},
\end{equation}
where $D_L$ is the luminosity distance to the source, $f_{\rm orb}$ is the orbital frequency, $g(n, e)$ and $F(e)$ are given in \citet{Peters+1963} and $\mathcal{M}_c$ is the chirp mass, defined as
\begin{equation}\label{eq:chirp_mass}
    \mathcal{M}_c = \frac{(m_1 m_2)^{3/5}}{(m_1 + m_2)^{1/5}}.
\end{equation}

Note that increasing the length of the LISA mission allows more time for a DCO to evolve over the mission. Therefore the frequency limits in Eq.~\ref{eq:snr} are dictated by the LISA mission length. The SNR generally scales as $\sqrt{T_{\rm obs}}$ (with exceptions for sources very close to merging) and thus the SNR of a typical source in a 10-year LISA mission is approximately $1.58$ (=$\sqrt{10/4}$) times stronger than in a 4-year mission.

We use \href{https://legwork.readthedocs.io/en/latest/}{LEGWORK} \citep{Wagg+2021} to calculate the signal-to-noise ratio for each binary and the package ensures that enough harmonics are computed for each binary such that the error on the gravitational-wave luminosity remains below 1\%.

\subsubsection{Detection rate calculation}
For each physics variation model and DCO type, we first convert the COMPAS simulation results into a total number of DCOs in the Milky Way, $N_{\rm DCO}$. We do this by integrating the full mass and period distributions and stars and normalising to the total Milky Way mass. For more details see Appendix~\ref{app:rate_normalisation}.

We then determine the fraction of binaries that are detectable in each Milky Way instance by summing the adaptive importance sampling weights of the binaries that have an SNR greater than 7, and dividing by the total weights in the simulation. We multiply this fraction by $N_{\rm DCO}$ to find a detection rate (which we write as a total number of detections per LISA mission)
\begin{equation}
    N_{\rm detect} = \frac{\sum_{i = 0}^{N_{\rm MW}} w_i \phi(i)}{\sum_{i = 0}^{N_{\rm MW}} w_i} N_{\rm DCO},
\end{equation}
where $\phi(i) = 1$ if a binary is detectable and $0$ otherwise. We calculate the detection rate by Monte Carlo sampling 2500 Milky Way instances (each containing 200,000 DCOs) for each DCO type and every physics variation in order to obtain values for the uncertainty on the expected detection rate.