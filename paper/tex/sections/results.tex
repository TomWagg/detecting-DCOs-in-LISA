In this section we present our main results for the detectable LISA DCO population. We find that on average, for our fiducial model, a four year LISA mission will detect \confinv{34.4}{20.9}{46.6} BHBHs, \confinv{29.9}{20.1}{47.4} BHNSs and \confinv{7.7}{5.0}{10.3} NSNSs where the error bars represent the 90\% confidence interval. We first show the distribution of the sources on the sensitivity curve in Section~\ref{sec:dcos_on_sc}, before exploring the variations in the detection rate over different physics variations in Section~\ref{sec:detection_rate_analysis} and analysing the parameter distributions for detectable sources in the fiducial model in Section~\ref{sec:fiducial_distributions}.

\subsection{Distribution on the sensitivity curve}\label{sec:dcos_on_sc}

We illustrate the distribution of detectable DCOs on the sensitivity curve in Figure~\ref{fig:dcos_on_sc}. This shows that the detectable population of these massive DCOs is concentrated at comparatively lower frequencies than the LISA verification binaries (shown as stars in the top panel) and the more numerous WDWD population \citep[e.g.][]{Korol+2017}. This is expected since producing the same SNR as a BHBH, BHNS or NSNS with a relatively lower mass (circular) WDWD requires a higher frequency. This finding is in agreement with \citet{Sesana+2020} and as noted in that work, this could possibly be used to distinguish these more massive DCOs from WDWDs. There are several other notable features in these distributions and we plot grid lines of constant distance and inspiral time to help to explain the shape of the distribution.

The straight diagonal lines show where a binary at a fixed distance and with the average chirp mass (annotated in each panel) would lie on the sensitivity curve for different frequencies. We would therefore expect that if a population was entirely circular, it should be bounded approximately between the $0.1$--$30 \unit{kpc}$ lines (roughly the minimum and maximum distance to a source in the Milky Way). From inspection of the bottom panels with each individual DCO type we see that, though this is the true for a large fraction of the population, there is a distinct subpopulation of binaries that extend downwards around $2 \times 10^{-3} \unit{Hz}$. This offshoot is composed of eccentric binaries for which the circular distance contours do not apply. For instance, we plot the 90\% contour of only the circular sources in our sample over the density distribution in each of the bottom panels and it is clear that the circular subsample is bounded by the distance lines. We also plot a line of constant distance at $30 \unit{kpc}$ for an eccentric binary with $e = 0.97$ to show the differences for eccentric sources. We also note that the peak of the density distribution coincides with the centre of the Milky Way as expected, since binaries are most likely to be formed towards the centre of the Galaxy.

We also plot vertical lines that give the inspiral time for a circular binary with the average chirp mass (annotated in each panel). From these lines, it is clear why the density distribution decreases with increasing frequency, since high frequencies correspond to short inspiral times and thus DCOs will spend less time in these regimes. We also see that the tail of the high frequency sources is more numerous near to the Galactic centre than at short distances. This is simply because there are more sources in the galactic centre and so the chances of `catching' a binary at high frequency are better.

\begin{figure*}[tp]
    \centering
    \includegraphics[width=\textwidth]{dcos_on_sc.png}
    \caption{Density distribution of detectable DCOs plotted over the LISA sensitivity curve, where the panels correspond to \textbf{top:} combined density plot for all DCO types (BHBH, BHNS and NSNS) \textbf{bottom:} three panels with individual density distribution of different DCO types. In each panel, we plot the total signal of binaries at their dominant frequency $n f_{\rm orb}$, such that $n$ is the harmonic that produces the most relative gravitational wave luminosity ($n = 2$ for circular binaries). If the density of points is below our lowest contour (2\%) then we plot the points as scatter points, where their sizes corresponds to their STROOPWAFEL weights. The inset colourbars indicate the percentage of the population represented by each contour and the annotated mass is the average chirp mass for all binaries in the panel, which is used in plotting the grid lines. We plot diagonal lines of constant distance, where the straight lines show the signal for a circular binary of average chirp mass, whilst the curved line shows the signal for an eccentric binary with $e = 0.97$. The vertical lines indicate the inspiral time for a circular binary with the average chirp mass. In the top panel we overlay the LISA verification binaries from \citet{Kupfer+2018}. In the bottom panels we add the 90\% contour line for only the circular sources to show how the distribution changes without eccentric sources present.}
    \label{fig:dcos_on_sc}
\end{figure*}

\subsection{Detection rates}\label{sec:detection_rate_analysis}
We find that for our fiducial model, a four year LISA mission will detect \confinv{34.4}{20.9}{46.6} BHBHs, \confinv{29.9}{20.1}{47.4} BHNSs and \confinv{7.7}{5.0}{10.3} NSNSs, where the error bars represent the 90\% confidence interval. Increasing the LISA mission length to ten years changes the number of detections to \confinv{42.0}{17.3}{17.3}, \confinv{44.5}{17.8}{20.7} and \confinv{19.3}{8.0}{9.7} respectively. In Figure~\ref{fig:detection_rates}, we show the expected number of LISA detections for each model variation and discuss the prominent trends in the following sections. We show the rates and uncertainties plotted in this figure in Table~\ref{tab:detection_rates}.

\begin{figure*}[p]
    \centering
    \includegraphics[width=\textwidth]{dco_detections.pdf}
    \caption{The number of expected detections in the LISA mission for different DCO types and model variations. Error bars show the 50\% (solid) and 90\% (dotted) confidence intervals. The left axis and grid lines show the number of detections in a four year LISA mission and the right axis shows an approximation of the number of detections in a 10 year mission (we scale the axis by $\sqrt{T_{\rm obs}}$, see Table~\ref{tab:detection_rates} for exact rates). Each model is described in further detail in Table~\ref{tab:physics_variations} and details of the fiducial assumptions are in Section~\ref{sec:fiducial_physics}.}
    \label{fig:detection_rates}
\end{figure*}

\subsubsection{BHBH detection rate trends}
The BHBH detection rate is markedly robust across physics variations, with the expected detections in each model staying within 25\% of the fiducial rate (with the exception of model \modOpt{}). Thus even if there are changes in our understanding of the underlying physics before the LISA mission commences, the expected BHBH detection rate is unlikely to change significantly.

The exception to this statement is model \modOpt{}, in which we allow Hertzsprung gap donors to survive common envelope events. A large fraction of the progenitors of BHs in this mass range expand significantly during the Hertzsprung gap phase and initiate common envelope events. Therefore, though the detectable fraction does not change significantly, the increased population of BHBHs in the Milky Way leads to this model predicting 2.5 times more detections.

\subsubsection{BHNS detection rate trends}
In contrast, the BHNS detection rate is very sensitive to changes in binary physics assumptions. Therefore, once LISA flies and we know the actual number of detections, we can compare to each model and possibly provide some constraint on binary evolution physics. There are several notable trends in the BHNS detection rate in the middle pane of Figure~\ref{fig:detection_rates}.

As $\beta$ increases in models \modBetaLow{}-\modBetaHigh{}, the BHNS detection rate steadily decreases. This may seem unintuitive since a higher mass transfer efficiency should lead to more massive compact objects and thus a more detectable population. However, one must also consider that most of these DCOs are formed through a common envelope event and so retaining more of the envelope during mass transfer means that the eventual ejection of the envelope is much more difficult, thus leading to more stellar mergers and fewer detectable BHNSs \citep[e.g.][]{Kruckow+2018}.

\tom{@Selma, the trend with common envelopes still confuses me, specifially, why does it not increase when $\alpha=2.0$? We never quite resolved this in the thread in zpro\_tom\_wagg with me and Lieke. I do see that the BHBH have a lot of only stable mass transfer and so reasonably are not too affected. NSNS basically only come through CE events and so sensibly are strongly affected but BHNS have $\sim 70\%$ classic channel and so should be affected strongly. But we don't see an increase with $\alpha = 2.0$. Any thoughts?}
% For a similar reason, the rate is decreased when $\alpha$ is decreased in model \modAlphaLow{}, as this reduces the amount of orbital energy that is used to eject the envelope and thus leads to more stellar mergers. \todo{Why isn't the opposite true for model \modAlphaHigh{}? It seems the number of bound DCOs \textit{does} increase but the merging total decreases...?}

Enforcing that case BB mass transfer is always unstable (model \modCaseBB{}) decreases the detection rate as fewer NSs are produced and thus fewer BHNSs form. This is explained in further detail in Section~\ref{sec:NSNS_detection_trends}. For the same reason as the BHBH rate, model \modOpt{} has a higher number of detections. This change is less prominent than in the BHBH case as the progenitors tend to be lower masses and initiate a CE event less frequently during the Hertzsprung gap phase. 

The Fryer \textit{rapid} prescription (model \modRapid{}) leads to a higher detection rate for BHNSs because progenitors that would become black holes in the \textit{delayed} prescription, instead become neutron stars and so more BHNSs are formed instead of BHBHs. For the same reason, increasing the maximum neutron star mass (model \modNSHigh{}) increases the detection rate and the inverse is true when it is decreased (model \modNSLow{}).

Finally, models \modSigLow{}-\modNoBH{} show increased detection rates since lower kicks result in fewer disrupted binaries and hence a more numerous detectable population. Following this logic it makes sense that model \modSigLower{} produces more detections than model \modSigLow{}. The model with no BH kick (\modNoBH{}) is slightly lower than model \modSigLower{} as the number of surviving binaries is limited by the neutron star kick more than the black hole kick.

\subsubsection{NSNS detection rate trends}\label{sec:NSNS_detection_trends}

As $\beta$ increases the NSNS detection rate increases, the opposite trend to that seen in the BHNS rate. This is for two main reasons: firstly the ejection of a common envelope is less problematic for the less massive NSNS binaries. Moreover, the increased mass transfer efficiency means that systems that were previously below the mass necessary to become a NS can now accrete enough mass to form a NS. Although the same is true for more massive stars becoming BHs instead of NSs, due to the IMF, there is a net flux of more stars becoming NSs.

There is a drastic decrease in detections for model \modCaseBB{} by nearly two orders of magnitude. This is because the majority of NSNS binaries are formed through case BB mass transfer and setting this mass transfer to be always unstable results in many of these binaries to merge before they could become NSNSs. As a result the total number of detections decreases, however, interestingly the remaining population represent more massive progenitors (that would not go through case BB mass transfer) and thus is skewed to higher masses and has a \textit{higher} detectable fraction.

The vast majority of NSNSs in our sample are formed through the common envelope channel and thus changing the value of $\alpha_{\rm CE}$ has an effect on the rate. We see that decreasing $\alpha_{\rm CE}$ (model \modAlphaLow) leads to a lower rate as there is less energy available to eject the envelope and so more binaries result to stellar mergers rather than NSNSs and similarly we see an inverse trend when increasing $\alpha_{\rm CE}$ (model \modAlphaHigh).

As we found in the BHNS trends, a lower value for the core-collapse supernova velocity dispersion increases the detection rate in models \modSigLow{} and \modSigLower{}, whilst changing the PISN or BH kick prescription (models \modNoPISN{} and \modNoBH{}) of course has no effect on the NSNS population.

\subsection{Distributions for the fiducial model}\label{sec:fiducial_distributions}

In Figure~\ref{fig:fiducial_pdf_distributions}, we show the distribution of the individual parameters of the population of detectable binaries and discuss the various features in the following sections.

\subsubsection{Black Hole Mass}
For both the BHBHs and BHNSs, the black hole mass distribution extends across relatively low masses, with $83\%$ and $90\%$ respectively below $10 \unit{M_{\odot}}$. This is because, at the high metallicities in the Milky Way, stellar winds are much stronger and strip away much of the stellar mass before BH formation. The mass distribution also extends down to $2.5 \unit{M_{\odot}}$, our fiducial maximum neutron star mass, since the \citet{Fryer+2012} \textit{delayed} remnant mass prescription does not produce a mass gap between neutron stars and black holes. Thus BHBHs and BHNSs detected by LISA could be ideal for ascertaining whether there exists a lower mass gap between neutron stars and black holes.

The bimodality of the BHBH distribution is a result of most detectable BHBHs in our sample having unequal mass ratios. The two peaks are from the primary and secondary black hole masses and we show their individual distributions with the dotted curves.

The reasoning for these unequal mass ratio is as follows: in order to produce a BHBH, most formation channels require at least the first mass transfer to be stable. This stability is strongly dependent on the mass ratio such that equal mass ratios (at the moment of mass transfer) are preferred for creating BHBHs. Yet, since stellar winds are so strong at high metallicity, and even stronger for more massive stars, the primary star will experience significant mass loss and so an initially \textit{unequal} mass ratio is preferred so that the masses are more balanced at the first instance of mass transfer. Since mass transfer occurs after the end of the main sequence for most of our BHBHs, the star will have a well defined core and these core masses, which go on to form BHs, will reflect the initially unequal mass ratios.

\subsubsection{Neutron Star Mass}
The neutron star mass distribution shows that most neutron stars have low masses, with $69\%$ and $87\%$ having masses below $1.5 \unit{M_{\odot}}$ for BHNSs and NSNSs respectively. The lacks of neutron stars around $1.7 \unit{M_{\odot}}$ and the subsequent small peaks are artifacts of the discontinuous nature of the \citet{Fryer+2012} remnant mass prescription.

\subsubsection{Eccentricity}
The eccentricity distributions show that detectable BHBHs are the most eccentric of the three DCOs. This may seem counter-intuitive since neutron stars receive stronger natal kicks, which cause the orbit to become eccentric. However, these stronger kicks often instead result in disrupted or too-wide binaries. In contrast, BHBHs can receive strong kicks that impart high eccentricity without disrupting and thus tend to be more eccentric. This effect is compounded by the fact that we can see BHBHs at lower orbital frequencies, meaning that they have not had as much time to circularise and so still have significant eccentricity by the time of the LISA mission.

\subsubsection{Orbital Frequency and Frequency Evolution}
The orbital frequency distributions for BHBHs, BHNSs and NSNSs peak at increasing frequencies. This is because a higher mass DCO at the same distance and eccentricity requires a lower frequency to produce the same signal-to-noise ratio and thus be detected. The BHBH distribution also has a tail that extends to $4 \times 10^{-6} \unit{Hz}$, which is comprised of highly eccentric binaries since eccentricity moves the dominant harmonic to higher frequencies. Similar tails are not as prevalent for BHNSs and NSNSs as they do not have as many eccentric binaries.

\subsubsection{Luminosity Distance}
Each DCO's luminosity distance distribution peaks around $8 \unit{kpc}$ since this is the distance to the centre of the Milky Way and thus the most dense location of DCOs. Each DCO also has a shoulder at lower distances since closer binaries are easier to detect. This shoulder is more prominent for the NSNS distribution since their lower relative masses require a smaller distance in order to be detected on average.

\subsubsection{Inspiral Time}
Each DCO has a strong peak at small inspiral times since higher metallicities lead to tighter binaries and thus shorter inspiral times. \todo{finish this}

\begin{figure*}[htbp]
    \centering
    \includegraphics[width=\textwidth]{distribution_grid_4yr.pdf}
    \caption{Distributions for various parameters of the DCOs that are detectable in a 4 year LISA mission in our fiducial model. Each panel shows the distribution of a single parameter, where the colour denotes the DCO type. We also plot the 1- and 2-$\sigma$ uncertainties (obtained via bootstrapping) with the dark and light shaded areas respectively. The first two rows (excluding metallicity) use kernel density estimators to show the distributions. The metallicity panel shows the distribution over the metallicity bins used in our population synthesis, which we show in the grid lines. The final three panels for the cumulative distribution functions for observables, normalised to the expected number of fiducial detections in a four year LISA mission. The dark shaded areas indicate regimes in which the quantity cannot be measured. The dotted lines in the angular resolution plot show the maximum angular resolution that can be covered by a single pointing of the labelled instrument. In Section~\ref{sec:fiducial_distributions} we discuss the features of the distributions.}
    \label{fig:fiducial_pdf_distributions}
\end{figure*}

\subsection{Model variations}