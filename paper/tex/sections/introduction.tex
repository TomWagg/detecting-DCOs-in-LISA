Since the first direct observation of gravitational waves \citep{Abbott+2016_first_detection}, the number of black hole (BH) and neutron star (NS) binaries observed by ground-based gravitational-wave detectors has rapidly grown \citep{Abbott+2019_GWTC1,Abbott+2020_GWTC2}, offering exciting insights into the formation, lives and deaths of massive binary stars \citep[e.g.][]{Abbott+2021_GWTC2_inference}.


The Laser Interferometer Space Antenna (LISA, \citealp{Amaro-Seoane+2017}) will provide observations in an entirely new frequency regime of gravitational waves. LISA will observe at lower frequencies than ground-based detectors ($10^{-5} \lesssim f / \unit{Hz} \lesssim 10^{-1}$) and so will enable the study of sources that are undetectable with ground-based detectors such as the mergers of supermassive black holes and extreme mass ratio inspirals \citep[e.g.][]{Begelman+1980, Klein+2016}. Moreover, this frequency regime is also of interest for the detection of local stellar mass binaries during their inspiral phase. LISA will be able to detect, and possibly localise a double compact object (DCO) on the sky many years in advance of the merger, which presents an opportunity for both multimessenger detections to search for electromagnetic counterparts and multiband detections that can help to constrain binary characteristics \citep[e.g.][]{Sesana+2016, Gerosa+2019}. In addition, LISA will be able to measure the eccentricities of DCOs that may yield further constraints on binary evolution, differentiate between formation channels and distinguish between DCO types \citep[e.g.][]{Nelemans+2001, Breivik+2016, Antonini+2017, Rodriguez+2018}. The maximum distance at which stellar mass sources in LISA are detectable is significantly lower than in ground-based detectors since the gravitational wave signal is weaker during the inspiral phase than at the merger. This means that LISA stellar mass sources can only be detected in local galaxies, with the majority residing in the Milky Way. Therefore, these sources could be used as a probe for the structure of our galaxy \citep[e.g.][]{Korol+2019}.

Traditionally, investigations into detecting stellar mass sources with LISA focus on double white dwarf (WDWD) binaries, as they are abundantly present in our galaxy and are expected to be the dominant source of stellar-mass binaries that are detectable by LISA \citep{Nelemans+2001,Ruiter+2010,Yu+2010,Nissanke+2012,Korol+2017,Lamberts+2018}. More recently, interest has grown in the detection of NS and BH binaries. Although they are more rare, LISA detections of these sources are potentially valuable for learning more about the evolution and endpoints of massive stars. In this paper we focus on NS and BH binaries.

 

Galactic double neutron star (NSNS) binaries have been observed with electromagnetic signals for several decades \citep[e.g.][]{Hulse+1975, Tauris+2017,Vigna-Gomez+2018} and more recently the mergers of NSNS binaries with ground-based gravitational wave detectors have been detected \citep[e.g.][]{Abbott+2017_NSNS}. The detection of a NSNS in LISA with a pulsar component could potentially connect these two populations if the binary is close to merging, as the binary could be observed from inspiral to merger. NSNS binaries are useful sources for understanding the origin of r-process elements \citep[e.g.][]{Eichler+1989} as well as the electromagnetic counterparts to gravitational wave signal such as kilonovae \citep[e.g.][]{Metzger+2017}, short gamma-ray bursts \citep[e.g.][]{Gompertz+2020}, radio emission \citep[e.g.][]{Hotokezaka+2016} and neutrinos \citep[e.g.][]{Kyutoku+2018}.

Double black hole (BHBH) binaries in the Milky Way present a greater observational challenge. To date, no BH has been observed to be in a binary with another compact object in the Milky Way and so LISA could provide the first detection of a Galactic BHBH. The only confirmed BHs in our galaxy have been discovered as components of X-ray binaries with companion stars \citep[e.g.][]{Bolton+1972,Webster+1972}. This sample of BHs has masses mainly constrained between $5$ and $10 \unit{M_\odot}$ \citep{Corral-Santana+2016}, a stark contrast to the more massive BHs observed with LIGO/Virgo that tend to have masses concentrated around $30 \unit{M_{\odot}}$ \citep{Abbott+2020_GWTC2}. These observations of X-ray binaries suggest the presence of a lower mass gap (from $2$-$5 \unit{M_{\odot}}$) in which there are no strong candidates for either black holes or neutron stars \citep{Ozel+2010,Farr+2011} but the gap's existence remains an open question \citep[e.g.][]{Kreidberg+2012, Mandel+2020}. Recently there has also been increased discussion over the maximum BH mass in our galaxy, with the claims of a $70 \unit{M_{\odot}}$ BH \citep{Liu+2019} which has subsequently been challenged \citep{El-Badry+2020, Abdul-Masih+2020, Shenar+2020,Eldridge+2020} and revised measurements of the mass of Cygnus X-1 \citep{Miller-Jones+2021}. A sample of BHBHs detected with LISA could possibly help to constrain the stellar mass BH mass distribution.

One particularly interesting and elusive gravitational wave source is a black hole neutron star binary (BHNS). Of all the events detected by ground-based detectors, none can be confidently attributed to the merger of a black hole and a neutron star, though several events such as GW190425 and GW190814 have not been ruled out as a BHNS merger \citep{Abbott+2020_GW190425,Abbott+2020_GW190814}. Predictions for the merger rate of BHNSs range across three orders of magnitude \citep[e.g.][]{Abadie+2010, Broekgaarden+2021} so the number of detections in LISA will be important in reducing this uncertainty, thereby refining our understanding of the remnants and evolution of massive stars. These binaries are expected to have electromagnetic counterparts that can studied in the same way as NSNSs. A distinctly exciting possibility is the detection of a pulsar--BH system or millisecond pulsar--BH system \citep{Narayan+1991}. These systems could be observed not only by gravitational wave detectors, but also radio telescopes such as MeerKAT and SKA, which will help to improve the measurement of individual system parameters and to constrain uncertain binary evolution processes \citep[e.g.][]{Pfahl+2005,Chattopadhyay+2020}.

For the purposes of this investigation, we consider the `classical' isolated binary evolution channel \citep[e.g.][]{Tutukov+1973,Tutukov+1993,Smarr+1976,Srinivasan+1989,Kalogera+2007,Belczynski+2016} in which compact objects are formed through through common envelope ejection or a phase of highly non-conservative mass transfer \citep{Heuvel+2011, vandenHeuvel+2017}. We do not, however, account for several alternative proposed formation channels, which could affect the rate and distribution of detectable NS and BH binaries in LISA. These channels include: dynamical formation in dense star clusters \citep[e.g.][]{Sigurdsson+1993,PortegiesZwart+2000,Miller+2009,Rodriguez+2015} and (active) galactic nuclei discs \citep[e.g.][]{Morris+1993, Antonini+2016, McKernan+2020}, isolated hierarchical triple evolution involving Kozai-Lidov oscillations \citep[e.g.][]{Stephan+2016, Silsbee+2017,Antonini+2017, Toonen+2020} and chemically homogenous evolution through efficient rotational mixing \citep[e.g.][]{deMink+2009, deMink+2016,Marchant+2016,duBuisson+2020}.

In this paper, we present predictions of the detection rate and distribution of binary properties (masses, frequency, eccentricity, distance, merger time) of BHBH, BHNS and NSNS binaries formed through isolated binary evolution in the Milky Way. We explore the effect of varying physical assumptions in our population synthesis model on our results as well as discuss the effect of extending the LISA mission length and the possibility of distinguishing detections.

Earlier work on BHBHs, BHNSs and NSNSs in LISA has used a variety of population synthesis codes, Milky Way models and LISA specifications, resulting in a wide range of predictions \citep{Nelemans+2001,Liu+2009,Belczynski+2010,Liu+2014,Lamberts+2019,Lau+2020,Breivik+2020,Sesana+2020}. We build upon previous efforts but with several important improvements. We explore the effect of varying binary physics assumptions by repeating our analysis for \nModels{} different models and comparing the effect on the detection rate and distributions of source parameters. We use a model for the Milky Way that accounts for the chemical enrichment history and calibrated on the latest APOGEE survey \citep{Majewski+2017,Frankel+2018}, whereas most others did not consider the effect of metallicity in detail (see however \citealp{Lamberts+2019, Sesana+2020}). We provide a full treatment of the eccentricity of detectable sources both for the inspiral evolution as well as gravitational wave signal during the LISA mission. Moreover, our binary population synthesis simulation is the most extensive of its kind to date and make use of the adaptive sampling algorithm STROOPWAFEL \citep{Broekgaarden+2019, Broekgaarden+2021}. Overall we simulate over 2 billion massive binaries to produce the DCO populations used in this work. We find that this large number of simulations is important to reduce the sampling noise.

All data produced in this study is publicly available on Zenodo \href{https://zenodo.org/record/4699713}{\faFileCode}\footnote{\url{https://zenodo.org/record/4699713}} as is the population used in our simulations \href{https://zenodo.org/record/4574727}{\faFileCode}\footnote{\url{https://zenodo.org/record/4574727}}. We make all code used to produce our results available in a Github repository \href{https://github.com/TomWagg/detecting-DCOs-in-LISA}{\faGithub}\footnote{\url{https://github.com/TomWagg/detecting-DCOs-in-LISA}}. In addition, the repository contains step-by-step Jupyter notebooks that explain how to reproduce and change each figure in the paper. In a companion paper, Wagg et al. (in prep), we present \href{https://legwork.readthedocs.io}{\texttt{LEGWORK}}\footnote{\url{https://legwork.readthedocs.io}}, a python package designed for making predictions for the detection of sources with LISA, which we use in this work.

Our paper is structured as follows. In Section~\ref{sec:method}, we describe our methods for synthesising a population of binaries, the variations of physical assumptions that we consider, how we simulate the Milky Way distribution of DCOs and our methods for calculating a detection rate for LISA. We present our main results in Section~\ref{sec:results}, analysing our findings for each DCO type and variation of physical assumptions. In Section~\ref{sec:discussion} we discuss these results. In Section~\ref{sec:compare_studies}, we compare and contrast our methods and findings to previous work and finish with our conclusions in Section~\ref{sec:conclusion}.