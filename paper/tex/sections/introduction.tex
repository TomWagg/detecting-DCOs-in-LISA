Since the first direct detection of gravitational waves by the LIGO scientific collaboration \citep{Abbott+2016_first_detection}, the number of black hole (BH) binaries and neutron star (NS) binaries detected by ground-based detectors has rapidly grown \citep{Abbott+2019_GWTC1,Abbott+2020_GWTC2}. These detections offer exciting insights into the endpoints of massive stars and the investigation of population statistics provides an essential tool for constraining uncertainties in binary evolution and predicting distributions of observable parameters for different double compact object (DCO) types. 

The Laser Interferometer Space Antenna (LISA, \citealp{Amaro-Seoane+2017}) presents an exciting new lens through which to view the realm of gravitational waves. In the same way that radio telescopes produce a different, but complementary, view of the sky than optical telescopes, LISA will detect a different and complementary population of gravitational wave sources. LISA will observe binaries at lower orbital frequencies than ground-based detectors ($10^{-5} \lesssim f / \unit{Hz} \lesssim 10^{-1}$) and therefore focus on the inspiral phase of stellar mass binaries rather than the merger. This will allow LISA to detect, and possibly localise a binary on the sky, far in advance of the merger, which presents an opportunity for both multimessenger detections to search for electromagnetic counterparts and multiband detections that would better constrain binary characteristics \citep[e.g.][]{Gerosa+2019}. In addition, DCOs may still have significant eccentricity in the LISA band and measurements of eccentricity may provide further constraints on binary evolution \citep[e.g.][]{Breivik+2016}, differentiate between formation channels and distinguish between DCO types. The maxmimum detectable distance for stellar mass sources in LISA is significantly lower than in ground-based detectors since the gravitational wave signal is weaker during the inspiral than at the merger. This means that LISA stellar mass sources are an excellent probe for our galaxy's history and evolution.

Traditionally, investigations into detecting stellar mass sources with LISA focus on double white dwarf (WDWD) binaries \citep{Ruiter+2010,Yu+2010,Nissanke+2012,Korol+2017,Lamberts+2018}. However, whilst NS and BH binaries are rare, they provide an excellent probe for the endpoints of massive stars.

Double black hole (BHBH) binaries are the dominant source in ground-based gravitational wave detections and several studies have tried to model the black hole mass function as well as identify the pair instability mass gap \citep[e.g.][]{Baxter+2021}. In contrast, BHBHs in LISA are not biased towards higher masses and instead will probe lower mass black holes, potentially offering insight about the presence of a lower mass gap between neutron stars and black holes.

Galactic double neutron star (NSNS) binaries have been observed with electromagnetic signals for several decades \citep[e.g.][]{Hulse+1975} and more recently the mergers of NSNS binaries with ground-based gravitational wave detectors have been observed \citep[e.g.][]{Abbott+2017_NSNS}. The detection of a NSNS in LISA in which at least one NS is a pulsar could connect these two populations as the binary could be observed from inspiral to merger. NSNS binaries are excellent sources for understanding the origin of r-process elements \citep[e.g.][]{Eichler+1989} as well as the electromagnetic counterparts to gravitational wave signal such as kilonovae \citep[e.g.][]{Metzger+2017}, short gamma-ray bursts \citep[e.g.][]{Gompertz+2020}, radio emission \citep[e.g.][]{Hotokezaka+2016} and neutrinos \citep[e.g.][]{Kyutoku+2018}.

One particularly interesting and elusive gravitational wave source is a black hole neutron star binary (BHNS). Of all the events detected by ground-based detectors, none can be confidently attributed to the merger of a black hole and a neutron star, though several events such as GW190425 and GW190814 have not been ruled out as a BHNS merger \citep{Abbott+2020_GW190425,Abbott+2020_GW190814}. Predictions for the merger rate of BHNSs range across three orders of magnitude \citep[e.g.][]{Broekgaarden+2021} so the number of detections in LISA will be important in reducing this uncertainty, thereby refining our understanding of the remnants and evolution of massive stars. These binaries are expected to have similar electromagnetic counterparts to NSNSs and so can studied in the same way. A distinctly exciting possibility is the detection of a pulsar--BH system or millisecond pulsar--BH system as these could be observed not only by gravitational wave detectors, but also radio telescopes such as MeerKAT and SKA, which will help to constrain uncertain binary evolution processes \citep{Chattopadhyay+2020}.

The detection of DCOs with LISA has been investigated in many previous studies through a combination of population synthesis and Milky Way modelling. Previous studies that investigate BHBH, BHNS and NSNS binaries, as opposed to the numerous WDWD population, are still rare. Earlier work has used a variety of population synthesis codes, Milky Way models and LISA specifications, resulting in a wide range of predictions \citep{Nelemans+2001,Liu+2009,Belczynski+2010,Liu+2014,Lamberts+2019,Lau+2020,Breivik+2020,Sesana+2020}.

We build upon previous efforts but will several important improvements. We explore the effect of varying binary physics assumptions by repeating our analysis for 15 different models and comparing the effect on the detection rate and distributions of source parameters. We use a model for the Milky Way that is dependent on the chemical enrichment history and calibrated on the latest GAIA and APOGEE surveys \citep{Frankel+2018}. In contrast to many previous works, we provide a full treatment of the eccentricity of detectable sources both for the inspiral evolution as well as gravitational wave signal during the LISA mission. Moreover, our binary population synthesis simulation is the most extensive of its kind, with 750 million binaries (one million binaries for each of 50 metallicity bins and 15 physics variations) evolved to produce the DCO populations used in this work \citep{Broekgaarden+2021}. In addition, we use the adaptive sampling algorithm STROOPWAFEL to further reduce our sampling noise \citep{Broekgaarden+2019}.

In this paper, we present the most extensive simulations to date for predictions of the detection rate and distribution of binary properties (masses, frequency, eccentricity, distance, merger time) of BHBH, BHNS and NSNS binaries formed through isolated binary evolution in the Milky Way. We explore 15 different models of physical assumptions in our population synthesis model and how the changes in these assumptions alter our results. We also discuss the effect of extending the LISA mission length and the possibility of distinguishing detections.

In Section~\ref{sec:method}, we describe our methods for synthesising a population of binaries, the variations of physical assumptions that we consider, how we simulate the Milky Way distribution of DCOs and our methods for calculating a detection rate for LISA. We present our main results in Section~\ref{sec:results}, analysing our findings for each DCO type and variation of physical assumptions. In Section~\ref{sec:discussion} we discuss these results. In Section~\ref{sec:compare_studies}, we compare and contrast our methods and findings to previous work and finish with our conclusions in Section~\ref{sec:conclusion}.