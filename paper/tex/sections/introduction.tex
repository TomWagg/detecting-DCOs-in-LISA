Since the first direct detection of gravitational waves by the LIGO scientific collaboration in 2015 \citep{Abbott+2016_first_detection}, there have been 50 gravitational wave events detected by LIGO \citep{Abbott+2019_GWTC1,Abbott+2020_GWTC2}. The investigation of double compact object (DCO) population statistics provides an essential tool for constraining uncertainties in binary evolution and predicting distributions of observable parameters for different DCO types. 

The Laser Interferometer Space Antenna (LISA, \citealp{Amaro-Seoane+2017}) provides an exciting new lens through which to view the realm of gravitational waves. LISA will observe binaries at lower orbital frequencies than LIGO ($10^{-5} \lesssim f / \unit{Hz} \lesssim 10^{-1}$) and therefore focus on the inspiral phase of stellar mass binaries rather than the merger. This will allow LISA to detect, and possibly localise a binary on the sky, far in advance of the merger, which allows for both multimessenger detections to search for electromagnetic counterparts and multiband detections that would better constrain binary characteristics \citep[e.g.][]{Gerosa+2019}. In addition, DCOs may still have significant eccentricity in the LISA band and measurements of eccentricity may provide further constraints on binary evolution \citep[e.g.][]{Vigna-Gomez+2018}, differentiate between formation channels and distinguish between DCO types.

One particularly interesting and elusive gravitational wave source is a black hole neutron star binary (BHNS). Of all the events detected by LIGO, none can be confidently attributed to the merger of a black hole and a neutron star, though several events such as GW190425 and GW190814 have not been ruled out as a BHNS merger \citep{Abbott+2020_GW190425,Abbott+2020_GW190814}.

Predictions for the merger rate of BHNSs range across three orders of magnitude \citep[e.g.][]{Broekgaarden+2021} so the number of detections in LISA will be important in reducing this uncertainty, thereby refining our understanding of the remnants and evolution of massive stars. Collating a sample of BHNS binaries will allow us to better understand whether there are electromagnetic counterparts to their gravitational wave signal such as kilonovae \citep[e.g.][]{Metzger+2017}, short gamma-ray bursts \citep[e.g.][]{Gompertz+2020}, radio emission \citep[e.g.][]{Hotokezaka+2016} and neutrinos \citep[e.g.][]{Kyutoku+2018}. Moreover, if the neutron star is observable (such as a pulsar), a BHNS would be ideal for measuring the Hubble constant \citep[e.g.][]{Feeney+2020} and the neutron star equation of state. A detection with LISA would allow these various signals to be measured for years before the merger and give observers ample time to prepare for the merger itself.

The detection of double compact objects with LISA has been investigated in many previous studies through a combination of population synthesis and Milky Way modelling. Several works focus only the more numerous population of double white dwarf population \citep{Ruiter+2010,Yu+2010,Nissanke+2012,Korol+2017,Lamberts+2018}, whilst others investigate the more massive BHBH, BHNS and NSNS binaries \citep{Nelemans+2001,Liu+2009,Belczynski+2010,Liu+2014,Lamberts+2019,Lau+2020,Breivik+2020,Sesana+2020}. Our work improves upon previous studies by: exploring the effect of varying binary physics assumptions with 15 models, using a model for the Milky Way that is dependent on the chemical enrichment history and providing a full treatment of the eccentricity of detectable sources. We compare these studies with each other and our work in more detail in Section.~\ref{sec:compare_studies} and provide a summary in Fig.~\ref{fig:previous_studies}.

In this paper, we present predictions for the detection rate and distribution of binary characteristics (masses, frequency, eccentricity, distance, merger time) of BHNS, BHBH and NSNS binaries formed through isolated binary evolution in the Milky Way using binaries synthesised with the \href{https://compas.science}{COMPAS} rapid population synthesis code \citep{Stevenson+2017, Vigna-Gomez+2018, Stevenson+2019} in tandem with the adaptive importance sampling algorithm STROOPWAFEL \citep{Broekgaarden+2019}. Additionally, we use a model of the Milky Way that not only includes the radial birth profiles but also the star formation and enrichment history. We explore 15 different models of physical assumptions in our population synthesis model and how the changes in these assumptions alter our results. In this study we consider three main questions: (1) how many of each type of DCO will LISA detect? (2) can we distinguish between different DCOs with LISA? (3) can we use detections in LISA to constrain uncertainties in binary evolution?

In Section~\ref{sec:method}, we describe our methods for synthesising a population of binaries, the variations of physical assumptions that we consider, how we simulate the Milky Way distribution of DCOs and our methods for calculating a detection rate for LISA. In Section~\ref{sec:results}, we present our main results for each type of DCO and variation of physical assumptions. In Section~\ref{sec:discussion} we discuss these results and finish with our conclusions in Section~\ref{sec:conclusion}.