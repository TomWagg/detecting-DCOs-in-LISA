Since the first direct observation of gravitational waves \citep{Abbott+2016_first_detection}, the number of black hole (BH) and neutron star (NS) binaries observed by ground-based gravitational-wave detectors has rapidly grown \citep{Abbott+2019_GWTC1,Abbott+2020_GWTC2,Abbott+2021_GWTC3}, offering exciting insights into the formation, lives and deaths of massive (binary) stars \citep[e.g.][]{Abbott+2021_GWTC2_inference}.

The Laser Interferometer Space Antenna (LISA, \citealp{Amaro-Seoane+2017, Colpi+2019}) will provide observations in an entirely new regime of gravitational waves. LISA will observe at lower frequencies ($10^{-5} \lesssim f / \unit{Hz} \lesssim 10^{-1}$) than ground-based detectors and so will enable the study of sources that are imperceptible by ground-based detectors, such as the mergers of supermassive black holes and extreme mass-ratio inspirals \citep[e.g.][]{Begelman+1980, Klein+2016}. Moreover, this frequency regime is also of interest for the detection of \textit{local} stellar-mass double compact objects (DCOs) millions of years before their merger. This presents an opportunity for both multi-messenger detections to search for electromagnetic counterparts and multiband gravitational-wave detections that can help to constrain binary characteristics \citep[e.g.][]{Sesana+2016, Gerosa+2019}. In addition, LISA will be able to measure the eccentricities of DCOs, which may yield further constraints on binary evolution, differentiate between formation channels and distinguish between DCO types \citep[e.g.][]{Nelemans+2001, Breivik+2016, Antonini+2017, Rodriguez+2018}. Unlike ground-based detectors, LISA only detects stellar-mass sources in local galaxies, with the majority residing in the Milky Way. These sources could be used as a probe for the structure of our galaxy \citep[e.g.][]{Korol+2019}.

Traditionally, predictions about the detection of stellar-mass sources with LISA focus on double white dwarf (WDWD) binaries, as they are abundantly present in our galaxy and are expected to be the dominant source of stellar-mass binaries that are detectable by LISA \citep{Nelemans+2001,Ruiter+2010,Yu+2010,Nissanke+2012,Korol+2017,Lamberts+2018}. More recently, interest has grown in the detection of NS and BH binaries. Although they are more rare, LISA detections of these sources are potentially valuable for learning more about the evolution and endpoints of massive stars. In this paper we focus on making LISA predictions for double black hole binaries (BHBHs), black hole neutron star binaries (BHNSs) and double neutron star binaries (NSNSs).

The detection of NSNSs in LISA could improve our understanding of many phenomena. Galactic NSNSs have been observed with electromagnetic signals for several decades (e.g. \citealp{Hulse+1975, Antoniadis+2016}, see also refs.\ in \citealp{Tauris+2017}) and more recently the mergers of NSNS binaries have been detected with ground-based gravitational-wave detectors \citep[e.g.][]{Abbott+2017_NSNS}. A LISA detectable NSNS with a pulsar component close to merger would be ideal for connecting these populations, as the binary could be observed from inspiral to merger. NSNS (and possible BHNS) binaries are useful sources for understanding the origin of r-process elements \citep[e.g.][]{Eichler+1989} as well as the electromagnetic counterparts to gravitational-wave signals, such as kilonovae \citep[e.g.][]{Li+1998, Metzger+2017}, short gamma-ray bursts \citep[e.g.][]{Berger+2014}, radio emission \citep[e.g.][]{Hotokezaka+2016} and neutrinos \citep[e.g.][]{Kyutoku+2018}.

BHBHs in the Milky Way present a greater observational challenge. To date, no BH has been observed in a BHBH binary in the Milky Way, and so LISA could provide the first detection of a Galactic BHBH. The only confirmed BHs in our galaxy have been discovered as components of X-ray binaries with companion stars \citep[e.g.][]{Bolton+1972,Webster+1972}. These detections have observed BHs with masses mainly constrained between $5$ and $10 \unit{M_\odot}$ \citep{Corral-Santana+2016}, a stark contrast to the more massive BHs observed with LIGO/Virgo that tend to contain at least one BH with a mass greater than $10 \unit{M_{\odot}}$ \citep{Abbott+2020_GWTC2}. These observations of X-ray binaries suggest the presence of a lower mass gap (from $2$-$5 \unit{M_{\odot}}$) in which there are no strong candidates for either black holes or neutron stars \citep{Ozel+2010,Farr+2011} but the gap's existence remains an open question \citep[e.g.][]{Kreidberg+2012, Mandel+2020}. Recently there has also been increased discussion over the maximum BH mass in our galaxy, with the claims of a $70 \unit{M_{\odot}}$ BH \citep{Liu+2019} which has subsequently been challenged (\citealp{El-Badry+2020, Abdul-Masih+2020, Shenar+2020,Eldridge+2020}, see also \citealp{Liu+2020}) and revised measurements of the mass of Cygnus X-1 \citep{Miller-Jones+2021}. A sample of BHBHs detected with LISA could possibly help to constrain the BH mass distribution.

One particularly interesting class of potential LISA sources is BHNSs. With the recent detection of two BHNSs by the LIGO scientific collaboration, the existence of these DCOs has been confirmed \citep{TheLIGOScientificCollaboration+2021}. However, with only two detections (not including the low-confidence candidate GW190426 \citealt{Abbott+2020_GWTC2} or GW190425, GW190814 and GW190917 which have not been ruled out as BHNSs \citealt{Abbott+2020_GW190425,Abbott+2020_GW190814, GWTC_2_1} ) and no electromagnetic counterparts, the formation rate and properties of BHNSs are still uncertain. Current predictions for the merger rate of BHNSs range across three orders of magnitude \citep[e.g.][]{Abadie+2010, Mandel+2021} so the number of detections in LISA will be important in reducing this uncertainty, thereby refining our understanding of the remnants and evolution of massive stars. Similar to NSNSs, these binaries are also expected to have electromagnetic counterparts. A distinctly exciting possibility is the detection of a pulsar--BH system or millisecond pulsar--BH system \citep{Narayan+1991, Pol+2021}. These systems could be observed not only by LISA, but also radio telescopes such as MeerKAT and SKA, which would help to improve the measurement of individual system parameters and to constrain uncertain binary evolution processes \citep[e.g.][]{Pfahl+2005,Chattopadhyay+2020}.

For the purposes of this investigation, we consider the `classical' isolated binary evolution channel \citep[e.g.][]{Tutukov+1973,Tutukov+1993,Smarr+1976,Srinivasan+1989,Kalogera+2007,Belczynski+2016} in which double compact objects are formed following common-envelope ejection or a phase of highly non-conservative mass transfer \citep{Heuvel+2011, vandenHeuvel+2017}. We do not, however, account for several alternative proposed formation channels, which could affect the rate and distribution of detectable NS and BH binaries in LISA. These channels include: dynamical formation in dense star clusters \citep[e.g.][]{Sigurdsson+1993,PortegiesZwart+2000,Miller+2009,Rodriguez+2015}, young/open star clusters \citep[e.g.][]{Ziosi+2014, DiCarlo+2020, Rastello+2020, Rastello+2021} and (active) galactic nuclei discs \citep[e.g.][]{Morris+1993, Antonini+2016, McKernan+2020}, isolated (hierarchical) triple evolution involving Kozai-Lidov oscillations \citep[e.g.][]{Stephan+2016, Silsbee+2017,Antonini+2017, Toonen+2020},  and chemically homogenous evolution through efficient rotational mixing \citep[e.g.][]{deMink+2009,Mandel+2016,Marchant+2016,Marchant+2017,duBuisson+2020}.

In this paper, we present models for the detection rate and distribution of binary properties (masses, frequency, eccentricity, distance, merger time) of BHBH, BHNS and NSNS binaries formed through isolated binary evolution in the Milky Way. We explore the effect of varying physical assumptions in our population synthesis model on our results as well as discuss the effect of extending the LISA mission length and the prospects for distinguishing DCO detections from the WDWD background.

Earlier work on BHBHs, BHNSs and NSNSs in LISA has used a variety of population synthesis codes, Milky Way models and LISA specifications, resulting in a wide range of predictions \citep{Nelemans+2001,Liu+2009,Belczynski+2010,Liu+2014,Lamberts+2019,Lau+2020,Breivik+2020,Sesana+2020}. We build upon previous efforts but with several important improvements. We explore the effects of varying binary physics assumptions by repeating our analysis for \nModels{} different models and comparing the effects on the detection rate and distributions of source parameters. We use a model for the Milky Way that accounts for the chemical enrichment history and is calibrated on the APOGEE survey \citep{Majewski+2017,Frankel+2018}, whereas most others did not consider the effect of metallicity in detail (see however \citealp{Lamberts+2019, Sesana+2020}). We provide a detailed treatment of the eccentricity of detectable sources, both for the inspiral evolution as well as gravitational wave signal during the LISA mission. Moreover, our binary population synthesis simulation is the most extensive of its kind to date and makes use of the adaptive sampling algorithm STROOPWAFEL \citep{Broekgaarden+2019, Broekgaarden+2021}. Overall we simulate over 2 billion massive binaries to produce the DCO populations used in this work. We find that this large number of simulations is important to reduce the sampling noise even when using adaptive importance sampling.

All data related to the predictions made in this study are publicly available on Zenodo at \citet{Wagg+2021_zenodo}, as are the populations used in our simulations at \citet[][BHBH]{Broekgaarden:2021-zenodo-BHBH} \citet[][BHNS]{Broekgaarden:2021-zenodo-BHNS} and  \citet[][NSNS]{Broekgaarden:2021-zenodo-NSNS}. We make all code used to produce our results available in a Github repository \href{https://github.com/TomWagg/detecting-DCOs-in-LISA}{\faGithub}\footnote{\url{https://github.com/TomWagg/detecting-DCOs-in-LISA}}. In addition, the repository contains step-by-step Jupyter notebooks that explain how to reproduce and change each figure in the paper. In a companion paper, \citet{Wagg+2021}, we present \href{https://legwork.readthedocs.io}{\texttt{LEGWORK}}\footnote{\url{https://legwork.readthedocs.io}}, the \textbf{L}ISA \textbf{E}volution and \textbf{G}ravitational \textbf{W}ave \textbf{Or}bit \textbf{K}it, a python package designed for making predictions for the detection of sources with LISA, which we use in this work.

Our paper is structured as follows. In Section~\ref{sec:method}, we describe our methods for synthesising a population of binaries, the variations of physical assumptions that we consider, how we simulate the Milky Way distribution of DCOs and our methods for calculating a detection rate for LISA. We present the main results for our fiducial model in Section~\ref{sec:results}, before exploring the variations in the detectable population when changing our physical assumptions in Section~\ref{sec:variations}. In Section~\ref{sec:discussion} we discuss these results. In Section~\ref{sec:compare_studies}, we compare and contrast our methods and findings to previous work and finish with our conclusions in Section~\ref{sec:conclusion}.