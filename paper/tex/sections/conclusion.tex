We provide predictions for the detection rate and population properties of LISA detectable BHBH, BHNS and NSNS.
To this end, we use the rapid population synthesis code COMPAS to simulate over two billion massive binaries, to explore the effect of varying underlying physics assumptions that represent the most common uncertainties in binary physics. We use an new empirically-informed analytical model to distribute the resulting BHBH, BHNS and NSNS populations in a Milky-Way like galaxy based on their birth metallicity, in order to estimate and investigate the LISA detectable population of BH and NS binaries.

Our main conclusions can be summarised as:
\begin{enumerate}
    \item \textbf{Detections:} We predict 30-300 detections in a 4-year LISA mission, across all our simulations for varying physics assumptions. Although the number of detections per type can vary by about 2 orders of magnitude, we find that the total detection rate is fairly robust, among the variations we have considered.
    
    Specifically, our fiducial model predicts a total of $124 \pm 11$ detections and out of these we find about $\BHBHFourYear{}\pm 9$ BHBHs, $\BHNSFourYear{}\pm 6$ BHNSs and $\NSNSFourYear{}\pm 3$ NSNSs. The errors quoted here are the $1$-$\sigma$ Poisson uncertainties resulting from the random initialisation of the Milky Way (see Table~\ref{tab:detection_rates}).
    
    \item \textbf{Benefits of extended LISA mission} Increasing the LISA mission length to 10 years results increases the number of detections to about 50-500 detections, a 60\% increase, because the number of detections scale approximately as $T_{\rm obs}^{0.5}$.
%    
    However, the real benefit is the improvement of the characterisation of the sources, since the relative error on the frequency derivative (which dominates the relative error in the chirp mass) scales as $T_{\rm obs}^{-2.5}$  for stationary sources (Eq.~\ref{eq:f_orb_dot_unc}).
    
    We find an increase of a factor 2.4 for the number of systems with chirp masses that can be measured better than 10\%. 
%    
    The number of systems with a sky localisation better than one degree increases by about 50\%.
%    
    The number of sources that can be unambiguously distinguished from WDWDs increases by almost a factor 2 (see Section~\ref{sec:WDWD_distinguish}).
    
    \item \textbf{Probing the black hole mass distribution and the lower mass gap:} We expect LISA to predominantly detect lower mass BHs (with 90\% of BHBH and BHNSs having BH masses lower than $11 \unit{M_\odot}$ in our fiducial simulations) in stark contrast to current ground-based detectors which are heavily biased towards high mass systems. This implies that LISA can potentially make important contributions to the debate about the existence of a lower mass gap (\citealt{Shao+2021} and see our Fig.~\ref{fig:lower_mass_gap_variation}).
    
    \item \textbf{Eccentricity distribution:} We find that for all DCO types a large fraction of detectable systems still have nonzero eccentricities ($e = 0.01$) when entering the LISA band, unlike what is expected for the more numerous WDWD LISA population. In particular, we find that this is the case for the vast majority of BHBHs and NSNSs and nearly half of BHNSs. Furthermore, over a fifth of detectable BHBHs have eccentricities large enough such that their primary gravitational wave emission occurs in a higher harmonic ($e > 0.3$).
    
    % We predict that for the observed population \BHBHNotCirc{}(\BHBHHighlyEccentric{}) of BHBHs, \BHNSNotCirc{}(\BHNSHighlyEccentric{}) of BHNSs and \NSNSNotCirc{}(\NSNSHighlyEccentric{})\% of NSNSs have eccentricity of $e > 0.01(0.3)$ at the start of the LISA mission.
    \item \sout{\textbf{Physics variations:} For BHBHs, we find that the detection rate is largely unaffected by changes in underlying physics assumptions, except for the optimistic CE scenario and increased Wolf-Rayet winds. Conversely, the BHNS detection rate varies widely with physics variation, ranging across three orders of magnitude. The NSNS detection rate shows some variation but only shows large changes under the assumption that case BB mass transfer is unstable and that core-collapse supernovae produce weaker kicks.} - We have now said most of this is point 1. We could additionally talk about \textit{which} assumptions are causing the largest variations.
    \item \textbf{Source identification:} For a 4(10)-year LISA mission we estimate that, of the detectable population, at least \BHBHNotWDWDFour{}(\BHBHNotWDWDTen{}) BHBHs, \BHNSNotWDWDFour{}(\BHNSNotWDWDTen{}) BHNSs and \NSNSNotWDWDFour{}(\NSNSNotWDWDTen{}) NSNSs will produce signals that are distinguishable from a signal produced by a WDWD. Additionally, we predict that we will be able to determine whether \BHBHEitherBHOrNSFour{}(\BHBHEitherBHOrNSTen{}) BHBHs, \BHNSEitherBHOrNSFour{}(\BHNSEitherBHOrNSTen{}) BHNSs and \NSNSEitherBHOrNSFour{}(\NSNSEitherBHOrNSTen{}) NSNSs are signals from a binary that contains a black hole.
    \item \textbf{Joint SKA-LISA detections:} We expect that \textit{if} every LISA detectable DCO contained a pulsar that is beaming towards Earth, SKA-1 would be able to detect at least 11 of these objects, whilst the increased number of detectable pulsars that could crowd the sky with SKA-2 sensitivity means that this total decreases to 6. This number is a pessimistic estimate and could be much higher if we consider that SKA and LISA could match their orbital frequency estimates to select the correct pulsar from a crowded field.
    \item \textbf{Importance of choice of MW model:} Many studies use Milky Way models that use fixed metallicity populations which are assigned irrespective of birth time or position, do not account for the inside-out growth of the thin disc and use constant star formation rates. The use of these simpler MW models may lead to an overestimation of the LISA NSNS detection rate and an underestimation of the BHNS detection rate. It may also introduce unphysical artifacts into DCO parameter distributions, particularly the mass distributions, which lead to inaccurate predictions.
\end{enumerate}