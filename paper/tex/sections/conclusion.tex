We provide predictions for the detection rate and population properties of LISA detectable BHBH, BHNS and NSNS.
To this end, we use the rapid population synthesis code COMPAS to simulate over two billion massive binaries, to explore the effect of parameters that are varied to represent the most common uncertainties in binary physics.  We use an  empirically-informed analytical model to distribute the resulting BHBH, BHNS and NSNS populations in a Milky-Way like galaxy based on their birth metallicity, in order to estimate and investigate the LISA detectable population of BH and NS binaries.

Our main conclusions can be summarised as:
\begin{itemize}
    \item \textbf{Detection rate:} We find that on average, for our fiducial model, a 4(10)-year LISA mission will detect \BHBHFourYear{}(\BHBHTenYear{}) BHBHs, \BHNSFourYear{}(\BHNSTenYear{}) BHNSs and \NSNSFourYear{}(\NSNSTenYear{}) NSNSs.
    \item \textbf{Black hole mass distribution:} We find that the black hole mass distribution for both BHBHs and BHNSs that are detectable with LISA are not skewed towards heavier masses (as in ground-based detectors), with 88\% and 91\% having masses less than $11 \unit{M_\odot}$ respectively. Moreover, we find that for LISA detectable DCOs, 35\% of BHBHs and 39\% of BHNSs have masses in the theoretical lower mass gap between NSs and BHs. This implies that LISA will be able provide conclusive evidence about the existence of a lower-mass gap. 
    \item \textbf{Eccentricity distribution:} We find that BH and NS binaries still have significant eccentricities when observable by the LISA mission, unlike the more numerous WDWD LISA population. We find that 86(20)\% of BHBHs, 46(8)\% of BHNSs and 86(12)\% of NSNSs have eccentricity of $e > 0.01(0.3)$ at the start of the LISA mission.
    \item \textbf{Physics variations:} For BHBHs, we find that the detection rate is largely unaffected by changes in underlying physics assumptions, except for the optimistic CE scenario and increased Wolf-Rayet winds. Conversely, the BHNS rate varies widely with physics variation, ranging from a mean of 4 to 138. The NSNS shows some variation but only shows large changes under the assumption that case BB mass transfer is unstable and that core-collapse supernovae produce weaker kicks.
    \item \textbf{Distinguishable sources:} For a 4(10)-year LISA mission, we estimate that of the detectable population, at least \BHBHNotWDWDFour{}(\BHBHNotWDWDTen{}) BHBHs, \BHNSNotWDWDFour{}(\BHNSNotWDWDTen{}) BHNSs and \NSNSNotWDWDFour{}(\NSNSNotWDWDTen{}) NSNSs will produce signals that are distinguishable from a signal produced by a WDWD. Additionally, we predict that we will be able to determine whether \BHBHDistinguishedFour{}(\BHBHDistinguishedTen{}) BHBHs, \BHNSDistinguishedFour{}(\BHNSDistinguishedTen{}) BHNSs and \NSNSDistinguishedFour{}(\NSNSDistinguishedTen{}) NSNSs are signals from a binary that contains a black hole.
    \item \textbf{Joint SKA-LISA detections:} We expect that \textit{if} the binary contains a pulsar that is beaming towards Earth, SKA-1 will be able to detect at least 10 LISA DCOs that contain a NS, whilst the increased number of pulsars that could crowd the sky in SKA-2 means that this total decreases to 6. This total could be much higher if we consider that SKA and LISA could match their orbital frequency estimates to select the correct pulsar from a crowded field.
    \item \textbf{Importance of choice of MW model:} Many studies use Milky Way models that use fixed metallicity populations which are assigned irrespective of birth time or position, do not account for the inside-out growth of the thin disc and use constant star formation rates. The use of these simpler MW models may lead to an overestimation of the LISA NSNS detection rate. It may also introduce unphysical artifacts into DCO parameter distributions, particularly the mass distributions, which lead to inaccurate predictions.
\end{itemize}