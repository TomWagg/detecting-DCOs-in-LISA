We provide predictions for the detection rate and population properties of LISA detectable BHBH, BHNS and NSNS.
%
We use a novel empirically-informed analytical model for the metallicity dependent star formation history of the Milky Way, calibrated against the APOGEE stellar spectroscopic survey. We use this to model Monte-Carlo realisations of the present-day BHBH, BHNS and NSNS populations in our Milky Way. 
%
For the binary population, we use the results of a large grid of simulations performed with the rapid population synthesis code COMPAS. These simulations have been optimised with the adaptive sampling algorithm STROOPWAFEL to preferentially sample NS and BH binaries. In total these comprise over two billion massive binaries that span 20 physics variations, which represent the most common uncertainties in binary physics.
%
To determine the detectability of sources with LISA we use LEGWORK package (Wagg et al.\ in prep), that was specifically developed for this purpose and is publicly available. 
%
We investigate the results expected for a 4- and 10-year LISA mission. Our main conclusions can be summarised as:
\begin{enumerate}
    \item \textbf{Total detections:} We predict 30-300 detections in a 4-year LISA mission, across all our simulations for varying physics assumptions. This increases to about 50-500 for a 10-year LISA mission. Although the number of detections per type can vary by about 2 orders of magnitude, we find that the total detection rate is fairly robust, among the variations we have considered (see Table~\ref{tab:detection_rates}).
    
     \item \textbf{Detections by type:} Specifically, our fiducial model predicts a total of $124 \pm 11$ detections and out of these we find about $\BHBHFourYear{}\pm 9$ BHBHs, $\BHNSFourYear{}\pm 6$ BHNSs and $\NSNSFourYear{}\pm 3$ NSNSs. The errors quoted here are the $1$-$\sigma$ Poisson uncertainties resulting from the random initialisation of the Milky Way (see Table~\ref{tab:detection_rates}).
    
    \item \textbf{Physics variations:}  Among the model variations we consider, we find that uncertainties related the onset of the common-envelope phase and the efficiency of ejection have a strong impact on the rate predictions. This is also true for models where we reduce the natal kicks. Finally, our models where we have increased the strength of the Wolf-Rayet winds, drastically reduces the number of systems containing a BH (see Fig.~\ref{fig:detection_rates})
    
    \item \textbf{Probing the black hole mass distribution and the lower mass gap:} We expect LISA to predominantly detect lower mass BHs (with 90\% of BHBH and BHNSs having BH masses lower than $11 \unit{M_\odot}$ in our fiducial simulations) in stark contrast to current ground-based detectors which are heavily biased towards high mass systems. This implies that LISA can potentially make important contributions to the debate about the existence of a lower mass gap (see Fig.~\ref{fig:lower_mass_gap_variation}).
    
    \item \textbf{Eccentricity distribution:} We find that for all DCO types a large fraction of detectable systems still have nonzero eccentricities ($e = 0.01$) when entering the LISA band, which can be used to distinguish them from the more numerous WDWD binaries, which are largely expected to be circular. In particular, for our fiducial model, we find that this is the case for around three quarters of detectable binaries. Furthermore, around 16\% of detectable binaries have eccentricities that are so high ($e > 0.3$) that the emission at frequencies corresponding to higher order harmonics start to dominate (see Fig.~\ref{fig:fiducial_pdf_distributions}).
    
    \item \textbf{Distinguishing from WDWD sources:} For about half of all detections we expect that we will be able to confidently determine the type of compact objects involved and this increases to 60\% for a 10-year LISA mission (see Sec.~\ref{sec:WDWD_distinguish}).
    
    \item \textbf{Chirp mass determinations:} For about 10\% of systems we expect to be able to determine the chirp mass better than 10\% and this increases to 15\% for a 10-year LISA mission (see Fig.~\ref{fig:m_c_unc}).
    
%    Based on our fiducial model we estimate that, for a 4(10)-year LISA mission, at least \BHBHNotWDWDFour{}(\BHBHNotWDWDTen{}) BHBHs, \BHNSNotWDWDFour{}(\BHNSNotWDWDTen{}) BHNSs and \NSNSNotWDWDFour{}(\NSNSNotWDWDTen{}) NSNSs will produce signals that are distinguishable from a signal produced by a WDWD. 
%    Additionally, we predict that we will be able to determine whether \BHBHEitherBHOrNSFour{}(\BHBHEitherBHOrNSTen{}) BHBHs, \BHNSEitherBHOrNSFour{}(\BHNSEitherBHOrNSTen{}) BHNSs and \NSNSEitherBHOrNSFour{}(\NSNSEitherBHOrNSTen{}) NSNSs are signals from a binary that contains a black hole (see Sec.~\ref{sec:identify_sources}).  The absolute number of systems that we can distinguish in different models changes, but the relative fraction remains roughly constant. 
    
    \item \textbf{Prospects for finding EM counterparts:} We expect about 13\% of detections with a sky localisation better than 1 degree. This fraction remains the same for a 10-year LISA mission, meaning that the number increases proportionally.
    This will be of interest for electromagnetic searches for counterparts, in particular for radio pulsar searches with SKA (see Fig.~\ref{sec:pulsar_matching}).
 
   \item \textbf{Benefits of extending the LISA mission:} The number of detections scale approximately as $T_{\rm obs}^{0.5}$, where $T_{\rm obs}$ is the mission length. Therefore, extending the LISA mission from 4- to 10-years increases the number of detections by about 60\%.
   A further important benefit is the improvement of the characterisation of the sources, since the relative error on the frequency derivative (which dominates the relative error in the chirp mass) scales as $T_{\rm obs}^{-2.5}$ for stationary sources (Eq.~\ref{eq:f_orb_dot_unc}).
   We find that the number of systems with chirp masses that can be measured better than 10\% increases by a factor of 2.4.
   In addition, the number of systems with a sky localisation better than one degree increases by a factor of 1.5.
   Overall, the number of sources that can be unambiguously distinguished from WDWDs increases by almost a factor of 2 (see Section~\ref{sec:WDWD_distinguish}).
\end{enumerate}