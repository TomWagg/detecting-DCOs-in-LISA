\subsection{Identification of GW sources}
It is important to note that, though we present predictions for the detection rates of specific DCO types, the nature of the source may not be immediately apparent from the gravitational wave signal.

\subsubsection{Distinguishing from WDWD population}
\subsubsection{Separating BHBHs, BHNSs and NSNSs}

\subsection{Matching LISA detection to radio pulsars}\label{sec:pulsar_matching}

\subsection{Caveats}\label{sec:caveats}
\todo{This section is only an outline for now}

\subsubsection{Population Synthesis Caveats}
\begin{itemize}
    \item Standard pop synth limitations
    \item Other channels
\end{itemize}

\subsubsection{Galaxy Caveats}
\begin{itemize}
    \item Systemic kicks
    \item No models for high $Z$ values in Frankel
    \item No halo or GC considered %GC don't produce NSNS well but other maybe (Kremer+??)
    \item Contribution of metal-poor stars not well known
    \item COMPAS models for naked helium stars are based on a single metallicity -- may be why NSNS rate is less dependent on MSSFR
\end{itemize}

\subsubsection{Detection Caveats}
\begin{itemize}
    \item Eccentricity uncertainty is pessimistic (doesn't actually seem to be the dominant problem)
\end{itemize}