In this section we discuss the prospects of (and methods for) identifying LISA sources (Sec.~\ref{sec:identify_sources}), the possibility of matching LISA signals to SKA detections (Sec.~\ref{sec:pulsar_matching}), the main caveats for this study (Sec.~\ref{sec:caveats}) and the possible contribution from other formation channels (Sec.~\ref{sec:other_formation_channels}). All predictions quoted in each subsection are derived for the fiducial model (model \modFid{}).

\subsection{Identification of GW sources}\label{sec:identify_sources}
It is important to note that, though we present predictions for the detection rates of specific DCO types, the nature of the source may not be immediately \edit1{(or ever)} apparent from the gravitational wave signal. LISA can detect a variety of sources, from exoplanets \citep[e.g.][]{Tamanini+2019} to common-envelopes \citep[e.g.][]{Ginat+2020, Renzo+2021} that may cause confusion. However, by far the most prominent will be the population of Galactic WDWDs detectable with LISA, which will be several orders of magnitude larger than the population of the more massive DCOs that we focus on in this paper \citep[e.g.][]{Korol+2017}. It is therefore imperative that we consider how to distinguish NS and BH binaries from this much more numerous population of sources. In addition to distinguishing them from WDWDs, we must consider how to discriminate between BHBHs, BHNSs and NSNSs themselves.

\subsubsection{Distinguishing from WDWD population}\label{sec:WDWD_distinguish}
The simplest way to check whether a source is a WDWD is to evaluate its chirp mass. The mass of a non-rotating white dwarf cannot be larger than the Chandrasekhar limit of $1.4 \unit{M_\odot}$ \citep{Chandrasekhar+1931, Hamada+1961}, so we can take the maximum chirp mass of a WDWD to be ${\sim}1.2 \unit{M_{\odot}}$. Therefore, any DCO with a chirp mass that satisfies $\mathcal{M}_c > 1.2 \unit{M_{\odot}} + \Delta \mathcal{M}_c$ must not be a WDWD (where $\Delta \mathcal{M}_c$ is the error on the chirp mass, estimated using Eq.~\ref{eq:chirp_mass_uncertainty}). We find that for the detectable population of a 4(10)-year LISA mission, \BHBHAboveMaxWDWDFourPerc{}(\BHBHAboveMaxWDWDTenPerc{})\% of BHBHs, \BHNSAboveMaxWDWDFourPerc{}(\BHNSAboveMaxWDWDTenPerc{})\% of BHNSs and \NSNSAboveMaxWDWDFourPerc{}(\NSNSAboveMaxWDWDTenPerc{})\% of NSNSs satisfy this condition. This method is not particularly effective for NSNSs since their average chirp mass, $1.17 \unit{M_\odot}$, is below the Chandrasekhar limit.

Another discriminator between WDWDs and other DCOs is eccentricity. WDWDs formed in the disc are thought to be formed mainly through isolated binary formation and have little to no eccentricity (e.g.\ \citealt{Nelemans+2001}, see however \citealt{Dosopoulou+2016a, Dosopoulou+2016b, Gosnell+2019}). This is because WDWDs formed through isolated binary evolution all experience a phase of mass transfer or a common envelope, which typically circularises the binary \citep[e.g.][]{Marsh+2004}. However, in contrast to the more massive DCOs that we study, WDWDs do not experience strong natal kicks which we find to be the main source of eccentricity. Therefore, if any system is detected with anything other than one detectable harmonic, this suggests that the system is unlikely to be a WDWD. We find that for a 4(10)-year LISA mission, \BHBHMultipleHarmonicsFourPerc{}(\BHBHMultipleHarmonicsTenPerc{})\% of BHBHs, \BHNSMultipleHarmonicsFourPerc{}(\BHNSMultipleHarmonicsTenPerc{})\% of BHNSs and \NSNSMultipleHarmonicsFourPerc(\NSNSMultipleHarmonicsTenPerc{})\% of NSNSs are detected with multiple harmonics (see also Sec.~\ref{sec:fiducial_distributions}). Both the absolute percentage and the relative improvement with an extended LISA mission is lower for the BHNSs with respect to other DCOs as we find that these BHNSs are less eccentric on average (see Fig.~\ref{fig:fiducial_pdf_distributions}d and discussion in Sec.~\ref{sec:fiducial_distributions}).

However, we should also consider that eccentric WDWDs could be formed through dynamical formation in Milky Way globular clusters \citep[e.g.][]{Willems+2007, Kremer+2018}, or with third companions \citep[e.g.][]{Antonini+2017}. This means that we cannot assume that eccentric binaries are not WDWDs unless they are detected in the Galactic plane (though even then there is a chance they were formed dynamically). We can use the sky localisation, scale height of the disc and distance to the source to estimate what fraction of eccentric sources can be localised to the Galactic plane. This condition can be written as $\sigma_\theta < \arcsin(z_{\rm plane} / D_L)$ or $D_L < z_{\rm plane}$, where we set the height of the Galactic plane, $z_{\rm plane}=0.95 \unit{kpc}$, to the scale height of the high-$\alpha$ disc. We apply this condition to find that the fraction of sources that are eccentric and localised within the disc for a 4(10)-year LISA mission are \BHBHEccInDiscFourPerc{}(\BHBHEccInDiscTenPerc{})\% for BHBHs, \BHNSEccInDiscFourPerc{}(\BHNSEccInDiscTenPerc{})\% for BHNSs and \NSNSEccInDiscFourPerc{}(\NSNSEccInDiscTenPerc{})\% for NSNSs. Note that although the fractions are the same for the 10-year mission, the absolute number of detections is still greater.

Overall, combining these methods (chirp mass, eccentricity and sky localisation) we find that for a 4(10)-year mission, LISA will detect at least \BHBHNotWDWDFour{}(\BHBHNotWDWDTen{}) BHBHs, \BHNSNotWDWDFour{}(\BHNSNotWDWDTen{}) BHNSs and \NSNSNotWDWDFour{}(\NSNSNotWDWDTen{}) NSNSs that are distinguishable from the WDWD population. Thus we will be able to confidently distinguish approximately half of all detected sources from WDWDs. This increases to roughly 60\% for a 10-year mission. We highlight that, though the overall number of LISA detections in an extended mission only increases by a factor of $\sqrt{T_{\rm obs}}$, the number of distinguishable detections increases by a greater factor since each of the more numerous sources are better measured. This further underlines the benefits of extending the LISA mission to 10 years.

\subsubsection{Discriminating between BHBHs, BHNSs and NSNSs}

The problem of discriminating between the BHBH, BHNS and NSNS populations can be more difficult than distinguishing them from WDWDs. For NSNSs, we can follow a similar method to the WDWDs (see Sec.~\ref{sec:WDWD_distinguish}) by applying our knowledge of the maximum mass of a neutron star. Following our fiducial assumption, we can take the maximum mass of a neutron star as $2.5 \unit{M_{\odot}}$ and thus the maximum chirp mass that a system can attain without one of the components being a black hole is $\mathcal{M}_{c} = 2.2 \unit{M_\odot}$. For a 4(10)-year LISA mission, the fraction of systems that are above or below this limit (and thus \textit{must} respectively contain or not contain a BH component) by more than $\Delta \mathcal{M}_c$ is \BHBHEitherBHOrNSFourPerc{}(\BHBHEitherBHOrNSTenPerc{})\% for BHBHs, \BHNSEitherBHOrNSFourPerc{}(\BHNSEitherBHOrNSTenPerc{})\% for BHNSs and \NSNSEitherBHOrNSFourPerc{}(\NSNSEitherBHOrNSTenPerc{})\% of NSNSs, which in terms of absolute detections is \BHBHEitherBHOrNSFour{}(\BHBHEitherBHOrNSTen{}) for BHBHs, \BHNSEitherBHOrNSFour{}(\BHNSEitherBHOrNSTen{}) for BHNSs and \NSNSEitherBHOrNSFour{}(\NSNSEitherBHOrNSTen{}) for NSNSs.

For separating the BHBH and BHNS population one could do so probabilistically given the properties that are measured, particularly the orbital frequency and eccentricity, since these distributions are fairly different for the two DCO types (see Fig.~\ref{fig:fiducial_pdf_distributions}). This method would pose a challenge, however, as it would likely only indicate which type was more likely rather than discriminate between them with strong evidence.

Another possible solution would be the existence of electromagnetic counterparts to the gravitational wave signal. In Section~\ref{sec:pulsar_matching} we consider the possibility of detecting a pulsar within a BHNS or NSNS system. This could be used to identify the type of the source.

\subsection{Matching LISA detections to pulsars with the SKA}\label{sec:pulsar_matching}
Since the vast majority of the LISA detectable population of DCOs will not merge for many years, the main type of electromagnetic counterpart for this population is pulsars. Therefore, for this section we focus only on BHNSs and NSNSs since no BHBH system will contain a pulsar. The joint detection of a binary pulsar with LISA and the Square Kilometre Array (SKA, \citealt{Dewdney+2009}) would not only help to constrain the parameters of the binary, but also enable investigation of other compact object physics. A pulsar(PSR)+BH can provide stringent tests of theories of gravity, in particular the ``No-hair theorem'' \citep{Keane+2015}. Alternatively, an ultrarelativstic PSR+NS system could be used to measure the neutron star equation of state up to an order of magnitude more accurately than other proposed observational constraints \citep{Kyutoku+2019, Thrane+2020}.

We estimate that on average, given the number of detectable pulsars and the SKA sky area, each pulsar in the SKA occupies a region with an angular resolution of $\sigma_{\theta} < 1.3^\circ$ or $0.7^\circ$ for SKA-1 and SKA-2 respectively (see Appendix~\ref{app:ska_area}). Therefore, any DCOs containing NSs localised by LISA with an angular resolution lower than these values can be unambiguously matched to the radio signal in the SKA. By considering Fig.~\ref{fig:ang_res}, approximately $11$ and $6$ (for SKA-1 and SKA-2) DCOs will satisfy this constraint.

If there is more than one pulsar in the region given by the LISA sky localisation, one can compare the measured parameters of the system in LISA and the SKA. Both the SKA and LISA will measure the orbital frequency to high precision, as well as the time derivative of the frequency and chirp mass to a lesser precision, of each of these systems. Therefore, one could perform a targeted search with the SKA that checks the sky location given by LISA, only looking for binary pulsars with orbital frequencies within the uncertainties. If there was \textit{still} more than one possible pulsar one could also check against the chirp mass. In this way, we expect it will be possible to get a joint detection between the SKA and LISA even when the sky area implied by the LISA detection contains more than one pulsar.

In order to assess the efficacy of this method, we would need to know the probability that two random binary pulsars would have orbital frequencies and chirp masses close enough that one could not tell which pulsar matches the LISA detection. This would require simulating the SKA population of pulsars with a code such as PSRPOPPy \citep{Bates+2014} to find the frequency and chirp mass distribution, which is beyond the scope of this paper. However, the uncertainty in the orbital frequency of a binary on the detection threshold (${\rm SNR} = 7$) for a 4-year LISA mission is $2.5 \times 10^{-9} \unit{Hz}$ and $1.0 \times 10^{-9} \unit{Hz}$ for a 10-year mission (calculated using Eq.~\ref{eq:f_orb_unc}). Therefore, we expect that the SKA could likely isolate the correct binary pulsar to match to a LISA detection even when several are present in the sky localisation region.

\subsection{Caveats}\label{sec:caveats}

Our predictions are subject to various uncertainties which can be broadly divided into two different categories: those arising from the progenitor models for the population of DCOs and those arising from the choices we have made when placing these DCOs in our model for the Milky-Way. Although we are unable, at present, to evaluate the impact of all these uncertainties, the reader should nevertheless keep in mind that they are likely very substantial. Most of these concerns are not unique to these study, but apply to most of the predictions available in present literature. We highlight a few main concerns. 

\paragraph{Progenitor models} Our binary-star progenitors models have been computed with a rapid population synthesis code (see Sec.~\ref{sec:COMPAS_explained}). This code relies on approximate parametric prescriptions for the stellar evolutionary tracks of single tracks and simple algorithms to mimic the effects of evolutionary and binary interaction  processes. Even though we explicitly consider the impact of some of the main physics uncertainties (see Sect.~\ref{sec:variation_assumptions}) the list of variations that we considered is far from exhaustive. Moreover, it is by no means guaranteed that the parametric prescriptions used in this code lead to realistic results, even when varying the values of the parameters to their extremes. We stress in particular the uncertainties affecting our most massive progenitor models. Observational constraints are scarce for high mass stars and practically non-existent for the rapid evolutionary phases \citep[e.g.][]{Langer2012, Mapelli+2021}. This is even more true for the evolution of massive stars at low metallicity. In addition to our limited understanding of massive stars, we note that the rapid population synthesis code, such as the one employed to compute the models used in this study, rely on extrapolations of the original fitting formulae to approximate the evolutionary tracks for these higher mass progenitors \citep{Hurley+2000,Hurley+2002}. \edit1{A further caveat is whether the population synthesis predictions used in our study realistically describe the population of compact objects. \citet{Broekgaarden+2021b} shows that none of the physics variations that we use can be excluded at present based on the overall constraints on the GW rates, but this does not mean that they are accurate. For example, at present it is unclear whether these models reproduce the features in the inferred mass distribution \citep[van Son et al.\ in prep.][]{Farah+2021,TheLIGOScientificCollaboration+2021,Li+2021,Veske+2021,Tiwari2022,Edelman+2022}.}
 
\paragraph{Populating the Milky Way} Our Milky Way model is semi-empirical and has been calibrated based on observations. Unfortunately, the early evolution of the (metallicity dependence of the) star formation history is poorly constrained. We do not expect this to be a major concern, as most of the double compact objects have relatively short delay times of less than $2 \unit{Gyr}$ (see Fig.~\ref{fig:fiducial_pdf_distributions}e), but this is a caveat that should be kept in mind. Furthermore, to estimate the rate of detectable systems, we rely on normalisation choices (e.g.\ how many detectable double compacts are formed per unit of star formation). This depends heavily on the initial mass function, as low mass stars account for most of the mass while high mass stars are the progenitors of double compact objects. Further choices, such as the binary fraction and the initial distributions of binary parameters also play a lesser but probably still significant role \citet[e.g.][]{deMink+2015, Chruslinska+2017, Klencki+2018}. 

We also note that, for reasons of computational efficiency, we have not accounted for the spatial velocities resulting from the Blaauw-Boersma kick \citep{Blaauw+1961,Boersma1961}. In test simulations we find that accounting for this spreads out the population (increasing the typical height above the Galactic plane and Galactocentric radius), but we find that the impact on the rate is limited. In light of the other much larger uncertainties, we felt that this was justified (see however, e.g., \citealt{Brandt+1995, Abbott+2017_GW170817_progenitor}). \edit1{We have further ignored a possible contribution coming from the Galactic halo, as \citet{Sesana+2020} estimates this is not significant compared to the contributions from the Galactic bulge and discs}. However, this may not be true for other formation channels other than those we have considered here.

\edit2{\paragraph{Correspondence with ground-based detections} The models used in this paper were shown to be consistent with the inferred merger rates from the GWTC-3 catalogue available at the time of submission (for example the BHBH rate was quoted as $16-130 \unit{Gpc^{-3}}{yr^{-1}}$ in section IV.A of version 1 and 2 on the arXiv in Nov 2021 \citealt{GWTC-3-previous}).
We note, however, that the upper limit on the rate has come down by a factor two since, after the completion and submission of our work. The inferred BHBH merger rate is now quoted as $16-61 \unit{Gpc^{-3}}{yr^{-1}}$ in the final version of the GWTC-3 paper that appeared in Feb 2022 \citep{Abbott+2021_GWTC3}. This means that some model variations may over predict the current rates.}

\subsection{Other formation channels}\label{sec:other_formation_channels}

In this paper we considered the formation of NS and BH binaries formed via isolated binary evolution, through the classical CE channel, the stable mass transfer channel and variations on these (see Fig.~\ref{fig:formation_channels}). We did not consider further possible contributions from other formation channels, which may play a role.

We highlight the possible role of dynamical formation in globular clusters. \citet{Kremer+2018} predict, for a nominal 4-year LISA mission, that 21 sources will have SNR $> 7$, of which 7 are BHBHs, 0 are BHNSs and 1 is a NSNS \citep[see Table~1][]{Kremer+2018}. This is significantly lower than the rates we predict for nearly every model variation. If true, this would mean that formation through isolated binary formation will dominate the LISA detections. 

\citet{Banerjee+2020} investigates formation of LISA detectable BHBHs in young massive and open stellar clusters and estimates approximately 128 BHBHs with SNR $>5$ in a 5-year LISA mission \citep[see Table~1, Column 9][]{Banerjee+2020}. Although this is similar to the number we predict for our fiducial model, we note these authors adopt a threshold SNR required for a detection that is lower and a mission length is slightly longer than what is typically assumed (i.e. SNR $>7$ and 4 years, as we have also adopted in our work). We expect that, after correcting for this and making a fair comparison, our fiducial model predicts about twice as many detections.

The contribution of triples systems \citep[e.g.][]{Antonini+2017}, or even higher order multiple systems \citep[e.g.][]{Vynatheya+2021} will likely also be of interest, in particular for the formation of eccentric sources. We are, however, not aware of specific predictions for the detection rates that we can compare to directly.

