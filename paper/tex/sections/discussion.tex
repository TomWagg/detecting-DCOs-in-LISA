\subsection{Identification of GW sources}
It is important to note that, though we present predictions for the detection rates of specific DCO types, the nature of the source may not be immediately apparent from the gravitational wave signal.

\subsubsection{Distinguishing from WDWD population}
\subsubsection{Separating BHBHs, BHNSs and NSNSs}

\subsection{Matching LISA detection to radio pulsars}\label{sec:pulsar_matching}
Since the vast majority of the population will not merge for many years, the main type of electromagnetic counterparts for the LISA Galactic DCO population is pulsars. Therefore, for this section we focus only on BHNSs and NSNSs since no BHBH system will contain a pulsar. In this section we investigate how we can use SKA to match pulsars to LISA signals with some back-of-the-envelope estimates.

First, we can consider how many pulsars SKA is likely to detect. \citet{Keane+2015} uses PSRPOPPy \citep{Bates+2014} to simulate the Milky Way pulsar population. They find that for SKA-1, approximately $10000$ pulsars will be discovered. The second phase of SKA should also have been approved and in operation by the time of the LISA mission and this phase would yield a total of $24000$-$30000$ pulsars \citep{Keane+2015}. Moreover, we are only interested in pulsars that are part of a binary system. We can estimate the pulsar binary fraction as the fraction of known pulsars that are in binaries using the ATNF Pulsar Catalogue\footnote{\url{https://www.atnf.csiro.au/research/pulsar/psrcat}} \citep{Manchester+2005}. $290$ of the $2872$ currently known pulsars are in binary systems and thus we can estimate the binary fraction of pulsars as $10\%$.

Next, we find the total number of pulsars SKA will detect in a patch on the sky. The total sky area that the SKA mission covers is approximately $7200 \unit{deg^2}$, which is calculated based on SKA-mid covering Galactic latitudes of $\abs{b} < 10^\circ$ and SKA-low covering $\abs{b} < 5^\circ$ \citep{Keane+2015}. If we assume that the pulsars are found uniformly across the sky, this means that roughly $0.14$ and $0.42$ binary pulsars are expected per square degree for SKA-1 and SKA-2 respectively. Note that this assumption is not realistic as pulsars will tend to be far more concentrated in the Galactic centre but we use it to provide an upper bound on these estimates.

Given these estimates, and by considering the last panel of Fig.~\ref{fig:fiducial_pdf_distributions}, approximately $10$ and $6$ (for SKA-1 and SKA-2) DCOs containing NSs will be localised well enough that \textit{if} the NS is a pulsar, SKA can unambiguously match it to the radio signal.

Furthermore, both SKA and LISA will measure the frequency, frequency derivative and chirp mass of each of these systems. Therefore, they can independently calculate these parameters and use them to match a GW signal to a radio signal. In this way, there can be more than one pulsar within the sky localisation provided to LISA and it is still possible to match to the correct pulsar.

\subsection{Caveats}\label{sec:caveats}
\textit{Population synthesis limitations:} As with any study involving a population synthesis code, our results rely on uncertain stellar physics and the use of approximate fitting formulae. We cannot use detailed stellar evolution codes to produce such a large sample of DCOs in a reasonable amount of time. Therefore COMPAS uses fitting formulae and approximate prescriptions based on (sometimes limited) grids of detailed models to describe the evolution of binary stars. This means that some of the finer points of evolution may be approximated but generally the final product of the stars on a population as a whole is reasonably close to the true result. In addition, much of the underlying physics is uncertain, such as the common envelope evolution and mass transfer physics. We attempt to understand the importance of these assumptions by creating many different physics variations. However, it is not reasonable to change every possible parameter (especially since many parts of the code have fixed assumptions) and therefore our results are subject to the accuracy of the assumptions made within the COMPAS code.

\textit{Underlying helium star models:} One major weakness is that the \citet{Hurley+2000} fitting formulae for the evolution of helium stars are based on a grid of models from $0.3 \unit{M_{\odot}}$ to $10 \unit{M_{\odot}}$, for a single metallicity ($Z= 0.02$) and thus the formulae have no metallicity dependence and are extrapolated for higher masses. A more detailed set of models in this regime could lead to large changes in the evolution of naked helium stars, a common progenitor of DCOs, and thus affect the detection rate of DCOs.

\textit{Limited metallicity range:} Another limitation of the stellar evolution fitting formulae that COMPAS uses is that they are limited to a metallicity range of $10^{-4} \le Z \le 0.03$ and should not be extrapolated outside this region. On the scale of the Universe (more relevant for LIGO predictions), this does not usually pose a significant problem as the population is usually fairly low metallicity. However, for local GW detection in the Milky Way (based on the metallicity relation in \citet{Frankel+2018}), the metallicity distribution can extend as far as $10^{-5} \le Z \le 0.06$, with a significant fraction of formation occurs past $Z = 0.03$. Therefore, for our study we had to reassign any metallicities outside of COMPAS' range. For any metallicity below the minimum, we places it in our lowest metallicity bin. For any metallicity above the minimum we placed it uniformly randomly in one of the top 5 highest bins (since using a single bin for many binaries led to unphysical artifacts in our results).

\textit{Other formation channels:} We also note that our findings are only the result of a single formation channel (isolated binary formation). We do not consider other channels such as dynamical formation or chemically homogeneous evolution, which could increase the detection rate and alter the parameter distributions. For instance, \citet{Kremer+2018} showed that around $21$ systems could be detected in Milky Way globular clusters through dynamical formation and thus different channels can still contribute significantly to the detection rate.

\textit{Halo and globular clusters:} Moreover, our model for the Milky Way, though more extensive than many previous studies, does not consider the contributions from the Galactic halo or globular clusters. \citet{Lamberts+2018} found that the halo's contribution to the detection rate was minimal and, since the metallicity distribution of the halo is uncertain, we did not include it in our galaxy model. The impact of globular clusters would have required a more detailed look into dynamical formation that was beyond the scope of this paper but we again highlight the work of \citet{Kremer+2018} that investigated these rates.

\textit{Systemic kicks:} Another important consideration about our galaxy is that we do not include the effect of systemic kicks on the final location of the sources. This would require integrating the orbital evolution of the millions of binaries in our sample and thus was not computationally reasonable to include. We investigated the effect of kicks for a small grid of binaries and found that though they would result in a more spread out distribution within the galaxy (with a smaller concentration in the galaxy centre), the overall distribution of positions would be relatively unchanged and very few sources have strong enough kicks to reach escape velocity for the Milky Way.

\textit{Eccentricity measurement uncertainty:} As noted in Sec.~\ref{sec:ecc_unc}, the method that we use to determine the eccentricity uncertainty is pessimistic as it requires each harmonic to be individually detectable. In reality this may not be necessary depending on the efficacy of matched-filter analysis of LISA data. For an eccentric source to have been detected within the LISA data, several harmonics would already have to have been matched as the same source. This could be done by looking in the same region of the sky for signals with similar chirp masses and distances to the most detectable harmonic in order to find other harmonics that are below the regular detection threshold. This would allow one to refine the measurement of the eccentricity uncertainty much further by comparing the many different harmonics. Therefore, the eccentricity uncertainty that we calculate in this study is a pessimistic estimate. Smaller eccentricity uncertainties would have two main effects on our results. Firstly, the chirp mass error would decrease slightly in the cases where it is dominated by the eccentricity uncertainty, however it is mainly dominated by the frequency derivative uncertainty since most sources are essentially stationary and so have extremely small chirps. Secondly, it would improve our ability to distinguish between WDWDs and these higher mass DCOs. However, until we know more about how LISA will search for eccentric sources, we rely upon our pessimistic estimates.